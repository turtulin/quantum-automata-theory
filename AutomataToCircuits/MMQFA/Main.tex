\section{Measure-Many One-Way Quantum Finite Automaton to Circuit}
\label{sec:mmqfa-to-circuit}


The circuit-level translation of a \gls{mm-1qfa} follows many of the same principles used for the measure-once case, with a crucial difference: instead of deferring measurement until the end of the computation, a projective measurement is performed after each input symbol is processed. This repeated measurement model introduces non-unitary branches into the computation, enabling the automaton to halt early—either accepting or rejecting—based on intermediate outcomes. As a result, the circuit must incorporate mid-computation measurements, conditional halting logic, and branching structure, reflecting the \gls{mm-1qfa}'s hybrid quantum-classical behaviour.

This section describes how to compile a \gls{mm-1qfa} into a quantum circuit that correctly reproduces its acceptance semantics, with attention to managing measurement timing, decomposing the three-outcome measurement structure, and ensuring termination guarantees for all input strings.

\subsection{Mapping Automaton Components to Circuit Elements}
\label{sec:mmqfa-mapping}

The \gls{mm-1qfa} model is defined by the tuple
\[
M = (Q, \Sigma, \delta, q_0, Q_{\text{acc}}, Q_{\text{rej}}),
\]
as introduced in Section~\ref{sec:mmqfa}. Compared to the \gls{mo-1qfa} case, the key structural difference lies in the measurement strategy: whereas a \gls{mo-1qfa} performs a single final measurement, a \gls{mm-1qfa} applies a projective measurement after every symbol.

The first compilation steps closely follow those in Section~\ref{sec:moqfa-to-circuit}, and only diverge when handling measurements:

\subsubsection*{State Register and Qubit Allocation}
The internal state set $Q$ is represented using $n = \lceil \log_2 |Q| \rceil$ qubits, encoding each classical state $q \in Q$ as a computational basis state $\ket{q}$. This is identical to the measure-once case.

\subsubsection*{Initialisation}
The quantum register is initialised in the basis state corresponding to the initial state $q_0 \in Q$. If $q_0$ is a computational basis vector under the chosen encoding, this requires only a few $X$ gates to flip the default all-zero state to $\ket{q_0}$.

\subsubsection*{Symbol-Dependent Unitary Evolution}
Each input symbol $\sigma \in \Sigma$ is associated with a unitary matrix $U_\sigma$ acting on the $n$-qubit register. The operator $U_\sigma$ is applied before measurement at each step. These matrices are later compiled into native gates as in the \gls{mo-1qfa} case.

\subsubsection*{Repeated Measurement}
After each unitary $U_\sigma$ is applied, a projective measurement is performed to decide whether the computation halts. The measurement is defined by three mutually orthogonal subspaces corresponding to:
\begin{itemize}
  \item the accepting states $Q_{\text{acc}}$,
  \item the rejecting states $Q_{\text{rej}}$, and
  \item the continuing (non-halting) states $Q_{\text{non}} = Q \setminus (Q_{\text{acc}} \cup Q_{\text{rej}})$.
\end{itemize}
The corresponding projectors are:
\[
P_{\text{acc}} = \sum_{q \in Q_{\text{acc}}} \ket{q}\bra{q}, \quad
P_{\text{rej}} = \sum_{q \in Q_{\text{rej}}} \ket{q}\bra{q}, \quad
P_{\text{non}} = I - P_{\text{acc}} - P_{\text{rej}}.
\]
The outcome determines whether the computation halts (accepts or rejects) or continues to the next symbol. This makes the circuit fundamentally hybrid: unitary evolution is interrupted by measurements and conditional control logic, which must be tracked classically.

\subsubsection*{Classical Control Flow}
Unlike the \gls{mo-1qfa} case, which requires only a final measurement, the \gls{mm-1qfa} circuit must include measurement operations at each time step, as well as classical post-processing to determine whether further computation is needed. In practical implementations, this control flow may be realised via classical conditionals or early termination logic, depending on the hardware platform and circuit model (e.g., mid-circuit measurements with feedback).

\vspace{6pt}
\begin{table}[H]
\centering
\footnotesize
\begin{tabularx}{\textwidth}{>{\raggedright\arraybackslash}p{0.18\textwidth}%
                        >{\raggedright\arraybackslash}p{0.25\textwidth}X}
\toprule
\textbf{Automaton part} & \textbf{Circuit realisation} & \textbf{Explanation} \\
\midrule
$Q$ & $n$-qubit basis states & Encode each $q \in Q$ as $\ket{q}$ with $n = \lceil \log_2 |Q| \rceil$. \\

$\Sigma$ & unitaries $U_\sigma$ & Each input symbol $\sigma$ applies $U_\sigma$ to the state register. \\

$\delta$ & set $\{U_\sigma\}$ & Transition operators, later decomposed into native gates. \\

$q_0$ & initial state $\ket{q_0}$ & Prepared from $\ket{0}^{\otimes n}$ using $X$ gates if necessary. \\

$Q_{\text{acc}}$, $Q_{\text{rej}}$ & repeated three-outcome measurements & After each $U_\sigma$, apply a measurement $\{P_{\text{acc}}, P_{\text{rej}}, P_{\text{non}}\}$. Outcome determines halting or continuation. \\

Control flow & classical feedback & Conditional termination after each step based on measurement results. \\
\bottomrule
\end{tabularx}
\caption{Mapping MM-1QFA components to quantum-circuit constructs.}
\label{tab:mmqfa-mapping}
\end{table}

\subsection{General Compilation Algorithm}
\label{sec:mmqfa-algorithm}

The compilation of a \gls{mm-1qfa} into a quantum circuit proceeds in two structured steps:
\begin{enumerate}
    \item \textbf{Template Generation:} A symbolic circuit is created to reflect the automaton's structure for input words of fixed length, using placeholders for both unitaries and measurements.
    \item \textbf{Instantiation and Execution:} For a specific word, the placeholders are replaced with the actual gates implementing the automaton's transitions, and the computation proceeds with intermediate measurements after each symbol.
\end{enumerate}

Let $F = Q_{\text{acc}}$ and $R = Q_{\text{rej}}$ denote, respectively, the sets of accepting and rejecting states. The three-outcome measurement applied after each step is defined by the orthogonal projectors:
\[
P_{\text{acc}} = \sum_{q \in F} \ket{q}\bra{q}, \qquad
P_{\text{rej}} = \sum_{q \in R} \ket{q}\bra{q}, \qquad
P_{\text{cont}} = I - P_{\text{acc}} - P_{\text{rej}}.
\]

\begin{algorithm}[H]
\caption{Template Generation for a \gls{mm-1qfa} Circuit}
\label{alg:mmqfa-template}
\begin{algorithmic}[1]
\Require Automaton $A = (Q, \Sigma, \delta, q_0, F, R)$,\; input length $L$
\Ensure Parametric circuit template with symbolic placeholders
\State $n \gets \lceil \log_2 |Q| \rceil$
\State Initialise $n$ qubits in state $\ket{q_0}$
\For{$i = 1$ \textbf{to} $L$}
   \State Insert symbolic gate $\boxed{U_{x_i}}$
   \State Insert symbolic measurement $\{P_{\text{acc}}, P_{\text{rej}}, P_{\text{cont}}\}$
\EndFor
\end{algorithmic}
\end{algorithm}

This circuit skeleton remains independent of any particular input string and can be reused for all words of the same length.

\medskip

To make the template executable, the symbolic gates are instantiated as concrete decompositions according to the input word $x = x_1x_2\dots x_L \in \Sigma^L$. At each step, the automaton either halts or continues depending on the result of the three-outcome measurement.
\begin{algorithm}[H]
  \caption{Instantiation and Execution of a Compiled \gls{mm-1qfa} Circuit}
  \label{alg:mmqfa-instantiation}
  \begin{algorithmic}[1]
  \Require Input $x = x_1x_2\dots x_L$, circuit template, gate library or decomposition scheme for each $U_\sigma$
  \Ensure Acceptance probability $p_M(x)$
  \State $p \gets 0$ \Comment{Cumulative acceptance probability}
  \State $w \gets 1$ \Comment{Cumulative continuation weight}
  \State Initialise quantum state $\ket{\psi_0} = \ket{q_0}$
  \For{$i = 1$ \textbf{to} $L$}
     \State Replace $\boxed{U_{x_i}}$ with gate decomposition for $U_{x_i}$
     \State Apply $U_{x_i}$ to current state $\ket{\psi_{i-1}}$ to obtain $\ket{\psi_i}$
     \State Compute outcome probabilities:
     \[
        p_{\text{acc}}^{(i)} = \|P_{\text{acc}} \ket{\psi_i}\|^2, \quad
        p_{\text{rej}}^{(i)} = \|P_{\text{rej}} \ket{\psi_i}\|^2, \quad
        p_{\text{cont}}^{(i)} = \|P_{\text{cont}} \ket{\psi_i}\|^2
     \]
     \State $p \gets p + w \cdot p_{\text{acc}}^{(i)}$
     \If{$p_{\text{cont}}^{(i)} = 0$}
        \State \textbf{break} \Comment{Computation halts in accepting or rejecting subspace}
     \Else
        \State Collapse $\ket{\psi_i}$ to $\frac{P_{\text{cont}} \ket{\psi_i}}{\|P_{\text{cont}} \ket{\psi_i}\|}$
        \State $w \gets w \cdot p_{\text{cont}}^{(i)}$
     \EndIf
  \EndFor
  \State \Return $p_M(x) = p$
  \end{algorithmic}
\end{algorithm}

  

This two-step strategy preserves the semantics of the original \gls{mm-1qfa}, interleaving unitary evolution and projective measurements. On platforms supporting mid-circuit measurements and classical branching, this logic can be directly implemented. On static circuit backends, equivalent semantics may be approximated using classically post-processed measurement results or circuit unfolding.
\medskip

\subsection{Step-by-Step Examples}

To illustrate how a \gls{mm-1qfa} is compiled into a circuit, we now present explicit examples that walk through each phase of the translation. These examples highlight the distinguishing features of the measure-many model, including intermediate measurements, early halting, and hybrid classical-quantum control. Each circuit begins with the symbolic template generated by the compiler, proceeds through instantiation for a specific input word, and concludes with either a full execution or early termination based on measurement outcomes. The goal is to clarify how the abstract operational semantics of the automaton are faithfully realised in the corresponding gate-level circuit.

\paragraph{Early Acceptance on $x = ab$} \label{ex:mmqfa-early-accept}
Consider a \gls{mm-1qfa} defined over the state space
\[
Q = \{q_0, q_1, q_2\}, \quad \Sigma = \{a, b\}, \quad q_0 \text{ initial}, \quad Q_{\text{acc}} = \{q_2\}, \quad Q_{\text{rej}} = \emptyset.
\]
Let the transition unitaries be defined as follows:
\begin{itemize}
  \item $U_a$ maps $\ket{q_0} \mapsto \ket{q_1}$,
  \item $U_b$ maps $\ket{q_1} \mapsto \ket{q_2}$.
\end{itemize}
The goal is to process the input string $x = ab$ of length $L = 2$ and observe early acceptance behaviour.

\begin{enumerate}
  \item \textbf{Qubit allocation.} The automaton has three basis states, so
  \[
  n = \lceil \log_2 3 \rceil = 2 \text{ qubits}.
  \]
  We assume an encoding where $q_0$, $q_1$, and $q_2$ correspond to $\ket{00}$, $\ket{01}$, and $\ket{10}$, respectively.

  \item \textbf{Initialisation.} The register is prepared in $\ket{q_0} = \ket{00}$.

  \item \textbf{Unitary evolution and template generation.} Since $x = ab$ has length 2, the compiler generates a circuit with two slots for $U_{x_1}$ and $U_{x_2}$, each followed by a three-outcome measurement. The symbolic circuit template is shown in subfigure~\ref{fig:mm1a}.

  \item \textbf{Instantiation.} We substitute $U_a$ and $U_b$ into the template to obtain the concrete circuit for $x = ab$, shown in subfigure~\ref{fig:mm1b}.

  \item \textbf{Gate synthesis.} Suppose $U_a$ and $U_b$ are implemented using Pauli-$X$ gates acting on the encoded state transitions (e.g., $\ket{00} \mapsto \ket{01}$ and $\ket{01} \mapsto \ket{10}$). The resulting circuit uses two $X$ gates interleaved with measurements, as shown in subfigure~\ref{fig:mm1c}.

  \item \textbf{Measurement semantics.} After applying $U_a$, the system moves from $\ket{q_0}$ to $\ket{q_1}$. Since $q_1 \notin Q_{\text{acc}} \cup Q_{\text{rej}}$, the first measurement yields \textsc{continue}. After applying $U_b$, the system transitions to $\ket{q_2} \in Q_{\text{acc}}$, so the second measurement yields \textsc{accept} and the computation halts. Thus,
  \[
  p_M(ab) = 1.
  \]
\end{enumerate}
\vspace{1em}
\begin{figure}[H]
\centering

% First row: symbolic template
\begin{subfigure}{0.9\textwidth}
\centering
\begin{quantikz}
\lstick{$\ket{q_0} = \ket{00}$} & \gate{\scriptstyle U_{x_1}} & \meter{} \qwbundle[alternate]{} 
                                & \gate{\scriptstyle U_{x_2}} & \meter{} \qwbundle[alternate]{}
\end{quantikz}
\caption{Step 1: Symbolic circuit template}
\label{fig:mm1a}
\end{subfigure}

\vspace{1em}

% Second row: instantiation and synthesis
\begin{subfigure}{0.45\textwidth}
\centering
\begin{quantikz}
\lstick{$\ket{q_0} = \ket{00}$} & \gate{\scriptstyle U_a} & \meter{} \qwbundle[alternate]{}
                                & \gate{\scriptstyle U_b} & \meter{} \qwbundle[alternate]{}
\end{quantikz}
\caption{Step 2: Instantiation for $x = ab$}
\label{fig:mm1b}
\end{subfigure}
\hfill
\begin{subfigure}{0.45\textwidth}
\centering
\begin{quantikz}
\lstick{$\ket{q_0} = \ket{00}$} & \gate{X} & \meter{} \qwbundle[alternate]{}
                                & \gate{X} & \meter{} \qwbundle[alternate]{}
\end{quantikz}
\caption{Step 2: Gate-level synthesis}
\label{fig:mm1c}
\end{subfigure}

\caption{Compilation stages for Example~\ref{ex:mmqfa-early-accept}. Input $ab$ causes an early accept: after $U_b$, the system reaches $\ket{q_2}$, a final state.}
\label{fig:mm1-horizontal}
\end{figure}


\paragraph{Early Rejection on $x = b$} \label{ex:mmqfa-early-reject}
Consider a \gls{mm-1qfa} defined over the state space
\[
Q = \{q_0, q_1\}, \quad \Sigma = \{b\}, \quad q_0 \text{ initial}, \quad Q_{\text{acc}} = \emptyset, \quad Q_{\text{rej}} = \{q_1\}.
\]
Let the transition function be defined such that:
\begin{itemize}
  \item $U_b$ maps $\ket{q_0} \mapsto \ket{q_1}$.
\end{itemize}
We examine the behaviour of the automaton on the input string $x = b$ of length $L = 1$.

\begin{enumerate}
  \item \textbf{Qubit allocation.} The automaton has two basis states, so the number of required qubits is
  \[
  n = \lceil \log_2 2 \rceil = 1.
  \]
  The classical states are encoded as $\ket{q_0} = \ket{0}$ and $\ket{q_1} = \ket{1}$.

  \item \textbf{Initialisation.} The register is initialised in the state $\ket{q_0} = \ket{0}$, which requires no gate-level preparation.

  \item \textbf{Unitary evolution and template generation.} The input $x = b$ consists of a single symbol, so the compiler generates a template with one symbolic unitary gate followed by a measurement. This is shown in subfigure~\ref{fig:mm2a}.

  \item \textbf{Instantiation.} The placeholder $U_{x_1}$ is replaced by the unitary $U_b$, as shown in subfigure~\ref{fig:mm2b}.

  \item \textbf{Gate synthesis.} Assuming $U_b$ maps $\ket{0} \mapsto \ket{1}$, it can be implemented directly as a Pauli-$X$ gate:
  \[
  U_b = X = 
  \begin{pmatrix}
  0 & 1 \\
  1 & 0
  \end{pmatrix}.
  \]

  \item \textbf{Measurement semantics.} The measurement is performed immediately after $U_b$. Since $U_b \ket{q_0} = \ket{q_1}$ and $q_1 \in Q_{\text{rej}}$, the outcome is \textsc{reject} with probability 1. The computation halts after a single step, and the automaton returns output $0$:
  \[
  p_M(b) = \|P_{\text{rej}} \ket{q_1}\|^2 = 1.
  \]
\end{enumerate}

\vspace{1em}
\begin{figure}[H]
\centering

% First row: symbolic template
\begin{subfigure}{0.9\textwidth}
\centering
\begin{quantikz}
\lstick{$\ket{q_0} = \ket{0}$} & \gate{\scriptstyle U_{x_1}} & \meter{} \qwbundle[alternate]{}
\end{quantikz}
\caption{Step 1: Symbolic circuit template}
\label{fig:mm2a}
\end{subfigure}

\vspace{1em}

% Second row: instantiation and synthesis
\begin{subfigure}{0.45\textwidth}
\centering
\begin{quantikz}
\lstick{$\ket{q_0} = \ket{0}$} & \gate{\scriptstyle U_b} & \meter{} \qwbundle[alternate]{}
\end{quantikz}
\caption{Step 2: After instantiation for $x = b$}
\label{fig:mm2b}
\end{subfigure}
\hfill
\begin{subfigure}{0.45\textwidth}
\centering
\begin{quantikz}
\lstick{$\ket{q_0} = \ket{0}$} & \gate{X} & \meter{} \qwbundle[alternate]{}
\end{quantikz}
\caption{Step 2: Gate-level synthesis}
\label{fig:mm2c}
\end{subfigure}

\caption{Compilation stages for Example~\ref{ex:mmqfa-early-reject}. The input $b$ triggers an immediate transition to $\ket{q_1}$, a rejecting state. The automaton halts after one step and returns \textsc{reject}.}
\label{fig:mm2-horizontal}
\end{figure}



