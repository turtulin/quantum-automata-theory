% ============================================================
\section{Measure-Many One-Way Quantum Finite Automaton to Circuit}
\label{sec:mmqfa-to-circuit}

The construction of a quantum circuit from a \emph{measure-many}
QFA (MM-1QFA) parallels the measure-once case but differs
crucially in that a projective measurement is performed
\emph{after every input symbol}, rather than only once at the end.
This allows the automaton to halt---accepting or rejecting---as
soon as a decisive outcome is observed, possibly before the whole
word has been read.

\subsection{Structural Similarity with Measure-Once QFA}

An MM-1QFA is a tuple
\[
  A=(Q,\Sigma,\delta,q_0,F),
\]
with the same meanings for $Q,\Sigma,\delta,q_0$ and $F$ as before.
The first compilation steps are identical to
Section~\ref{sec:moqfa-to-circuit}:

\begin{enumerate}[leftmargin=*,label=\textbf{\arabic*.}]
\item \textbf{Number of qubits.}
      Choose $n=\lceil\log_2|Q|\rceil$ so that each
      $q\in Q$ is a basis vector $\ket{q}$ of an $n$-qubit register.
\item \textbf{Initial state.}
      Prepare the register in $\ket{q_0}$ from $\ket{0}^{\otimes n}$.
\item \textbf{Symbol unitaries.}
      Reading $\sigma\in\Sigma$ applies $U_\sigma$ to the register.
\end{enumerate}

\noindent
The divergence comes next: a three-outcome measurement is inserted
after \emph{each} $U_\sigma$.

\subsection{Circuit Construction}

Let $R\subseteq Q$ be the (optional) set of \emph{rejecting}
states---if none are designated, set $R=Q\setminus F$.
After every symbol the register is measured with
\[
  P_{\text{acc}}=\sum_{q\in F}\ket{q}\!\bra{q},
  \quad
  P_{\text{rej}}=\sum_{q\in R}\ket{q}\!\bra{q},
  \quad
  P_{\text{cont}}=I-P_{\text{acc}}-P_{\text{rej}}.
\]
Outcome \textsc{accept} (resp.\ \textsc{reject})
halts the computation and outputs $1$ (resp.\ $0$);
outcome \textsc{continue} applies the next $U_\sigma$.
Algorithm \ref{alg:mmqfa} formalises this behaviour.

\begin{algorithm}[H]
\caption{MM-1QFA circuit execution on a word $x=x_1\!\dots x_L$}
\label{alg:mmqfa}
\begin{algorithmic}[1]
\Require $A=(Q,\Sigma,\delta,q_0,F)$, (optional) $R$, input $x$
\State Initialise register in $\ket{q_0}$
\For{$i=1$ \textbf{to} $L$}
   \State Apply $U_{x_i}$
   \State Measure in $\{P_{\text{acc}},P_{\text{rej}},P_{\text{cont}}\}$
   \If{\textbf{accept}} \Return output $1$ \EndIf
   \If{\textbf{reject}} \Return output $0$ \EndIf
\EndFor
\State \Return output $0$ \Comment{default reject if no halt earlier}
\end{algorithmic}
\end{algorithm}

\subsection{Step-by-Step Examples}

%------------------------------------------------------------------
\paragraph{Example 1: early acceptance on $x=ab$.}

\begin{figure}[htb]
\centering
\begin{subfigure}{0.30\textwidth}
\centering
\begin{quantikz}[row sep=0.15cm]
\lstick{$\ket{q_0}$} & \gate[style={draw=blue!50,fill=blue!10}]{U_{x_1}}
                     & \meter{} \qw
                     & \gate[style={draw=blue!50,fill=blue!10}]{U_{x_2}}
                     & \meter{} \cw \\
\end{quantikz}
\caption{Template ($L=2$)}
\end{subfigure}\hfill
%
\begin{subfigure}{0.30\textwidth}
\centering
\begin{quantikz}[row sep=0.15cm]
\lstick{$\ket{q_0}$} & \gate{U_a} & \meter{} \qw
                     & \gate{U_b} & \meter{} \cw
\end{quantikz}
\caption{Instantiate ($x=ab$)}
\end{subfigure}\hfill
%
\begin{subfigure}{0.30\textwidth}
\centering
\footnotesize
\begin{quantikz}[row sep=0.15cm]
\lstick{$\ket{q_0}$} & \gate{X} & \meter{}
                     & \gate{X} & \meter{} \cw
\end{quantikz}
\caption{Gate level (example synthesis)}
\end{subfigure}
\caption{Compilation stages for Example 1 (early accept). 
If the first measurement yields \textsc{continue}, the second
measurement halts with \textsc{accept}.}
\label{fig:mmqfa-early-accept}
\end{figure}

%------------------------------------------------------------------
\paragraph{Example 2: early rejection on $x=b$.}

\begin{figure}[htb]
\centering
\begin{subfigure}{0.42\textwidth}
\centering
\begin{quantikz}[row sep=0.15cm]
\lstick{$\ket{q_0}$} & \gate[style={draw=blue!50,fill=blue!10}]{U_{x_1}} & \meter{} \cw
\end{quantikz}
\caption{Template ($L=1$)}
\end{subfigure}\hfill
%
\begin{subfigure}{0.42\textwidth}
\centering
\begin{quantikz}[row sep=0.15cm]
\lstick{$\ket{q_0}$} & \gate{U_b} & \meter{} \cw
\end{quantikz}
\caption{Instantiate \& synthesis ($x=b$)}
\end{subfigure}
\caption{Example 2: the first measurement finds the register in a
rejecting state and halts immediately with output $0$.}
\label{fig:mmqfa-early-reject}
\end{figure}

\bigskip
The early-halting capability reduces runtime on many inputs but
requires mid-circuit measurements and classical control, hence the
MM-1QFA circuit is typically larger and more intricate than its
measure-once counterpart. Nevertheless, the translation preserves
language recognition exactly: a string accepted (rejected) by the
automaton causes the circuit to output~$1$ ($0$) with the same
probability.
% ============================================================
