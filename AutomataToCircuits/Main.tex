
\chapter{Automata to Circuits}
\label{chap:automata-to-circuits}
\noindent
\glspl{qfa} furnish a concise, mathematically transparent model of finite-memory computation, yet practical algorithms must ultimately be recast as quantum circuits that manipulate qubits through finite sequences of gates and measurements.  The purpose of this chapter is to articulate, in a systematic manner, how a quantum automaton defined at the symbolic level is translated into a concrete circuit description suitable for compilation on \glspl{nisq} hardware.  By mapping each automaton primitive onto circuit counterparts we obtain designs that are executable on present devices, support quantitative resource accounting and admit gate-level formal verification.

\medskip
\noindent
Section~\ref{sec:moqfa-to-circuit} outlines the compilation workflow for the \gls{mo-1qfa}, illustrating how its fundamental components can be encoded within a quantum circuit model.  The translation process preserves the computational semantics of the original automaton while making it compatible with standard circuit synthesis techniques.

\medskip
\noindent
Section~\ref{sec:mmqfa-to-circuit} extends the methodology to the \gls{mm-1qfa}, whose intermediate measurements create early-halt branches and classical control flow.  Particular attention is devoted to expressing the three-outcome measurement paradigm with standard two-outcome projective tests, to limiting ancilla overhead when discarding rejected branches, and to maintaining language-recognition semantics in the presence of realistic noise.

\medskip
\noindent
A template-first compilation philosophy is retained throughout: for an input word of length $L$ the compiler emits a parameterised skeleton in which each placeholder gate is later instantiated with the concrete operator $U_{\sigma_i}$ attached to the $i$-th symbol.  This separation between structural aspects fixed by the automaton and numerical parameters dictated by the input encourages component reuse across multiple words and eases the deployment of \glspl{qfa} as high-speed recognisers within larger quantum applications.  Two complementary instantiation strategies are considered in Section~\ref{sec:unitary-operators-instantiation}.  The first, an offline synthesis approach, compiles every operator $U_\sigma$ ahead of execution and stores the resulting gate sequences as reusable fragments.  The second adopts a parameter-loading paradigm in which a generic template containing analytic Euler-angle rotations is populated at runtime with classically computed angles that depend on the input word, thereby reducing memory overhead and enabling just-in-time adaptation to specific problem instances.

\medskip
\noindent
Upon completing the chapter the reader will possess a reproducible method for converting any \gls{mo-1qfa} or \gls{mm-1qfa} into an architecture-independent gate-level description, together with practical criteria for choosing state encodings, measurement decompositions and synthesis back-ends.  These results pave the way for future research on two-way and hybrid models and represent a decisive step toward a unified tool-chain for automata-driven quantum software engineering.

\section{\glsentrylong{mo-1qfa} to Circuit}
\label{sec:moqfa-to-circuit}

% ============================================================
\section{Measure-Many One-Way Quantum Finite Automaton to Circuit}
\label{sec:mmqfa-to-circuit}

The construction of a quantum circuit from a \emph{measure-many}
QFA (MM-1QFA) parallels the measure-once case but differs
crucially in that a projective measurement is performed
\emph{after every input symbol}, rather than only once at the end.
This allows the automaton to halt---accepting or rejecting---as
soon as a decisive outcome is observed, possibly before the whole
word has been read.

\subsection{Structural Similarity with Measure-Once QFA}

An MM-1QFA is a tuple
\[
  A=(Q,\Sigma,\delta,q_0,F),
\]
with the same meanings for $Q,\Sigma,\delta,q_0$ and $F$ as before.
The first compilation steps are identical to
Section~\ref{sec:moqfa-to-circuit}:

\begin{enumerate}[leftmargin=*,label=\textbf{\arabic*.}]
\item \textbf{Number of qubits.}
      Choose $n=\lceil\log_2|Q|\rceil$ so that each
      $q\in Q$ is a basis vector $\ket{q}$ of an $n$-qubit register.
\item \textbf{Initial state.}
      Prepare the register in $\ket{q_0}$ from $\ket{0}^{\otimes n}$.
\item \textbf{Symbol unitaries.}
      Reading $\sigma\in\Sigma$ applies $U_\sigma$ to the register.
\end{enumerate}

\noindent
The divergence comes next: a three-outcome measurement is inserted
after \emph{each} $U_\sigma$.

\subsection{Circuit Construction}

Let $R\subseteq Q$ be the (optional) set of \emph{rejecting}
states---if none are designated, set $R=Q\setminus F$.
After every symbol the register is measured with
\[
  P_{\text{acc}}=\sum_{q\in F}\ket{q}\!\bra{q},
  \quad
  P_{\text{rej}}=\sum_{q\in R}\ket{q}\!\bra{q},
  \quad
  P_{\text{cont}}=I-P_{\text{acc}}-P_{\text{rej}}.
\]
Outcome \textsc{accept} (resp.\ \textsc{reject})
halts the computation and outputs $1$ (resp.\ $0$);
outcome \textsc{continue} applies the next $U_\sigma$.
Algorithm \ref{alg:mmqfa} formalises this behaviour.

\begin{algorithm}[H]
\caption{MM-1QFA circuit execution on a word $x=x_1\!\dots x_L$}
\label{alg:mmqfa}
\begin{algorithmic}[1]
\Require $A=(Q,\Sigma,\delta,q_0,F)$, (optional) $R$, input $x$
\State Initialise register in $\ket{q_0}$
\For{$i=1$ \textbf{to} $L$}
   \State Apply $U_{x_i}$
   \State Measure in $\{P_{\text{acc}},P_{\text{rej}},P_{\text{cont}}\}$
   \If{\textbf{accept}} \Return output $1$ \EndIf
   \If{\textbf{reject}} \Return output $0$ \EndIf
\EndFor
\State \Return output $0$ \Comment{default reject if no halt earlier}
\end{algorithmic}
\end{algorithm}

\subsection{Step-by-Step Examples}

%------------------------------------------------------------------
\paragraph{Example 1: early acceptance on $x=ab$.}

\begin{figure}[htb]
\centering
\begin{subfigure}{0.30\textwidth}
\centering
\begin{quantikz}[row sep=0.15cm]
\lstick{$\ket{q_0}$} & \gate[style={draw=blue!50,fill=blue!10}]{U_{x_1}}
                     & \meter{} \qw
                     & \gate[style={draw=blue!50,fill=blue!10}]{U_{x_2}}
                     & \meter{} \cw \\
\end{quantikz}
\caption{Template ($L=2$)}
\end{subfigure}\hfill
%
\begin{subfigure}{0.30\textwidth}
\centering
\begin{quantikz}[row sep=0.15cm]
\lstick{$\ket{q_0}$} & \gate{U_a} & \meter{} \qw
                     & \gate{U_b} & \meter{} \cw
\end{quantikz}
\caption{Instantiate ($x=ab$)}
\end{subfigure}\hfill
%
\begin{subfigure}{0.30\textwidth}
\centering
\footnotesize
\begin{quantikz}[row sep=0.15cm]
\lstick{$\ket{q_0}$} & \gate{X} & \meter{}
                     & \gate{X} & \meter{} \cw
\end{quantikz}
\caption{Gate level (example synthesis)}
\end{subfigure}
\caption{Compilation stages for Example 1 (early accept). 
If the first measurement yields \textsc{continue}, the second
measurement halts with \textsc{accept}.}
\label{fig:mmqfa-early-accept}
\end{figure}

%------------------------------------------------------------------
\paragraph{Example 2: early rejection on $x=b$.}

\begin{figure}[htb]
\centering
\begin{subfigure}{0.42\textwidth}
\centering
\begin{quantikz}[row sep=0.15cm]
\lstick{$\ket{q_0}$} & \gate[style={draw=blue!50,fill=blue!10}]{U_{x_1}} & \meter{} \cw
\end{quantikz}
\caption{Template ($L=1$)}
\end{subfigure}\hfill
%
\begin{subfigure}{0.42\textwidth}
\centering
\begin{quantikz}[row sep=0.15cm]
\lstick{$\ket{q_0}$} & \gate{U_b} & \meter{} \cw
\end{quantikz}
\caption{Instantiate \& synthesis ($x=b$)}
\end{subfigure}
\caption{Example 2: the first measurement finds the register in a
rejecting state and halts immediately with output $0$.}
\label{fig:mmqfa-early-reject}
\end{figure}

\bigskip
The early-halting capability reduces runtime on many inputs but
requires mid-circuit measurements and classical control, hence the
MM-1QFA circuit is typically larger and more intricate than its
measure-once counterpart. Nevertheless, the translation preserves
language recognition exactly: a string accepted (rejected) by the
automaton causes the circuit to output~$1$ ($0$) with the same
probability.
% ============================================================

\section{Unitary Operators Instantiation}
\label{sec:unitary-operators-instantiation}

Once the circuit skeleton is constructed for a given \gls{mo-1qfa} or \gls{mm-1qfa}, the remaining compilation step consists in instantiating the placeholder unitaries $U_{\sigma}$ with actual gate-level operators. This section presents the main strategies available to achieve such instantiation, highlighting the trade-offs between offline preprocessing and dynamic runtime synthesis.

\subsection{Offline Synthesis}

The offline synthesis strategy precomputes a gate decomposition for each unitary matrix $U_{\sigma}$ associated with the alphabet $\Sigma$. This approach is particularly suitable when the automaton is fixed and used to process multiple inputs of the same language class. The steps are:

\begin{itemize}
    \item For every $\sigma \in \Sigma$, extract the unitary matrix $U_{\sigma}$ defined by the automaton's transition function $\delta$.
    \item Decompose $U_{\sigma}$ into a circuit of elementary gates from a fixed universal set (e.g., Clifford+T or $\{\mathrm{CNOT}, R_z, H\}$).
    \item Store each gate sequence as a reusable fragment in a gate library.
\end{itemize}

This method guarantees high performance at runtime, as no decomposition is needed during execution. However, it requires more memory to store all precompiled gate sequences, and it lacks adaptability in contexts where $U_{\sigma}$ changes dynamically or is defined procedurally.

\subsection{Template-Based Parameter Loading}

An alternative method is template-based parameter loading, where the circuit skeleton includes parametrized gates (e.g., Euler-angle rotations) and the actual rotation angles are injected at runtime based on the specific $U_{\sigma}$ required. This is achieved by:

\begin{itemize}
    \item Designing each $U_{\sigma}$ as a composition of generic rotation gates (e.g., $R_z(\theta_1) R_y(\theta_2) R_z(\theta_3)$).
    \item Computing the angles $\theta_1, \theta_2, \theta_3$ classically using a synthesis algorithm (e.g., ZYZ decomposition) from the matrix representation of $U_{\sigma}$.
    \item Populating the parametrized gates of the circuit with the computed angles just before execution.
\end{itemize}

This strategy supports adaptive and memory-efficient compilation, especially useful when the automaton model is generated on-the-fly or when circuits are embedded in larger configurable pipelines. The downside is the runtime overhead incurred by angle computation and dynamic loading.

\subsection{Hybrid and Optimized Approaches}

In practice, a hybrid scheme combining both methods is often adopted. Common unitary matrices with known decompositions can be stored offline, while less frequent or dynamically generated ones are handled through runtime parameter loading. Furthermore, if the automaton contains symmetries (e.g., cyclic state transitions), structural optimizations can reduce the number of distinct unitaries needed, enabling further compression of the circuit template.

\subsection{Summary}

Unitary instantiation closes the automaton-to-circuit translation by assigning concrete quantum operations to each input-driven evolution. Offline synthesis prioritizes speed and repeatability; template-based methods emphasize flexibility and memory economy. The choice depends on the application domain—static recognizers may favor offline strategies, while programmable quantum systems benefit from dynamic parameter loading.

