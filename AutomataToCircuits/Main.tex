\chapter{Automata To Circuits}
\label{chap:automata-to-circuits}

This chapter describes the process of converting finite automata into circuits. The conversion is done in two steps: first, we convert the automata into a binary decision diagram (BDD), and then we convert the BDD into a circuit. The BDD is a data structure that represents a Boolean function, and it can be used to simplify the representation of the automata. The circuit is a hardware representation of the automata that can be used for various applications, such as model checking or formal verification.
The conversion process is described in detail, and the resulting circuits are analyzed for their size and complexity. The chapter also discusses the advantages and disadvantages of using circuits for representing automata, and it provides examples of how the conversion process can be applied to different types of automata.
The chapter is structured as follows:
\begin{itemize}
    \item Section introduces the concept of converting automata to circuits and provides an overview of the process.
    \item Section describes the conversion of automata to BDDs, including the algorithms used and the properties of BDDs.
    \item Section discusses the conversion of BDDs to circuits, including the types of circuits that can be generated and their properties.
    \item Section provides examples of the conversion process applied to different types of automata.
    \item Section concludes the chapter with a summary of the main points and a discussion of future work in this area.
\end{itemize}
The chapter is intended for readers with a background in formal methods, automata theory, and circuit design. It assumes familiarity with the basic concepts of finite automata, BDDs, and circuits. The chapter is self-contained and provides all the necessary background information for understanding the conversion process.


\section{\glsentrylong{moqfa} to Circuit}
\label{sec:moqfa-to-circuit}

\section{\glsentrylong{mmqfa} to Circuit}
\label{sec:mmqfa}

This section describes the \glsentryfull{mmqfa} model, which is a generalisation of the \glsentryfull{moqfa} model. The \glsentrylong{mmqfa} model is defined in a similar way to the \glsentrylong{moqfa} model, but it allows for multiple measurements at each state. This section will cover the definition of the \glsentrylong{mmqfa} model, its acceptance conditions, and its properties.