\chapter{Conclusion}
\label{chap:conclusion}

This thesis systematically analyzed \glspl{qfa} by comparing them to classical models like \glspl{dfa}, \glspl{nfa}, and \glspl{pfa}. The goal was to clarify where quantum advantages exist, quantify their costs, and identify practical limitations.

Three key results emerged. First, one-way \glspl{qfa} such as \glspl{mm-1qfa} achieve exponential state reduction over classical automata for regular languages but require careful error management. Second, two-way and hybrid models like \glspl{2qcfa} extend recognition to non-regular languages (e.g., $\{a^n b^n\}$) with constant quantum memory, though classical control introduces synchronization overhead. Third, generalized models (\glspl{gqfa}) and interactive protocols (\glspl{qmip}) reveal fundamental trade-offs: while they theoretically recognize complex languages, their reliance on noise modeling or multi-prover coordination makes them experimentally challenging.

These findings have concrete implications. For algorithm designers, hybrid models like \glspl{1qfac} offer a pragmatic balance—using minimal quantum resources while retaining classical reliability. For hardware developers, the analysis underscores the need for error correction tailored to automata (e.g., handling decoherence in \glspl{mo-1qfa}'s unitary evolution). The undecidability of equivalence problems for many \glspl{qfa} also highlights a theoretical barrier: verifying quantum automata behavior may require new formal methods.

Future work should prioritize practical over theoretical novelty. Compiling \glspl{qfa} into quantum circuits would test their real-world viability, while equivalence studies between \glspl{qfa} and \glspl{qtm} could unify computational models. For industry, implementing \glspl{2qcfa} on NISQ devices to solve simple context-sensitive problems (e.g., XML validation) might demonstrate near-term utility.

In summary, this work clarifies what quantum automata can and cannot do today. It provides a roadmap for leveraging their strengths—state efficiency and parallelism—while cautioning against underestimating their fragility. The path forward lies in bridging formal theory with engineering constraints, not chasing abstract generality.