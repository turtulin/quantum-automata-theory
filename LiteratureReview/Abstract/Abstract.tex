\section{Abstract and Algebraic Quantum Finite Automata}
\label{sec:abstract-qfa}

This section explores quantum automata models defined through algebraic structures and abstract mathematical frameworks, providing foundational insights into quantum computational paradigms.

%%%%%%%%%%%%%%%%%%%%%%%%%%%%%%%%%%%%%%%%%%%%%%%%%%%%%%%%%%%%%%
\subsection{\acrfull{abstract-qfa} (Manin's Framework)}
\label{subsec:abstract-qfa}

\begin{definition}[\gls{abstract-qfa}]
An abstract quantum automaton is defined as:
\[
\mathcal{A} = (\mathcal{H}, \Sigma, \{\mathcal{I}_\sigma\}, \ket{\psi_0}, \mathcal{P})
\]
\textbf{where:}
\begin{itemize}
    \item $\mathcal{H}$: Hilbert space of quantum states
    \item $\Sigma$: Input alphabet
    \item $\mathcal{I}_\sigma: \mathcal{H} \rightarrow \mathcal{H}$: Isometric operators for each $\sigma \in \Sigma$
    \item $\ket{\psi_0} \in \mathcal{H}$: Initial state
    \item $\mathcal{P} = \{P_a, P_r\}$: Projection operators for acceptance/rejection
\end{itemize}
\end{definition}

\paragraph{Operation:}
\begin{enumerate}
    \item \textbf{State Evolution:} For input $w = \sigma_1...\sigma_n$:
    \[
    \ket{\psi_w} = \mathcal{I}_{\sigma_n} \circ \cdots \circ \mathcal{I}_{\sigma_1} (\ket{\psi_0})
    \]
    \item \textbf{Measurement:} Apply projective measurement $\mathcal{P}$
\end{enumerate}

\paragraph{Language Acceptance:}
Recognizes \textbf{all languages in BQP} through polynomial-time simulations of concrete QFA models \cite{manin1980computable, gruska2005}. Specifically:
\[
L(\mathcal{A}) = \{w \in \Sigma^* \mid \|P_a\ket{\psi_w}\|^2 \geq 2/3\}
\]

\paragraph{Advantages:}
\begin{itemize}
    \item Provides categorical framework for unifying all QFA variants
    \item Enables complexity analysis via operator algebra methods
\end{itemize}

\paragraph{Limitations:}
\begin{itemize}
    \item No direct physical implementation scheme
    \item Undecidable equivalence problem for isometric operators
\end{itemize}

%%%%%%%%%%%%%%%%%%%%%%%%%%%%%%%%%%%%%%%%%%%%%%%%%%%%%%%%%%%%%%
\subsection{\acrfull{olva}}
\label{subsec:olva}

\begin{definition}[\gls{olva}]
An \gls{olva} is defined as:
\[
M = (L, \Sigma, \delta, l_0, F)
\]
\textbf{where:}
\begin{itemize}
    \item $L$: Orthomodular lattice modeling quantum logic
    \item $\delta: L \times \Sigma \rightarrow L$: Transition function preserving lattice operations
    \item $l_0 \in L$: Initial element
    \item $F \subseteq L$: Accepting elements
\end{itemize}
\end{definition}

\paragraph{Operation:}
\begin{enumerate}
    \item \textbf{State Transitions:} For input $\sigma$:
\[
l_{i+1} = \delta(l_i, \sigma) \land (l_i \lor \delta(l_i, \sigma)^\perp)
\]
    \item \textbf{Acceptance:} Check $l_n \in F$ using lattice order
\end{enumerate}

\paragraph{Language Acceptance:}
Recognizes \textbf{quantum decision languages} closed under:
\[
L = \{w \mid \bigvee_{i=1}^n (p_i(w) \land \neg q_i(w)) \in F\}
\]
where $p_i,q_i$ are lattice-valued predicates \cite{gudder1978quantum}.

\paragraph{Advantages:}
\begin{itemize}
    \item Models quantum superposition through lattice joins
    \item Captures contextuality via non-distributive lattices
\end{itemize}

\paragraph{Limitations:}
\begin{itemize}
    \item NP-hard membership problem for general lattices
    \item Limited to propositional quantum logics
\end{itemize}

%%%%%%%%%%%%%%%%%%%%%%%%%%%%%%%%%%%%%%%%%%%%%%%%%%%%%%%%%%%%%%
\subsection{\acrfull{mpsqfa}}
\label{subsec:mpsqfa}

\begin{definition}[\gls{mpsqfa}]
A \gls{mpsqfa} is defined as:
\[
M = (\Sigma, \{A_\sigma\}, \ket{\psi_0}, \bra{\psi_F}, D)
\]
\textbf{where:}
\begin{itemize}
    \item $A_\sigma \in \mathbb{C}^{D \times D}$: Transition matrices for $\sigma \in \Sigma$
    \item $\ket{\psi_0} \in \mathbb{C}^D$: Initial MPS
    \item $\bra{\psi_F} \in \mathbb{C}^D$: Final measurement vector
    \item $D$: Bond dimension
\end{itemize}
\end{definition}

\paragraph{Operation:}
\begin{enumerate}
    \item \textbf{Input Processing:} For $w = \sigma_1...\sigma_n$:
    \[
    \alpha(w) = \bra{\psi_F} A_{\sigma_n} \cdots A_{\sigma_1} \ket{\psi_0}
    \]
    \item \textbf{Acceptance:} $w$ accepted iff $|\alpha(w)|^2 \geq \theta$
\end{enumerate}

\paragraph{Language Acceptance:}
Simulates \textbf{1D quantum spin systems} recognizing languages with:
\[
L = \{w \in \Sigma^n \mid \text{tr}\left(\bigotimes_{i=1}^n A_{w_i}\right) \neq 0\}
\]
for nearest-neighbor Hamiltonians \cite{vidal2003efficient}.

\paragraph{Advantages:}
\begin{itemize}
    \item Efficient simulation of quantum many-body systems
    \item Polynomial-time training via density matrix renormalization
\end{itemize}

\paragraph{Limitations:}
\begin{itemize}
    \item Restricted to 1D entanglement structures
    \item Bond dimension $D$ grows exponentially with language complexity
\end{itemize}

%%%%%%%%%%%%%%%%%%%%%%%%%%%%%%%%%%%%%%%%%%%%%%%%%%%%%%%%%%%%%%
\subsection*{Comparative Analysis and Hierarchy}
\begin{figure}[h]
\centering
\begin{tikzpicture}[node distance=2.2cm]
\node (mps) [draw, align=center] {\gls{mpsqfa}\\ 1D Spin Systems};
\node (ol) [draw, right=of mps, align=center] {\gls{olva}\\ Quantum Logics};
\node (abs) [draw, right=of ol, align=center] {\gls{abstract-qfa}\\ BQP Class};

\draw[->] (mps) edge node[above, align=center] {+ Algebraic Structure\\ + Entanglement} (ol)
          (ol) edge node[above, align=center] {+ Categorical Framework\\ + Universal Models} (abs);
\end{tikzpicture}
\caption{Expressive hierarchy of abstract/algebraic QFAs. \gls{mpsqfa} simulates 1D spin systems \cite{vidal2003efficient}, while \gls{abstract-qfa} captures BQP via Manin's framework \cite{manin1980computable}.}
\label{fig:abstract-hierarchy}
\end{figure}

\textbf{Key Comparisons:}
\begin{itemize}
    \item \textbf{Expressiveness:} MPSQFA $\subset$ OLVA $\subset$ AbstractQFA
    \item \textbf{Complexity:} MPSQFA (P), OLVA (NP), AbstractQFA (BQP)
    \item \textbf{Implementation:} MPSQFA physically realizable, AbstractQFA purely theoretical
\end{itemize}