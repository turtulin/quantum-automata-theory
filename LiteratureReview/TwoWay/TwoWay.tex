\section{\acrfull{2qfa}}
\label{sec:two-way-qfas}

\glspl{2qfa} extend one-way models by permitting bidirectional head movement, thereby enhancing their computational power. In \gls{2qfa} models the head may move left, right, or remain stationary, which allows the automaton to exploit quantum interference over multiple passes of the input. In the following subsections, we detail the standard models, hybrid variants, and multihead/tape extensions. For each model, we provide a formal definition, describe its operation, discuss language acceptance properties, and list key features and limitations.

%%%%%%%%%%%%%%%%%%%%%%%%%%%%%%%%%%%%%%%%%%%%%%%%%%%%%%%%%%%%%%
\subsection{Standard Models}
\label{subsec:two-way-standard}

%%%%%%%%%%%%%%%%%%%%%%%%%%%%%%%%%%%%%%%%%%%%%%%%%%%%%%%%%%%%%%
\subsubsection{\acrfull{2qfa}}
\label{sssec:2qfa}
\begin{definition}[\gls{2qfa}]
A \gls{2qfa} is defined as 
\[
M = (Q, \Sigma, \delta, q_0, Q_{\text{acc}}, Q_{\text{rej}}),
\]
where:
\begin{itemize}
    \item \( Q \) is a finite set of quantum states, partitioned into accepting states \( Q_{\text{acc}} \), rejecting states \( Q_{\text{rej}} \), and non-halting states \( Q_{\text{non}} = Q \setminus (Q_{\text{acc}} \cup Q_{\text{rej}}) \).
    \item \( \Sigma \) is the input alphabet, extended with left and right end-markers (e.g., \( \# \) and \( \$ \)).
    \item \( \delta: Q \times \Gamma \to \mathbb{C}^{\, Q \times \{ \leftarrow, \downarrow, \rightarrow \}} \) is the transition function (with \( \Gamma = \Sigma \cup \{ \#, \$ \} \)) that specifies both the unitary evolution and the head movement.
    \item \( q_0 \in Q \) is the initial state.
\end{itemize}
\end{definition}

\textbf{Operation}:  
Given an input \( w = \sigma_1\sigma_2\ldots\sigma_n \), the automaton repeatedly applies the corresponding unitary operators while moving its head left, right, or remaining stationary as dictated by \(\delta\). Intermediate measurements (or deferred measurement techniques) are employed so that quantum interference can be exploited across multiple passes of the input.

\paragraph{Language Acceptance:}  
A \gls{2qfa} is capable of accepting non-regular languages with bounded error. For example, it has been shown that the language 
\[
L_{\text{eq}} = \{a^n b^n \mid n \geq 0\}
\]
can be recognized with bounded error in linear time \cite{kondacs1997power, yakaryilmaz2010succinctness}.

\paragraph{Key Features and Limitations:}
\begin{itemize}
    \item \textbf{Key Features:}
    \begin{itemize}
        \item Ability to recognize non-regular languages such as \( L_{\text{eq}} \) with bounded error.
        \item Exploits quantum interference, which can lead to exponential state savings over classical two-way automata.
    \end{itemize}
    \item \textbf{Limitations:}
    \begin{itemize}
        \item Requires quantum registers whose size may scale with the input length.
        \item Error management and decoherence pose significant challenges for practical implementations.
    \end{itemize}
\end{itemize}

%%%%%%%%%%%%%%%%%%%%%%%%%%%%%%%%%%%%%%%%%%%%%%%%%%%%%%%%%%%%%%
\subsection{Hybrid Models}
\label{subsec:two-way-hybrid}

%%%%%%%%%%%%%%%%%%%%%%%%%%%%%%%%%%%%%%%%%%%%%%%%%%%%%%%%%%%%%%
\subsubsection{\gls{2qcfa}}
\label{sssec:2qcfa}
\begin{definition}[\gls{2qcfa}]
A \gls{2qcfa} is defined as 
\[
M = (S, Q, \Sigma, \Theta, \delta, s_0, q_0, S_{\text{acc}}, S_{\text{rej}}),
\]
where:
\begin{itemize}
    \item \( S \) is a finite set of classical control states.
    \item \( Q \) is a finite set of quantum states.
    \item \( \Sigma \) is the input alphabet, augmented with appropriate end-markers.
    \item \( \Theta \) assigns quantum operations (unitary transformations or measurements) based on the current classical state and input symbol.
    \item \( \delta: S \times \Sigma \to S \) governs the classical state transitions and head movements.
    \item \( s_0 \in S \) and \( q_0 \in Q \) are the initial classical and quantum states, respectively.
    \item \( S_{\text{acc}} \) and \( S_{\text{rej}} \) are the sets of accepting and rejecting classical states.
\end{itemize}
\end{definition}

\textbf{Operation}:  
In a \gls{2qcfa}, the classical component directs the head movement and determines when to invoke quantum operations via \( \Theta \). Adaptive measurements are performed based on the current classical state, allowing the automaton to decide acceptance using a small quantum register.

\paragraph{Language Acceptance:}  
\gls{2qcfa} can recognize certain non-regular languages, such as \( L_{\text{eq}} \) and palindromes, in polynomial time while using only a constant number of quantum states \cite{ambainis2002quantum}.

\paragraph{Key Features and Limitations:}
\begin{itemize}
    \item \textbf{Key Features:}
    \begin{itemize}
        \item Combines classical control with quantum computation for enhanced efficiency.
        \item Recognizes non-regular languages in polynomial time with a minimal quantum register.
    \end{itemize}
    \item \textbf{Limitations:}
    \begin{itemize}
        \item Synchronization between classical and quantum components increases design complexity.
        \item The decidability of the equivalence problem for 2QCFA remains an open issue.
    \end{itemize}
\end{itemize}

%%%%%%%%%%%%%%%%%%%%%%%%%%%%%%%%%%%%%%%%%%%%%%%%%%%%%%%%%%%%%%
\subsection{Multihead/Tape Extensions}
\label{subsec:multihead-tape}

%%%%%%%%%%%%%%%%%%%%%%%%%%%%%%%%%%%%%%%%%%%%%%%%%%%%%%%%%%%%%%
\subsubsection{\gls{2tqcfa}}
\label{sssec:2tqcfa}
\begin{definition}[\gls{2tqcfa}]
A \gls{2tqcfa} is defined as 
\[
M = (S, Q, \Sigma_1 \times \Sigma_2, \Theta, \delta, s_0, q_0, S_{\text{acc}}, S_{\text{rej}}),
\]
where:
\begin{itemize}
    \item \( \Sigma_1 \) and \( \Sigma_2 \) are the input alphabets for the two tapes.
    \item The remaining components (\( S, Q, \Theta, \delta, s_0, q_0, S_{\text{acc}}, S_{\text{rej}} \)) are defined analogously to those in a 2QCFA.
    \item The transition function \(\delta\) synchronizes the head movements on both tapes.
\end{itemize}
\end{definition}

\textbf{Operation}:  
The automaton simultaneously reads from two tapes, allowing it to verify language properties that involve correlations between substrings on different tapes. This parallel processing can be exploited to recognize languages like 
\[
L = \{a^n b^n c^n \mid n \geq 1\},
\]
by comparing corresponding symbols from the two tapes \cite{zheng2012two}.

\paragraph{Language Acceptance:}  
2TQCFA accept languages that require cross-tape comparisons and correlation between substrings, thereby extending the recognition power beyond that of single-tape models.

\paragraph{Key Features and Limitations:}
\begin{itemize}
    \item \textbf{Key Features:}
    \begin{itemize}
        \item Enhanced recognition power compared to single-tape 2QCFA.
        \item Efficient processing of languages with interdependent substrings.
    \end{itemize}
    \item \textbf{Limitations:}
    \begin{itemize}
        \item Increased complexity in synchronizing head movements across two tapes.
        \item Higher error-correction overhead due to parallel tape processing.
    \end{itemize}
\end{itemize}

%%%%%%%%%%%%%%%%%%%%%%%%%%%%%%%%%%%%%%%%%%%%%%%%%%%%%%%%%%%%%%
\subsubsection{\gls{ktqcfa}}
\label{sssec:ktqcfa}
\begin{definition}[\gls{ktqcfa}]
A \gls{ktqcfa} generalizes the two-tape model to \( k \) tapes. It is defined as 
\[
M = \Bigl(S, Q, \bigtimes_{i=1}^{k} \Sigma_i, \Theta, \delta, s_0, q_0, S_{\text{acc}}, S_{\text{rej}}\Bigr),
\]
where:
\begin{itemize}
    \item Each tape has its own alphabet \( \Sigma_i \) for \( i = 1, 2, \ldots, k \).
    \item The transition function \(\delta\) coordinates the head movements across all \( k \) tapes.
    \item The remaining components are defined similarly to those in a 2QCFA.
\end{itemize}
\end{definition}

\textbf{Operation}:  
The automaton processes input from \( k \) tapes simultaneously. This multi-tape configuration enables efficient recognition of languages with complex structural dependencies. For instance, languages such as 
\[
L = \{a^n b^{n^2} \mid n \geq 1\}
\]
can be recognized more effectively by performing parallel comparisons across the tapes \cite{zheng2012two}.

\paragraph{Language Acceptance:}  
kTQCFA are designed to accept languages with intricate interdependencies that benefit from simultaneous multi-tape processing, thereby extending the expressive power of quantum finite automata.

\paragraph{Key Features and Limitations:}
\begin{itemize}
    \item \textbf{Key Features:}
    \begin{itemize}
        \item Potential for exponential state savings compared to classical multi-tape automata.
        \item Greater expressive power for recognizing languages with complex structural dependencies.
    \end{itemize}
    \item \textbf{Limitations:}
    \begin{itemize}
        \item Increased complexity and overhead with the number of tapes \( k \).
        \item Synchronization of multiple tape heads becomes progressively challenging.
    \end{itemize}
\end{itemize}

%%%%%%%%%%%%%%%%%%%%%%%%%%%%%%%%%%%%%%%%%%%%%%%%%%%%%%%%%%%%%%
\subsection*{Summary of \glspl{2qfa} Models}
\begin{itemize}
    \item \textbf{Standard \gls{2qfa}}: Allow unrestricted bidirectional head movement with full quantum control. They can recognize non-regular languages such as \( L_{\text{eq}} \) with bounded error, though they require quantum registers that scale with the input.
    \item \textbf{Hybrid \gls{2qcfa}}: Combine classical control with a small quantum register. These automata recognize non-regular languages (e.g., \( L_{\text{eq}} \) and palindromes) in polynomial time while using only a constant number of quantum states.
    \item \textbf{Multihead/Tape Extensions (\gls{2tqcfa} and \gls{ktqcfa})}: Extend the two-way model to multiple tapes, enhancing recognition power for languages with complex structural dependencies at the cost of increased synchronization and error-correction complexity.
\end{itemize}

%%%%%%%%%%%%%%%%%%%%%%%%%%%%%%%%%%%%%%%%%%%%%%%%%%%%%%%%%%%%%%
\subsection*{Detailed Hierarchy of \glspl{2qfa} Models}
\label{subsec:two-way-hierarchy}

\begin{figure}[ht]
\centering
\resizebox{\linewidth}{!}{%
\begin{tikzpicture}[node distance=1.2cm and 1.5cm, auto,
    every node/.style={draw, rectangle, align=center, minimum width=2.8cm, minimum height=0.9cm, font=\small}]

    % Level 1: Standard Two-Way QFA
    \node (2QFA) {2QFA};

    % Level 2: Hybrid Two-Way Models
    \node (2QCFA) [above=of 2QFA] {2QCFA};

    % Level 3: Multihead/Tape Extensions
    \node (2TQCFA) [above left=of 2QCFA, xshift=-0.5cm] {2TQCFA};
    \node (kTQCFA) [above right=of 2QCFA, xshift=0.5cm] {kTQCFA};

    % Draw vertical arrows (increasing expressive power)
    \draw[->] (2QFA) -- (2QCFA);
    \draw[->] (2QCFA) -- (2TQCFA);
    \draw[->] (2QCFA) -- (kTQCFA);
    
    % Horizontal arrow for generalization
    \draw[<->] (2TQCFA) -- (kTQCFA) node[midway, fill=white] {generalization};
\end{tikzpicture}
}
\caption{Detailed hierarchy of \gls{2qfa} models by expressive power. Vertical arrows indicate an increase in expressive power from standard to hybrid and then to multi-tape extensions, while the horizontal arrow denotes that kTQCFA generalize 2TQCFA.}
\label{fig:two-way-detailed-hierarchy}
\end{figure}
