\section{One-Way Quantum Automata Models}
\label{sec:one-way-qfas}

\glspl{1qfa} process the input tape in a single left-to-right pass. In this section, we review the most relevant models of \glspl{1qfa} and their respective language recognition capabilities. We begin with the standard \glsentryfull{mo-1qfa} and \glsentryfull{mm-1qfa} models, followed by the \gls{lqfa}, \glsentryfull{1qfac}, \gls{cl-1qfa}, \glsentryfull{a-qfa}, and \gls{mlqfa} models. A detailed comparison of these models is provided, highlighting their relative expressive power and computational efficiency.

\subsection{\glsentrylong{mo-1qfa}}
\label{subsec:mo-1qfa}
\begin{definition}[\glsentrylong{mo-1qfa}]
A \gls{mo-1qfa} is defined as:
\[
M = (Q, \Sigma, \delta, q_0, F)
\]
\textbf{where:}
\begin{itemize}
    \item \( Q \) is the finite set of quantum states,
    \item \( \Sigma \) is the input alphabet,
    \item \( \delta \) is the transition function implemented by unitary operators,
    \item \( q_0 \in Q \) is the initial quantum state,
    \item \( F \subset Q \) is the set of accepting states.
\end{itemize}
\end{definition}

\paragraph{Operation:}  
The automaton processes the input string by sequentially applying unitary operators determined by \( \delta \) for each symbol in a left-to-right pass. A single projective measurement is performed at the end, with the outcome over \( F \) deciding acceptance.

%TODO: add explaination of the language
\paragraph{Language Acceptance:}  
Accepts \textbf{\glspl{rev}} (e.g., \( L_{\text{mod}} = \{a^{kp} \mid k \geq 0\} \)), a strict subset of regular languages \cite{kondacs1997power}.
\paragraph{Advantages:}
%TODO: transform the bullet points into a discussive paragraph
\begin{itemize}
    \item Minimal quantum resources due to a single measurement.
    \item Coherence is maintained throughout the computation.
\end{itemize}

\paragraph{Limitations:}
%TODO: transform the bullet points into a discussive paragraph
\begin{itemize}
    \item Limited to reversible regular languages.
    \item Reliance on a single final measurement may restrict recognition of more complex languages.
\end{itemize}

\subsection{\glsentrylong{mm-1qfa}}
\label{subsec:mm-1qfa}
\begin{definition}[\glsentrylong{mm-1qfa}]
A \gls{mm-1qfa} is defined as 
\[
M = (Q, \Sigma, \delta, q_0, Q_{acc}, Q_{rej}, Q_{non})
\]
\textbf{where:}
\begin{itemize}
    \item \( Q \) is the finite set of quantum states,
    \item \( \Sigma \) is the input alphabet,
    \item \( \delta \) is the unitary transition function,
    \item \( q_0 \in Q \) is the initial state,
    \item \( Q_{acc} \subset Q \) is the set of accepting states,
    \item \( Q_{rej} \subset Q \) is the set of rejecting states,
    \item \( Q_{non} \subset Q \) is the set of non-halting states.
\end{itemize}
\end{definition}

\paragraph{Operation:}  
For each input symbol \( \sigma \), the automaton applies the corresponding unitary operator \( U_\sigma \) and immediately performs a measurement. If the measurement yields a state in \( Q_{acc} \) or \( Q_{rej} \), the automaton halts; otherwise, it continues processing.

\paragraph{Language Acceptance:}  
Recognises \textbf{a subclass of regular languages with bounded error} (e.g., \( L = \{w \mid |w|_a \equiv 1 \mod 2\} \)) \cite{ambainis1998}.

\paragraph{Advantages:}
%TODO: transform the bullet points into a discussive paragraph
\begin{itemize}
    \item Bounded error recognition via intermediate measurements.
    \item Exponential state reduction for specific language families.
\end{itemize}

\paragraph{Limitations:}
%TODO: transform the bullet points into a discussive paragraph
\begin{itemize}
    %TODO: is it a subset of regular languages?
    \item Limited to regular languages.
    \item Increased complexity due to frequent measurements.
\end{itemize}

\subsection{\glsentrylong{lqfa}}
\label{subsec:lqfa}
\begin{definition}[\glsentrylong{lqfa} \cite{ambainis1998}]
A \gls{lqfa} is defined as 
\[
M = (Q, \Sigma, \delta, q_0, F)
\]
\textbf{where:}
\begin{itemize}
    \item \( Q \) is the finite set of quantum states,
    \item \( \Sigma \) is the input alphabet,
    \item \( \delta \) is the transition function implementing unitary steps,
    \item \( q_0 \in Q \) is the initial state,
    \item \( F \subset Q \) is the set of accepting states.
\end{itemize}
\end{definition}

\paragraph{Operation:}  
After each input symbol, a unitary transformation dictated by \( \delta \) is applied, immediately followed by a measurement. This immediate measurement ensures that only critical quantum states are preserved for further computation.

\paragraph{Language Acceptance:}  
Accepts a \textbf{subset of regular languages} (e.g., certain parity languages) where the measurement process preserves the necessary states.

%TODO: not complete
\paragraph{Advantages:}
%TODO: transform the bullet points into a discussive paragraph
\begin{itemize}
    \item Integrates quantum evolution with classical-like decision making.
    \item Reduced state complexity for symmetric languages.
\end{itemize}

\paragraph{Limitations:}
%TODO: transform the bullet points into a discussive paragraph
\begin{itemize}
    \item Susceptible to premature state collapse due to frequent measurements.
    \item Not closed under concatenation.
\end{itemize}

\subsection{\glsentrylong{1qfac}}
\label{subsec:1qfac}
\begin{definition}[\glsentrylong{1qfac}]
A \gls{1qfac} is defined as 
\[
M = (S, Q, \Sigma, \delta, \mu, s_0, q_0, F)
\]
\textbf{where:}
\begin{itemize}
    \item \( S \) is the finite set of classical states,
    \item \( Q \) is the finite set of quantum states,
    \item \( \Sigma \) is the input alphabet,
    \item \( \delta \) is the classical transition function over \( S \),
    \item \( \mu \) is the quantum operation function acting on \( Q \),
    \item \( s_0 \in S \) is the initial classical state,
    \item \( q_0 \in Q \) is the initial quantum state,
    \item \( F \subset S \times Q \) is the set of accepting state pairs.
\end{itemize}
\end{definition}

\paragraph{Operation:}  
The automaton proceeds in two intertwined stages:
\begin{enumerate}
    \item A classical state transition is performed using \( \delta \).
    \item The quantum register is simultaneously updated via \( \mu \).
\end{enumerate}
A final joint measurement over the composite state \( (s, q) \) determines acceptance.

\paragraph{Language Acceptance:}  
Recognises \textbf{all regular languages} while requiring only \( O(\log n) \) quantum states.

\paragraph{Advantages:}
%TODO: transform the bullet points into a discussive paragraph
\begin{itemize}
    \item Efficient simulation of classical automata with quantum enhancements.
    \item Maintains a small quantum register.
\end{itemize}

\paragraph{Limitations:}
%TODO: transform the bullet points into a discussive paragraph
\begin{itemize}
    \item Equivalence checking is undecidable \cite{hirvensalo2008}.
    \item The quantum component remains sensitive to decoherence.
\end{itemize}
\cite{qiu2009}

\subsection{\glsentrylong{cl-1qfa}}
\label{subsec:cl-1qfa}
\begin{definition}[\glsentrylong{cl-1qfa}]
A \gls{cl-1qfa} is defined as 
\[
M = (Q, \Sigma, \delta, q_0, \mathcal{L})
\]
\textbf{where:}
\begin{itemize}
    \item \( Q \) is the finite set of quantum states,
    \item \( \Sigma \) is the input alphabet,
    \item \( \delta \) is the quantum transition function,
    \item \( q_0 \in Q \) is the initial quantum state,
    \item \( \mathcal{L} \subseteq \Sigma^* \) is a predetermined control language.
\end{itemize}
\end{definition}

\paragraph{Operation:}  
The automaton evolves its quantum state according to \( \delta \) while outcomes from intermediate measurements are interpreted using the control language \( \mathcal{L} \) to guide acceptance or rejection.

\paragraph{Language Acceptance:}  
Designed to recognise \textbf{regular languages closed under Boolean operations} (e.g., \( L = a\Sigma^* \cup \Sigma^* b \)).

\paragraph{Advantages:}
%TODO: transform the bullet points into a discussive paragraph
\begin{itemize}
    \item Strong closure properties via Boolean operations.
    \item Modularity through separation of quantum evolution and classical control.
\end{itemize}

\paragraph{Limitations:}
%TODO: transform the bullet points into a discussive paragraph
\begin{itemize}
    \item Precomputation of \( \mathcal{L} \) introduces overhead.
    \item Integration of two paradigms increases overall complexity.
\end{itemize}

\subsection{\glsentrylong{a-qfa}}
\label{subsec:a-qfa}
\begin{definition}[\glsentrylong{a-qfa}]
An \gls{a-qfa} is defined as 
\[
M = (Q, \Sigma, \delta, q_0, F)
\]
with an expanded Hilbert space 
\[
\mathcal{H}_{\text{total}} = \mathcal{H} \otimes \mathcal{H}_{\text{ancilla}},
\]
\textbf{where:}
\begin{itemize}
    \item \( Q \) is the finite set of quantum states in \( \mathcal{H} \),
    \item \( \Sigma \) is the input alphabet,
    \item \( \delta \) is the quantum transition function,
    \item \( q_0 \in Q \) is the initial quantum state,
    \item \( F \subset Q \) is the set of accepting states,
    \item \( \mathcal{H}_{\text{ancilla}} \) represents the ancilla qubits used to enhance computation.
\end{itemize}
\end{definition}

\paragraph{Operation:}  
Ancilla qubits are employed to simulate nondeterminism or enhance interference, with entanglement between the main system and ancilla allowing the automaton to explore additional computational pathways.

\paragraph{Language Acceptance:}  
Recognises \textbf{all regular languages and some context-free languages} (e.g., \( L = \{a^n b^n \mid n \geq 0\} \)) with one-sided error \cite{yakaryilmaz2011}.

\paragraph{Advantages:}
%TODO: transform the bullet points into a discussive paragraph
\begin{itemize}
    \item Enables exponential state reduction via quantum entanglement.
    \item Extends recognition capabilities beyond classical regular languages.
\end{itemize}

\paragraph{Limitations:}
%TODO: transform the bullet points into a discussive paragraph
\begin{itemize}
    \item Increased complexity in managing ancilla qubits.
    \item Enhanced risk of error propagation between subsystems.
\end{itemize}

\subsection{\glsentrylong{mlqfa}}
\label{subsec:ml-qfa}
\begin{definition}[\glsentrylong{mlqfa}]
A \gls{mlqfa} is defined as 
\[
M = (Q, \Sigma, \delta, q_0, F)
\]
\textbf{where:}
\begin{itemize}
    \item \( Q \) is the finite set of quantum states,
    \item \( \Sigma \) is the input alphabet,
    \item \( \delta \) is the transition function applied to blocks of \( k \) symbols,
    \item \( q_0 \in Q \) is the initial quantum state,
    \item \( F \subset Q \) is the set of accepting states,
    \item \( k \geq 2 \) is the block size used for processing the input.
\end{itemize}
\end{definition}

\paragraph{Operation:}  
The automaton processes the input in blocks of \( k \) symbols, applying a unitary operator to each block, thereby capturing long-range dependencies within the input.

\paragraph{Language Acceptance:}  
Recognises \textbf{context-free languages} (e.g., \( L = \{ww^R \mid w \in \Sigma^*\} \)) \cite{ravikumar2003}.

\paragraph{Advantages:}
%TODO: transform the bullet points into a discussive paragraph
\begin{itemize}
    \item Efficient parallel processing of input blocks.
    \item Facilitates polynomial-time recognition of complex patterns.
\end{itemize}

\paragraph{Limitations:}
%TODO: transform the bullet points into a discussive paragraph
\begin{itemize}
    \item Exponential growth in state complexity with increasing \( k \).
    \item Design of unitary operators for large \( k \) is challenging.
\end{itemize}

\subsection*{Detailed Hierarchy of \gls{1qfa} Models}
\label{subsec:hierarchy-diagram}

%TODO: enhance the diagram
\begin{figure}[ht]
    \centering
    \begin{tikzpicture}[node distance=1.5cm]
        \node (MO1QFA) {\gls{mo-1qfa}};
        \node (MM1QFA) [above=of MO1QFA] {\gls{mm-1qfa}};
        \node (LQFA) [right=of MO1QFA] {\gls{lqfa}};
        \node (1QFAC) [above=of MM1QFA] {\gls{1qfac}};
        \node (CL1QFA) [above=of LQFA] {\gls{cl-1qfa}};
        \node (A-QFA) [above=of CL1QFA] {\gls{a-qfa}};
        \node (MLQFA) [above=of 1QFAC] {\gls{mlqfa}};
        
        \draw[->] (MO1QFA) -- (MM1QFA);
        \draw[->] (MO1QFA) -- (LQFA);
        \draw[->] (MM1QFA) -- (1QFAC);
        \draw[->] (LQFA) -- (CL1QFA);
        \draw[->] (CL1QFA) -- (A-QFA);
    \end{tikzpicture}
    \caption{Hierarchy of one-way \gls{qfa} models. Arrows indicate strict increases in expressive power.}
    \label{fig:qfa-hierarchy}
\end{figure}

%TODO: make this comparison more effective
\subsection*{Comparison to Classical Finite Automata}
\label{subsec:classical-comparison}
Quantum models exhibit both advantages and limitations compared to their classical counterparts:
\begin{itemize}
    \item \textbf{State efficiency}: \gls{mo-1qfa} require fewer states for periodic languages \cite{kondacs1997power}, but cannot recognise all regular languages.
    \item \textbf{Error bounds}: \gls{mm-1qfa} achieve bounded error for specific languages \cite{ambainis1998}, unlike deterministic finite automata.
    \item \textbf{Closure properties}: Classical automata are closed under all Boolean operations \cite{rabin1959}, while most \gls{qfa} variants are not.
\end{itemize}

%TODO: change occurrencies of regular languages to regs
