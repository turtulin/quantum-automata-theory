%%%%%%%%%%%%%%%%%%%%%%%%%%%%%%%%%%%%%%%%%%%%%%%%%%%%%%%%%%%%%%
\section{\gls{1qfa}}
\label{sec:one-way-qfas}

One-way \glspl{qfa} process the input tape in a single left-to-right pass. In this section, we review the standard models of one-way \gls{qfa}—including both the measure-once and measure-many variants—as well as a notable alternative model, the \gls{lqfa}. For each model, we detail a uniform definition that specifies the symbols and their meanings, the operational mechanism, language acceptance criteria, advantages, and limitations. A schematic summary at the end of each subsection highlights the key features.

% %%%%%%%%%%%%%%%%%%%%%%%%%%%%%%%%%%%%%%%%%%%%%%%%%%%%%%%%%%%%%%
% \subsection{Standard Models}
% \label{subsec:standard-models}

% This subsection covers the primary models of one-way quantum finite automata. We discuss the measure-once, measure-many, and LQFA models below.

%%%%%%%%%%%%%%%%%%%%%%%%%%%%%%%%%%%%%%%%%%%%%%%%%%%%%%%%%%%%%%
\subsection{\acrfull{mo-1qfa}}
\label{subsec:mo-1qfa}
\begin{definition}[\gls{mo-1qfa}]
A measure-once one-way quantum finite automaton is defined as 
\[
M = (Q, \Sigma, \delta, q_0, F)
\]
\textbf{where:}
\begin{itemize}
    \item \( Q \) is the finite set of quantum states.
    \item \( \Sigma \) is the input alphabet.
    \item \( \delta \) is the transition function implemented by unitary operators.
    \item \( q_0 \in Q \) is the initial quantum state.
    \item \( F \subset Q \) is the set of accepting states.
\end{itemize}
\end{definition}

\paragraph{Operation:}  
The automaton processes the input string by sequentially applying unitary operators determined by \( \delta \) for each symbol in a left-to-right pass. A single projective measurement is performed at the end, with the outcome over \( F \) deciding acceptance.

\paragraph{Language Acceptance:}  
Accepts \textbf{reversible regular languages} (e.g., \( L_{\text{mod}} = \{a^{kp} \mid k \geq 0\} \)).

\paragraph{Advantages:}
\begin{itemize}
    \item Minimal quantum resources due to a single measurement.
    \item Coherence is maintained throughout the computation.
\end{itemize}

\paragraph{Limitations:}
\begin{itemize}
    \item Limited to reversible regular languages.
    \item Reliance on a single final measurement may restrict recognition of more complex languages.
\end{itemize}

%%%%%%%%%%%%%%%%%%%%%%%%%%%%%%%%%%%%%%%%%%%%%%%%%%%%%%%%%%%%%%
\subsection{\acrfull{mm-1qfa}}
\label{subsec:mm-1qfa}
\begin{definition}[\gls{mm-1qfa}]
A measure-many one-way quantum finite automaton is defined as 
\[
M = (Q, \Sigma, \delta, q_0, Q_{acc}, Q_{rej}, Q_{non})
\]
\textbf{where:}
\begin{itemize}
    \item \( Q \) is the finite set of quantum states.
    \item \( \Sigma \) is the input alphabet.
    \item \( \delta \) is the unitary transition function.
    \item \( q_0 \in Q \) is the initial state.
    \item \( Q_{acc} \subset Q \) is the set of accepting states.
    \item \( Q_{rej} \subset Q \) is the set of rejecting states.
    \item \( Q_{non} \subset Q \) is the set of non-halting states.
\end{itemize}
\end{definition}

\paragraph{Operation:}  
For each input symbol \( \sigma \), the automaton applies the corresponding unitary operator \( U_\sigma \) and immediately performs a measurement. If the measurement yields a state in \( Q_{acc} \) or \( Q_{rej} \), the automaton halts; otherwise, it continues processing.

\paragraph{Language Acceptance:}  
Recognises \textbf{all regular languages with bounded error} (e.g., \( L = \{w \mid |w|_a \equiv |w|_b \mod 2\} \)).

\paragraph{Advantages:}
\begin{itemize}
    \item Bounded error recognition via intermediate measurements.
    \item Exponential state reduction for specific language families.
\end{itemize}

\paragraph{Limitations:}
\begin{itemize}
    \item Limited to regular languages.
    \item Increased complexity due to frequent measurements.
\end{itemize}

%%%%%%%%%%%%%%%%%%%%%%%%%%%%%%%%%%%%%%%%%%%%%%%%%%%%%%%%%%%%%%
\subsection{\gls{lqfa}}
\label{subsec:lqfa}
\begin{definition}[\gls{lqfa}]
A Latvian quantum finite automaton is defined as 
\[
M = (Q, \Sigma, \delta, q_0, F)
\]
\textbf{where:}
\begin{itemize}
    \item \( Q \) is the finite set of quantum states.
    \item \( \Sigma \) is the input alphabet.
    \item \( \delta \) is the transition function implementing unitary steps.
    \item \( q_0 \in Q \) is the initial state.
    \item \( F \subset Q \) is the set of accepting states.
\end{itemize}
\end{definition}

\paragraph{Operation:}  
After each input symbol, a unitary transformation dictated by \( \delta \) is applied, immediately followed by a measurement. This immediate measurement ensures that only critical quantum states are preserved for further computation.

\paragraph{Language Acceptance:}  
Accepts a \textbf{subset of regular languages} (e.g., certain parity languages) where the measurement process preserves the necessary states.

\paragraph{Advantages:}
\begin{itemize}
    \item Integrates quantum evolution with classical-like decision making.
    \item Reduced state complexity for symmetric languages.
\end{itemize}

\paragraph{Limitations:}
\begin{itemize}
    \item Susceptible to premature state collapse due to frequent measurements.
    \item Not closed under concatenation.
\end{itemize}

%%%%%%%%%%%%%%%%%%%%%%%%%%%%%%%%%%%%%%%%%%%%%%%%%%%%%%%%%%%%%%
% \subsubsection*{Schematic Summary of Standard Models}
% \label{subsec:standard-summary}
% \begin{itemize}
%     \item \textbf{MO-1QFA:} Uniform definition with one final measurement; efficient but limited to reversible regular languages.
%     \item \textbf{MM-1QFA:} Uniform definition with intermediate measurements; achieves bounded error for regular languages at the expense of complexity.
%     \item \textbf{LQFA:} Uniform definition with immediate measurements; blends quantum and classical control but is sensitive to measurement effects.
% \end{itemize}

%%%%%%%%%%%%%%%%%%%%%%%%%%%%%%%%%%%%%%%%%%%%%%%%%%%%%%%%%%%%%%
% \subsection{Hybrid Models}
% \label{subsec:hybrid-models}

% Hybrid models combine classical control with quantum operations. The following definitions use a uniform structure that specifies all symbols.

%%%%%%%%%%%%%%%%%%%%%%%%%%%%%%%%%%%%%%%%%%%%%%%%%%%%%%%%%%%%%%
\subsection{\acrfull{1qfac}}
\label{subsec:1qfac}
\begin{definition}[\gls{1qfac}]
A one-way quantum finite automaton with classical states is defined as 
\[
M = (S, Q, \Sigma, \delta, \mu, s_0, q_0, F)
\]
\textbf{where:}
\begin{itemize}
    \item \( S \) is the finite set of classical states.
    \item \( Q \) is the finite set of quantum states.
    \item \( \Sigma \) is the input alphabet.
    \item \( \delta \) is the classical transition function over \( S \).
    \item \( \mu \) is the quantum operation function acting on \( Q \).
    \item \( s_0 \in S \) is the initial classical state.
    \item \( q_0 \in Q \) is the initial quantum state.
    \item \( F \subset S \times Q \) is the set of accepting state pairs.
\end{itemize}
\end{definition}

\paragraph{Operation:}  
The automaton proceeds in two intertwined stages:
\begin{enumerate}
    \item A classical state transition is performed using \( \delta \).
    \item The quantum register is simultaneously updated via \( \mu \).
\end{enumerate}
A final joint measurement over the composite state \( (s, q) \) determines acceptance.

\paragraph{Language Acceptance:}  
Recognises \textbf{all regular languages} while requiring only \( O(\log n) \) quantum states.

\paragraph{Advantages:}
\begin{itemize}
    \item Efficient simulation of classical automata with quantum enhancements.
    \item Maintains a small quantum register.
\end{itemize}

\paragraph{Limitations:}
\begin{itemize}
    \item Equivalence checking is undecidable.
    \item The quantum component remains sensitive to decoherence.
\end{itemize}

%%%%%%%%%%%%%%%%%%%%%%%%%%%%%%%%%%%%%%%%%%%%%%%%%%%%%%%%%%%%%%
\subsection{\acrfull{cl-1qfa}}
\label{subsec:cl-1qfa}
\begin{definition}[\gls{cl-1qfa}]
A control-language-based quantum finite automaton is defined as 
\[
M = (Q, \Sigma, \delta, q_0, \mathcal{L})
\]
\textbf{where:}
\begin{itemize}
    \item \( Q \) is the finite set of quantum states.
    \item \( \Sigma \) is the input alphabet.
    \item \( \delta \) is the quantum transition function.
    \item \( q_0 \in Q \) is the initial quantum state.
    \item \( \mathcal{L} \subseteq \Sigma^* \) is a predetermined control language.
\end{itemize}
\end{definition}

\paragraph{Operation:}  
The automaton evolves its quantum state according to \( \delta \) while outcomes from intermediate measurements are interpreted using the control language \( \mathcal{L} \) to guide acceptance or rejection.

\paragraph{Language Acceptance:}  
Designed to recognise \textbf{regular languages closed under Boolean operations} (e.g., \( L = a\Sigma^* \cup \Sigma^* b \)).

\paragraph{Advantages:}
\begin{itemize}
    \item Strong closure properties via Boolean operations.
    \item Modularity through separation of quantum evolution and classical control.
\end{itemize}

\paragraph{Limitations:}
\begin{itemize}
    \item Precomputation of \( \mathcal{L} \) introduces overhead.
    \item Integration of two paradigms increases overall complexity.
\end{itemize}

%%%%%%%%%%%%%%%%%%%%%%%%%%%%%%%%%%%%%%%%%%%%%%%%%%%%%%%%%%%%%%
% \subsubsection*{Schematic Summary of Hybrid Models}
% \label{subsec:hybrid-summary}
% \begin{itemize}
%     \item \textbf{1QFAC:} Uniformly defined hybrid model combining classical state transitions with quantum operations; efficient for regular languages.
%     \item \textbf{CL-1QFA:} Uniform definition leveraging a control language; strong closure properties but incurs precomputation overhead.
% \end{itemize}

% %%%%%%%%%%%%%%%%%%%%%%%%%%%%%%%%%%%%%%%%%%%%%%%%%%%%%%%%%%%%%%
% \subsection{Enhanced Models}
% \label{subsec:enhanced-models}

% Enhanced models extend basic QFA frameworks by incorporating advanced mechanisms. The following definitions are provided in a uniform manner.

%%%%%%%%%%%%%%%%%%%%%%%%%%%%%%%%%%%%%%%%%%%%%%%%%%%%%%%%%%%%%%



%%%%%%%%%%%%%%%%%%%%%%%%%%%%%%%%%%%%%%%%%%%%%%%%%%%%%%%%%%%%%%


%%%%%%%%%%%%%%%%%%%%%%%%%%%%%%%%%%%%%%%%%%%%%%%%%%%%%%%%%%%%%%
\subsection{\acrfull{a-qfa}}
\label{subsec:a-qfa}
\begin{definition}[\gls{a-qfa}]
An ancilla-based quantum finite automaton is defined as 
\[
M = (Q, \Sigma, \delta, q_0, F)
\]
with an expanded Hilbert space 
\[
\mathcal{H}_{\text{total}} = \mathcal{H} \otimes \mathcal{H}_{\text{ancilla}},
\]
\textbf{where:}
\begin{itemize}
    \item \( Q \) is the finite set of quantum states in \( \mathcal{H} \).
    \item \( \Sigma \) is the input alphabet.
    \item \( \delta \) is the quantum transition function.
    \item \( q_0 \in Q \) is the initial quantum state.
    \item \( F \subset Q \) is the set of accepting states.
    \item \( \mathcal{H}_{\text{ancilla}} \) represents the ancilla qubits used to enhance computation.
\end{itemize}
\end{definition}

\paragraph{Operation:}  
Ancilla qubits are employed to simulate nondeterminism or enhance interference, with entanglement between the main system and ancilla allowing the automaton to explore additional computational pathways.

\paragraph{Language Acceptance:}  
Recognises \textbf{all regular languages and some context-free languages} (e.g., \( L = \{a^n b^n\} \)) with one-sided error.

\paragraph{Advantages:}
\begin{itemize}
    \item Enables exponential state reduction via quantum entanglement.
    \item Extends recognition capabilities beyond classical regular languages.
\end{itemize}

\paragraph{Limitations:}
\begin{itemize}
    \item Increased complexity in managing ancilla qubits.
    \item Enhanced risk of error propagation between subsystems.
\end{itemize}

%%%%%%%%%%%%%%%%%%%%%%%%%%%%%%%%%%%%%%%%%%%%%%%%%%%%%%%%%%%%%%
% \subsubsection*{Schematic Summary of Enhanced Models}
% \label{subsec:enhanced-summary}
% \begin{itemize}
%     \item \textbf{EQFA:} Uses superoperators for non-unitary evolution; robust against noise but with undecidable equivalence.
%     \item \textbf{OTQFA:} Employs CPTP maps for realistic open-system dynamics; unifies quantum and classical aspects with resource overhead.
%     \item \textbf{A-QFA:} Incorporates ancilla qubits to boost computational power; recognises broader language classes with increased complexity.
% \end{itemize}

% %%%%%%%%%%%%%%%%%%%%%%%%%%%%%%%%%%%%%%%%%%%%%%%%%%%%%%%%%%%%%%
% \subsection{Advanced Variants}
% \label{subsec:advanced-variants}

% Advanced variants extend computational power with modifications such as limited bidirectional movement and block processing. Uniform definitions are provided below.

%%%%%%%%%%%%%%%%%%%%%%%%%%%%%%%%%%%%%%%%%%%%%%%%%%%%%%%%%%%%%%


%%%%%%%%%%%%%%%%%%%%%%%%%%%%%%%%%%%%%%%%%%%%%%%%%%%%%%%%%%%%%%
\subsection{\acrfull{mlqfa}}
\label{subsec:ml-qfa}
\begin{definition}[\gls{mlqfa}]
A multi-letter quantum finite automaton is defined as 
\[
M = (Q, \Sigma, \delta, q_0, F)
\]
\textbf{where:}
\begin{itemize}
    \item \( Q \) is the finite set of quantum states.
    \item \( \Sigma \) is the input alphabet.
    \item \( \delta \) is the transition function applied to blocks of \( k \) symbols.
    \item \( q_0 \in Q \) is the initial quantum state.
    \item \( F \subset Q \) is the set of accepting states.
    \item \( k \geq 2 \) is the block size used for processing the input.
\end{itemize}
\end{definition}

\paragraph{Operation:}  
The automaton processes the input in blocks of \( k \) symbols, applying a unitary operator to each block, thereby capturing long-range dependencies within the input.

\paragraph{Language Acceptance:}  
Recognises \textbf{context-sensitive languages} (e.g., \( L = \{ww^R\} \)) by leveraging parallel processing of symbol blocks.

\paragraph{Advantages:}
\begin{itemize}
    \item Efficient parallel processing of input blocks.
    \item Facilitates polynomial-time recognition of complex patterns.
\end{itemize}

\paragraph{Limitations:}
\begin{itemize}
    \item Exponential growth in state complexity with increasing \( k \).
    \item Design of unitary operators for large \( k \) is challenging.
\end{itemize}

%%%%%%%%%%%%%%%%%%%%%%%%%%%%%%%%%%%%%%%%%%%%%%%%%%%%%%%%%%%%%%
% \subsubsection*{Schematic Summary of Advanced Variants}
% \label{subsec:advanced-summary}
% \begin{itemize}
%     \item \textbf{1.5QFA:} Uniformly defined model with limited leftward motion; bridges one-way and two-way models for bounded error recognition of non-regular languages.
%     \item \textbf{MLQFA:} Uniform definition for block processing; captures long-range dependencies in context-sensitive languages at the expense of state complexity.
% \end{itemize}

%%%%%%%%%%%%%%%%%%%%%%%%%%%%%%%%%%%%%%%%%%%%%%%%%%%%%%%%%%%%%%
\subsection*{Detailed Hierarchy of One-Way \gls{qfa} Models}
\label{subsec:hierarchy-diagram}

\begin{figure}[ht]
    \centering
    \begin{tikzpicture}[node distance=1.5cm]
        \node (MO1QFA) {\gls{mo-1qfa}};
        \node (MM1QFA) [above=of MO1QFA] {\gls{mm-1qfa}};
        \node (LQFA) [right=of MO1QFA] {\gls{lqfa}};
        \node (1QFAC) [above=of MM1QFA] {\gls{1qfac}};
        \node (CL1QFA) [above=of LQFA] {\gls{cl-1qfa}};
        \node (EQFA) [above=of 1QFAC] {\gls{eqfa}};
        \node (OTQFA) [right=of EQFA] {\gls{otqfa}};
        \node (A-QFA) [above=of CL1QFA] {\gls{a-qfa}};
        \node (OnePointFiveQFA) [above=of EQFA] {\gls{1.5qfa}};
        \node (MLQFA) [above=of OTQFA] {\gls{mlqfa}};
        
        % Vertical edges
        \draw[->] (MO1QFA) -- (MM1QFA);
        \draw[->] (MO1QFA) -- (LQFA);
        \draw[->] (MM1QFA) -- (1QFAC);
        \draw[->] (LQFA) -- (CL1QFA);
        \draw[->] (1QFAC) -- (EQFA);
        \draw[->] (1QFAC) -- (OTQFA);
        \draw[->] (CL1QFA) -- (A-QFA);
        \draw[->] (EQFA) -- (OnePointFiveQFA);
        \draw[->] (OTQFA) -- (MLQFA);
    \end{tikzpicture}
    \caption{Hierarchy of one-way \gls{qfa} models. Arrows indicate strict increases in expressive power.}
    \label{fig:qfa-hierarchy}
\end{figure}

\noindent\textbf{Schematic Summary of the Hierarchy:}
\begin{itemize}
    \item The hierarchy progresses from basic models (MO-1QFA) to more expressive variants (MLQFA).
    \item Each arrow represents an increase in computational power or model complexity.
    \item Hybrid, enhanced, and advanced variants build upon foundational models to recognise broader classes of languages.
\end{itemize}
