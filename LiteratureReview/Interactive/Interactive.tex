
\section{Interactive Quantum Automata}
\label{sec:interactive-quantum}

Interactive quantum automata integrate communication between a resource-limited quantum verifier and an (often unbounded) prover.

%%%%%%%%%%%%%%%%%%%%%%%%%%%%%%%%%%%%%%%%%%%%%%%%%%%%%%%%%%%%%%
\subsubsection{Quantum Interactive Proof Systems (QIP)}
\label{sssec:qip}
\begin{definition}[Quantum Interactive Proof (QIP) System]
A Quantum Interactive Proof (QIP) system consists of a polynomial-time quantum verifier \( V \) and an unbounded quantum prover \( P \) who exchange messages via a quantum channel. The verifier is typically modeled as a QFA with limited memory. A system with \( k \) rounds of interaction is denoted as \(\text{QIP}(k)\) \cite{nishimura2009application, zheng2015power}.
\end{definition}

\textbf{Operation}: The verifier processes an input \( w \) by alternating between local quantum operations and rounds of communication with the prover. The overall transition is governed by a function
\[
\delta: Q \times \Sigma \times \Gamma \to \mathbb{C}^{\, Q \times Q},
\]
where \( \Gamma \) is the communication alphabet. Acceptance is determined by measuring the verifier’s final state after the prescribed rounds \cite{zheng2015power}.

\textbf{Key Features}:
\begin{itemize}
    \item QIP systems have been shown to be equivalent in power to PSPACE in the classical setting.
    \item QFA-based verifiers (e.g., using 2QFA) can recognize languages beyond the regular class with bounded error.
\end{itemize}

\textbf{Limitations}:
\begin{itemize}
    \item Requires precise control of quantum communication channels.
    \item The verifier’s state complexity may scale with the input for complex languages.
\end{itemize}

%%%%%%%%%%%%%%%%%%%%%%%%%%%%%%%%%%%%%%%%%%%%%%%%%%%%%%%%%%%%%%
\subsubsection{Quantum Merlin-Arthur (QMIP) Systems}
\label{sssec:qmip}
\begin{definition}[QMIP System]
Quantum Merlin-Arthur (QMIP) systems extend QIP by involving multiple non-communicating provers. Formally, a QMIP system with \( k \) provers is denoted as \(\text{QMIP}(k)\). In these systems, the verifier receives quantum witness states from each prover and performs a quantum verification procedure \cite{scegulnaja2010postselection, yamakami2014constant}.
\end{definition}

\textbf{Operation}: The verifier (modeled as a 2QFA-based machine) receives quantum witness states from the provers and processes them using a quantum circuit. The final state is measured to decide acceptance. The provers are not allowed to communicate with each other, and the verification is non-adaptive \cite{scegulnaja2010postselection}.

\textbf{Key Features}:
\begin{itemize}
    \item QMIP systems capture complexity classes such as MIP\(^*\), and are strictly more powerful than single-prover QIP systems.
    \item They can recognize languages (e.g., the palindrome language \( L_{\text{pal}} = \{ww^R \mid w \in \{0,1\}^*\} \)) with exponential state savings compared to classical multiprover interactive proofs.
\end{itemize}

\textbf{Limitations}:
\begin{itemize}
    \item Managing entanglement among multiple provers complicates the verification process.
    \item Practical implementations are highly sensitive to noise and decoherence.
\end{itemize}
