\section{Interactive Quantum Automata}
\label{sec:interactive-quantum}

Interactive quantum automata integrate communication between a resource-limited quantum verifier and an (often unbounded) prover. These models extend standard \gls{qfa} by introducing interactive protocols where the verifier and prover exchange quantum messages over multiple rounds.

%%%%%%%%%%%%%%%%%%%%%%%%%%%%%%%%%%%%%%%%%%%%%%%%%%%%%%%%%%%%%%
\subsubsection{\gls{qip} Systems}
\label{sssec:qip}
\begin{definition}[\gls{qip} System]
A \gls{qip} system consists of a polynomial-time quantum verifier \( V \) and an unbounded quantum prover \( P \) who exchange messages via a quantum channel. The verifier is typically modeled as a \gls{qfa} with limited memory. A system with \( k \) rounds of interaction is denoted as \(\text{QIP}(k)\) \cite{nishimura2009application, zheng2015power}.
\end{definition}

\textbf{Operation}:  
The verifier processes an input \( w \) by alternating between local quantum operations and rounds of communication with the prover. The overall transition is governed by a function
\[
\delta: Q \times \Sigma \times \Gamma \to \mathbb{C}^{\, Q \times Q},
\]
where \( \Gamma \) is the communication alphabet. After the prescribed rounds of interaction, the verifier performs a measurement on its final state to decide acceptance.

\paragraph{Language Acceptance:}  
A QIP system accepts an input \( w \) if the verifier’s final measurement yields an accepting outcome with probability exceeding a specified threshold (typically above \( \frac{1}{2} \)), thereby ensuring bounded-error recognition of the language.

\paragraph{Key Features and Limitations:}
\begin{itemize}
    \item \textbf{Key Features:}
    \begin{itemize}
        \item \gls{qip} systems have been shown to be equivalent in power to PSPACE in the classical setting.
        \item \gls{qfa}-based verifiers (e.g., using \gls{2qfa}) can recognize languages beyond the regular class with bounded error.
    \end{itemize}
    \item \textbf{Limitations:}
    \begin{itemize}
        \item Precise control of quantum communication channels is required.
        \item The verifier’s state complexity may scale with the input for more complex languages.
    \end{itemize}
\end{itemize}

%%%%%%%%%%%%%%%%%%%%%%%%%%%%%%%%%%%%%%%%%%%%%%%%%%%%%%%%%%%%%%
\subsubsection{\gls{qmip} Systems}
\label{sssec:qmip}
\begin{definition}[\gls{qmip} System]
    \gls{qmip} systems extend QIP by involving multiple non-communicating provers. Formally, a QMIP system with \( k \) provers is denoted as \(\text{QMIP}(k)\). In these systems, the verifier receives quantum witness states from each prover and performs a quantum verification procedure \cite{scegulnaja2010postselection, yamakami2014constant}.
\end{definition}

\textbf{Operation}:  
The verifier, typically modeled as a \gls{2qfa}-based machine, receives quantum witness states from the provers and processes them using a quantum circuit. A final measurement is performed on the verifier’s state to decide acceptance. The provers do not communicate with each other, and the verification process is non-adaptive.

\paragraph{Language Acceptance:}  
A QMIP system accepts an input if the combined verification procedure yields an accepting outcome with probability above a predetermined threshold, thereby ensuring bounded-error recognition of the target language.

\paragraph{Key Features and Limitations:}
\begin{itemize}
    \item \textbf{Key Features:}
    \begin{itemize}
        \item QMIP systems capture complexity classes such as MIP\(^*\) and are strictly more powerful than single-prover QIP systems.
        \item They can recognize languages (e.g., the palindrome language \( L_{\text{pal}} = \{ww^R \mid w \in \{0,1\}^*\} \)) with exponential state savings compared to classical multiprover interactive proofs.
    \end{itemize}
    \item \textbf{Limitations:}
    \begin{itemize}
        \item Managing entanglement among multiple provers complicates the verification process.
        \item Practical implementations are highly sensitive to noise and decoherence.
    \end{itemize}
\end{itemize}

%%%%%%%%%%%%%%%%%%%%%%%%%%%%%%%%%%%%%%%%%%%%%%%%%%%%%%%%%%%%%%
\subsection*{Summary of Interactive Quantum Automata Models}
The interactive quantum automata models discussed above demonstrate the following trade-offs:
\begin{itemize}
    \item \textbf{\gls{qip} Systems:}  
    \begin{itemize}
        \item Employ a single prover with a polynomial-time quantum verifier.
        \item Are capable of recognizing languages with bounded error, leveraging quantum communication.
    \end{itemize}
    \item \textbf{\gls{qmip} Systems:}  
    \begin{itemize}
        \item Extend QIP by using multiple non-communicating provers.
        \item Offer enhanced verification power, capturing complexity classes such as MIP\(^*\), at the cost of increased complexity in managing entanglement and communication.
    \end{itemize}
\end{itemize}

%%%%%%%%%%%%%%%%%%%%%%%%%%%%%%%%%%%%%%%%%%%%%%%%%%%%%%%%%%%%%%
\subsection*{Graphical Comparison of Interactive Quantum Automata Models}
\begin{figure}[ht]
\centering
\begin{tikzpicture}[node distance=2.5cm, auto, every node/.style={draw, rectangle, align=center, minimum width=3.0cm, minimum height=1cm, font=\small}]
    \node (QIP) {QIP\\(Single-Prover)};
    \node (QMIP) [right=of QIP] {QMIP\\(Multi-Prover)};
    \draw[->, thick] (QIP) -- (QMIP) node[midway, above] {Extension};
\end{tikzpicture}
\caption{Graphical comparison of interactive quantum automata models. QMIP systems extend QIP systems by incorporating multiple non-communicating provers, thereby enhancing verification power.}
\label{fig:interactive-hierarchy}
\end{figure}
