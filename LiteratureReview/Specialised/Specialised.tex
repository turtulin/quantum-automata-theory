\section{Specialised Quantum Automata Models}
\label{sec:specialised-qfa}

This section examines quantum automata models designed for specific computational paradigms and specialised language recognition tasks.

%%%%%%%%%%%%%%%%%%%%%%%%%%%%%%%%%%%%%%%%%%%%%%%%%%%%%%%%%%%%%%
\subsection{\glsentrylong{qqa}}
\label{subsec:qqa}

\begin{definition}[\gls{qqa}]
A quantum queue automaton is defined as:
\[
M = (Q, \Sigma, \Gamma, \delta, q_0, F, \rho_{init})
\]
\textbf{where:}
\begin{itemize}
    \item $\Gamma = \{\ket{\gamma_i}\}$: Quantum queue with basis states
    \item $\rho_{init}$: Initial mixed state of the queue
    \item $\delta: Q \times \Sigma \times \Gamma \rightarrow Q \times \Gamma \times \{Enq, Deq\}$
\end{itemize}
\end{definition}

\paragraph{Operation:}
\begin{enumerate}
    \item \textbf{Enqueue:} Applies unitary $U_{enq}(\ket{q}\otimes\ket{\gamma}) = \ket{q'}\otimes\ket{\gamma\sigma}$
    \item \textbf{Dequeue:} Measures queue head via POVM $\{E_{deq}^\sigma\}$
    \item \textbf{Real-Time Processing:} Operates in linear time $O(n)$ \cite{yakaryilmaz2010succinctness}
\end{enumerate}

\paragraph{Language Acceptance:}
Recognises \textbf{real-time context-sensitive languages} including:
\[
L = \{a^nb^nc^n | n \geq 0\} \quad \text{with bounded error } \epsilon < 1/3
\]

\paragraph{Advantages:}
\begin{itemize}
    \item Quantum superposition enables parallel queue operations
    \item Exponential state compression vs classical queue automata
\end{itemize}

\paragraph{Limitations:}
\begin{itemize}
    \item Requires quantum memory for queue storage
    \item Entanglement maintenance between head and queue
\end{itemize}

%%%%%%%%%%%%%%%%%%%%%%%%%%%%%%%%%%%%%%%%%%%%%%%%%%%%%%%%%%%%%%
\subsection{\glsentrylong{qpa}}
\label{subsec:qpa}

\begin{definition}[\gls{qpa}]
A quantum pushdown automaton extends QFA with stack:
\[
M = (Q, \Sigma, \Gamma, \delta, q_0, \bot, F)
\]
\textbf{where:}
\begin{itemize}
    \item $\Gamma$: Stack alphabet with bottom marker $\bot$
    \item $\delta: Q \times \Sigma \times \Gamma \rightarrow Q \times \Gamma^* \times \mathcal{U}(\mathcal{H})$
\end{itemize}
\end{definition}

\paragraph{Operation:}
\begin{enumerate}
    \item \textbf{Stack Operations:} Simultaneous push/pop via entanglement:
    \[
    \ket{q}\ket{\gamma} \xrightarrow{U_\sigma} \sum \alpha_i \ket{q_i}\ket{\gamma_i...\gamma_k}
    \]
    \item \textbf{Non-Determinism:} Quantum branching through stack superposition
\end{enumerate}

\paragraph{Language Acceptance:}
Recognises \textbf{non-context-free languages} including:
\[
L = \{a^nb^{2^n} | n \geq 0\} \quad \text{with probability } \geq 3/4. 
\]
\cite{bertoni2001quantum}

\paragraph{Advantages:}
\begin{itemize}
    \item Quantum stack enables efficient recognition of some context-sensitive languages \cite{ambainis2002quantum}
    \item Overcomes pumping lemma restrictions through interference
\end{itemize}

\paragraph{Limitations:}
\begin{itemize}
    \item Undecidable emptiness problem
    \item Requires error-correction for stack decoherence
\end{itemize}

%%%%%%%%%%%%%%%%%%%%%%%%%%%%%%%%%%%%%%%%%%%%%%%%%%%%%%%%%%%%%%
\subsection{\glsentrylong{psqfa}}
\label{subsec:psqfa}

\begin{definition}[\gls{psqfa}]
A postselection QFA \cite{aaronson2005complexity} is defined as:
\[
M = (Q, \Sigma, \delta, q_0, Q_{post}, \Pi_{acc})
\]
\textbf{where:}
\begin{itemize}
    \item $Q_{post} \subset Q$: Postselection subspace
    \item $\Pi_{acc}$: Projector onto accepting states
\end{itemize}
\end{definition}

\paragraph{Operation:}
\begin{enumerate}
    \item \textbf{Postselection:} Conditions probability space on:
    \[
    \text{Pr}_{post}[w] = \frac{\|\Pi_{acc}\ket{\psi_w}\|^2}{\|\Pi_{post}\ket{\psi_w}\|^2}
    \]
    \item \textbf{Error Amplification:} Converts weak measurements to definitive outcomes \cite{yamakami2014constant}
\end{enumerate}

\paragraph{Language Acceptance:}
Solves \textbf{promise problems} with:
\[
\text{Pr}_{post}[w \in L] \geq 1 - \epsilon \quad \text{and} \quad \text{Pr}_{post}[w \notin L] \leq \epsilon
\]
for $\epsilon < 1/2^n$

\paragraph{Advantages:}
\begin{itemize}
    \item Tolerates exponentially small acceptance probabilities
    \item Enforces exact recognition through ratio tests
\end{itemize}

\paragraph{Limitations:}
\begin{itemize}
    \item Requires postselection oracle
    \item Limited to problems with non-zero acceptance gaps
\end{itemize}

%%%%%%%%%%%%%%%%%%%%%%%%%%%%%%%%%%%%%%%%%%%%%%%%%%%%%%%%%%%%%%
\subsection{\glsentrylong{qtm}}
\label{subsec:qtm}

\begin{definition}[\gls{qtm}]
A quantum Turing machine is defined as:
\[
M = (Q, \Sigma, \Gamma, \delta, q_0, q_{acc}, q_{rej})
\]
\textbf{where:}
\begin{itemize}
    \item $\Gamma$: Tape alphabet including blank symbol
    \item $\delta: Q \times \Gamma \rightarrow \mathbb{C}^{Q \times \Gamma \times \{L,R\}}$
\end{itemize}
\end{definition}

\paragraph{Operation:}
\begin{enumerate}
    \item \textbf{Quantum Transitions:} Evolves as:
    \[
    \ket{q,t} \xrightarrow{U} \sum c_i \ket{q_i,\sigma_i,d_i}
    \]
    \item \textbf{Observable Tape:} Measurement collapses superposition to classical configuration
\end{enumerate}

\paragraph{Language Acceptance:}
Recognises \textbf{all BQP languages} \cite{shor1994algorithms} with:
\[
w \in L \Rightarrow \text{Pr}[M \text{ accepts } w] \geq 2/3
\]
\[
w \notin L \Rightarrow \text{Pr}[M \text{ accepts } w] \leq 1/3
\]
as per complexity-theoretic bounds \cite{nielsen2010quantum}

\paragraph{Advantages:}
\begin{itemize}
    \item Universal model for quantum computation
    \item Solves factoring/discrete log in BQP
\end{itemize}

\paragraph{Limitations:}
\begin{itemize}
    \item Requires exponential resources for exact simulation
    \item Decoherence limits practical operation time
\end{itemize}

%%%%%%%%%%%%%%%%%%%%%%%%%%%%%%%%%%%%%%%%%%%%%%%%%%%%%%%%%%%%%%
\subsection{\glsentrylong{eqfa}}
\label{subsec:eqfa}

\begin{definition}[\gls{eqfa}]
An enhanced QFA with superoperators is defined as:
\[
\mathcal{E} = (\Sigma, Q, \{\mathcal{S}_\sigma\}, \rho_0, F)
\]
\textbf{where:}
\begin{itemize}
    \item $\mathcal{S}_\sigma$: CPTP maps for each $\sigma \in \Sigma$
    \item $\rho_0$: Initial density matrix
\end{itemize}
\end{definition}

\paragraph{Operation:}
\begin{enumerate}
    \item \textbf{Noisy Evolution:} Applies quantum channels:
    \[
    \rho_{i+1} = \mathcal{S}_{\sigma_i}(\rho_i)
    \]
    \item \textbf{General Measurement:} Uses POVM $\{E_a, E_r\}$ for decision
\end{enumerate}

\paragraph{Language Acceptance:}
Recognises \textbf{stochastic languages} with:
\[
L = \{w | \text{tr}(E_a \rho_w) > \lambda\}
\]
for threshold $\lambda \in [0,1]$ \cite{hirvensalo2012quantum}

\paragraph{Advantages:}
\begin{itemize}
    \item Models realistic noise and decoherence
    \item Subsumes classical PFA recognition capabilities
\end{itemize}

\paragraph{Limitations:}
\begin{itemize}
    \item Undecidable equivalence problem
    \item Quadratic slowdown vs unitary models
\end{itemize}

%%%%%%%%%%%%%%%%%%%%%%%%%%%%%%%%%%%%%%%%%%%%%%%%%%%%%%%%%%%%%%
\subsection*{Comparative Analysis and Hierarchy}
\begin{figure}[h]
\centering
\begin{tikzpicture}[node distance=2cm]
\node (psqfa) [draw, align=center] {\gls{psqfa}\\ Promise Problems};
\node (qqa) [draw, right=of psqfa, align=center] {\gls{qqa}\\ Real-Time CSL};
\node (qpa) [draw, right=of qqa, align=center] {\gls{qpa}\\ EXPTIME};
\node (eqfa) [draw, below=1cm of qqa, align=center] {\gls{eqfa}\\ Stochastic Langs};
\node (qtm) [draw, right=of qpa, align=center] {\gls{qtm}\\ BQP};

\draw[->] (psqfa) edge (qqa)
          (qqa) edge (qpa)
          (qpa) edge (qtm)
          (qqa) edge (eqfa);
\end{tikzpicture}
\caption{Hierarchy of specialised quantum automata. Arrows indicate increasing computational power.}
\label{fig:specialised-hierarchy}
\end{figure}

\textbf{Key Comparisons:}
\begin{itemize}
    \item \textbf{Expressiveness:} PSQFA < QQA < QPA < QTM with EQFA handling distinct stochastic class
    \item \textbf{Complexity:} QTM (BQP-complete), QPA (EXPTIME), QQA (PSPACE)
    \item \textbf{Noise Tolerance:} EQFA most robust, QTM most sensitive
\end{itemize}