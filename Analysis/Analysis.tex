\chapter{Comparative Analysis}
\label{chap:comparative-analysis}

This chapter systematically compares quantum finite automata (QFAs) across categories—one-way, hybrid, enhanced, two-way, and interactive—against classical automata and each other. We evaluate expressive power, resource trade-offs, and practical viability.

%%%%%%%%%%%%%%%%%%%%%%%%%%%%%%%%%%%%%%%%%%%%%%%%%%%%%%%%%%%%%%
\section{One-Way QFAs vs. Classical Automata}
\label{sec:one-way-comparison}
\begin{itemize}
    \item \textbf{\gls{mo-1qfa}}: 
    \begin{itemize}
        \item \textbf{Strengths}: Recognises reversible regular languages (e.g., \( L_{\text{mod}} = \{a^{kp}\} \)) with zero error. Requires fewer states than \gls{dfa} for periodic languages \cite{ambainis2009superiority}.
        \item \textbf{Weaknesses}: Cannot recognise non-reversible languages like \( a\Sigma^* \).
        \item \textbf{vs. DFA}: Exponential state advantage for periodic languages but strictly weaker in expressive power.
    \end{itemize}
    
    \item \textbf{\gls{mm-1qfa}}:
    \begin{itemize}
        \item \textbf{Strengths}: Recognises all regular languages with bounded error. Exponential state reduction for \( L_{\text{mod}} \) compared to \gls{pfa} \cite{kondacs1997power}.
        \item \textbf{Weaknesses}: Fails on non-regular languages (e.g., \( L_{\text{eq}} = \{a^n b^n\} \)).
        \item \textbf{vs. PFA}: Matches regular language recognition with bounded error but lacks stochastic language capabilities.
    \end{itemize}
    
    \item \textbf{\gls{lqfa}}:
    \begin{itemize}
        \item \textbf{Strengths}: Lower measurement overhead than MM-1QFA for symmetric languages.
        \item \textbf{Weaknesses}: Fragile to premature measurement collapse.
        \item \textbf{vs. NFA}: Less expressive due to measurement constraints.
    \end{itemize}
\end{itemize}

\begin{table}[ht]
\centering
\resizebox{\textwidth}{!}{%
\begin{tabular}{|l|c|c|c|c|}
\hline
\textbf{Model} & \textbf{Language Class} & \textbf{State Complexity} & \textbf{Error Type} & \textbf{Closure Properties} \\
\hline
MO-1QFA & Reversible Regular & \( O(\log n) \) & Zero error & Reversible operations only \\
MM-1QFA & All Regular & \( O(\log n) \) & Bounded error & Closed under reversal \\
LQFA & Subset of Regular & \( O(1) \) & Probabilistic & Non-closed under union \\
Classical DFA & Regular & \( O(n) \) & Deterministic & Full closure \\
\hline
\end{tabular}%
}
\caption{Comparison of one-way QFAs vs. classical DFA. QFAs offer state efficiency but weaker closure.}
\label{tab:one-way-vs-dfa}
\end{table}

%%%%%%%%%%%%%%%%%%%%%%%%%%%%%%%%%%%%%%%%%%%%%%%%%%%%%%%%%%%%%%
\section{Hybrid Models: Bridging Classical and Quantum}
\label{sec:hybrid-comparison}

\begin{itemize}
    \item \textbf{\gls{1qfac}}:
    \begin{itemize}
        \item \textbf{Strengths}: Recognises all regular languages with \( O(1) \) quantum states. Exponentially more succinct than \gls{dfa} \cite{zheng2012one}.
        \item \textbf{Weaknesses}: Undecidable equivalence problem.
        \item \textbf{vs. DFA}: Quantum advantage in state efficiency; classical control mimics DFA transitions.
    \end{itemize}
    
    \item \textbf{\gls{cl-1qfa}}:
    \begin{itemize}
        \item \textbf{Strengths}: Closed under Boolean operations via control language \( \mathcal{L} \).
        \item \textbf{Weaknesses}: Requires precomputed \( \mathcal{L} \), increasing design complexity.
        \item \textbf{vs. NFA}: Simulates nondeterminism via \( \mathcal{L} \)-guided measurements.
    \end{itemize}
\end{itemize}

\begin{figure}[ht]
\centering
\begin{tikzpicture}[node distance=2cm, every node/.style={draw, rectangle, align=center, minimum width=3cm}]
    \node (DFA) {DFA};
    \node (1QFAC) [right=of DFA] {1QFAC};
    \node (CL1QFA) [below=of 1QFAC] {CL-1QFA};
    \draw[->] (DFA) -- (1QFAC) node[midway, above] {+ Quantum States};
    \draw[->] (1QFAC) -- (CL1QFA) node[midway, right] {+ Control Language};
\end{tikzpicture}
\caption{Evolution from classical DFA to hybrid QFAs. 1QFAC adds quantum states; CL-1QFA adds control languages.}
\label{fig:hybrid-evolution}
\end{figure}

%%%%%%%%%%%%%%%%%%%%%%%%%%%%%%%%%%%%%%%%%%%%%%%%%%%%%%%%%%%%%%
\section{Enhanced Models: Beyond Unitarity}
\label{sec:enhanced-comparison}
\begin{itemize}
    \item \textbf{\gls{eqfa}}:
    \begin{itemize}
        \item \textbf{Strengths}: Recognises non-regular languages (e.g., \( \{a^n b^n c^n\} \)) under unbounded error \cite{paschen2000quantum}.
        \item \textbf{Weaknesses}: Undecidable equivalence; requires ancilla qubits.
        \item \textbf{vs. PFA}: Subsumes PFA capabilities but with higher error rates.
    \end{itemize}
    
    \item \textbf{\gls{otqfa}}:
    \begin{itemize}
        \item \textbf{Strengths}: Models decoherence, recognizing regular languages with noise resilience.
        \item \textbf{Weaknesses}: Limited to isolated cut-point languages.
        \item \textbf{vs. 2PFA}: Similar expressive power but with quantum state efficiency.
    \end{itemize}
    
    \item \textbf{\gls{a-qfa}}:
    \begin{itemize}
        \item \textbf{Strengths}: Exponentially fewer states than \gls{nfa} for context-free languages like \( \{a^n b^n\} \).
        \item \textbf{Weaknesses}: Ancilla management complicates implementation.
        \item \textbf{vs. PDA}: Quantum parallelism replaces stack mechanics for specific languages.
    \end{itemize}
\end{itemize}

\begin{table}[ht]
\centering
\begin{tabular}{|l|c|c|c|}
\hline
\textbf{Model} & \textbf{Non-Regular Languages} & \textbf{Decoherence Handling} & \textbf{Ancilla Overhead} \\
\hline
EQFA & Yes (unbounded error) & Poor & High \\
OTQFA & No & Excellent & None \\
A-QFA & Yes (bounded error) & Moderate & Moderate \\
Classical 2PFA & Yes (e.g., \( L_{\text{eq}} \)) & N/A & N/A \\
\hline
\end{tabular}
\caption{Enhanced QFAs vs. classical probabilistic automata. EQFA trades ancilla overhead for non-regularity.}
\label{tab:enhanced-vs-classical}
\end{table}

%%%%%%%%%%%%%%%%%%%%%%%%%%%%%%%%%%%%%%%%%%%%%%%%%%%%%%%%%%%%%%
\section{Two-Way and Multi-Tape QFAs}
\label{sec:two-way-comparison}

\begin{itemize}
    \item \textbf{\gls{2qfa}}:
    \begin{itemize}
        \item \textbf{Strengths}: Recognises \( L_{\text{eq}} \) in linear time with bounded error \cite{yakaryilmaz2010succinctness}.
        \item \textbf{Weaknesses}: Quantum register scales with input length.
        \item \textbf{vs. 2DFA}: Exponential state advantage but impractical for large inputs.
    \end{itemize}
    
    \item \textbf{\gls{2qcfa}}:
    \begin{itemize}
        \item \textbf{Strengths}: Constant quantum states for \( L_{\text{pal}} \) \cite{ambainis2002quantum}.
        \item \textbf{Weaknesses}: Classical-quantum synchronization overhead.
        \item \textbf{vs. 2NFA}: Quantum interference replaces nondeterministic branching.
    \end{itemize}
    
    \item \textbf{\gls{ktqcfa}}:
    \begin{itemize}
        \item \textbf{Strengths}: Recognises \( k \)-tape dependencies (e.g., \( \{a^n b^n c^n\} \)) with \( O(1) \) quantum states \cite{zheng2012two}.
        \item \textbf{Weaknesses}: Head synchronization complexity \( \propto k \).
        \item \textbf{vs. Multi-Tape DFA}: Quantum parallelism reduces state complexity exponentially.
    \end{itemize}
\end{itemize}

\begin{figure}[ht]
\centering
\begin{tikzpicture}[node distance=2.5cm, every node/.style={draw, rectangle, align=center, minimum width=3cm}]
    \node (2DFA) {2DFA};
    \node (2QFA) [right=of 2DFA] {2QFA};
    \node (2QCFA) [below=of 2QFA] {2QCFA};
    \node (kTQCFA) [below=of 2QCFA] {kTQCFA};
    \draw[->] (2DFA) -- (2QFA) node[midway, above] {+ Quantum Registers};
    \draw[->] (2QFA) -- (2QCFA) node[midway, right] {+ Classical Control};
    \draw[->] (2QCFA) -- (kTQCFA) node[midway, right] {+ Multiple Tapes"};
\end{tikzpicture}
\caption{Progression from classical two-way to quantum multi-tape models. Arrows indicate added features.}
\label{fig:two-way-progression}
\end{figure}

%%%%%%%%%%%%%%%%%%%%%%%%%%%%%%%%%%%%%%%%%%%%%%%%%%%%%%%%%%%%%%
\section{Interactive Models: Beyond Standard Computation}
\label{sec:interactive-comparison}

\begin{itemize}
    \item \textbf{\gls{qip}}:
    \begin{itemize}
        \item \textbf{Strengths}: Solves PSPACE-complete problems with quantum verifier-prover interaction \cite{zheng2015power}.
        \item \textbf{Weaknesses}: Requires fault-tolerant quantum channels.
        \item \textbf{vs. IP}: Exponential speedup for specific problems (e.g., group non-membership).
    \end{itemize}
    
    \item \textbf{\gls{qmip}}:
    \begin{itemize}
        \item \textbf{Strengths}: Recognises undecidable languages (e.g., \( \text{MIP}^* = \text{RE} \)) via multi-prover entanglement \cite{yamakami2014constant}.
        \item \textbf{Weaknesses}: Experimentally infeasible due to decoherence.
        \item \textbf{vs. MIP}: Unconditional security via quantum entanglement.
    \end{itemize}
\end{itemize}

\begin{table}[ht]
\centering
\begin{tabular}{|l|c|c|c|}
\hline
\textbf{Model} & \textbf{Complexity Class} & \textbf{Prover Type} & \textbf{Practicality} \\
\hline
QIP & PSPACE & Single (Quantum) & Moderate (needs error correction) \\
QMIP & RE & Multiple (Entangled) & Low (decoherence-sensitive) \\
Classical IP & PSPACE & Single (Classical) & High \\
\hline
\end{tabular}
\caption{Interactive QFAs vs. classical interactive proofs. QMIP’s expressiveness comes at experimental cost.}
\label{tab:interactive-vs-classical}
\end{table}

%%%%%%%%%%%%%%%%%%%%%%%%%%%%%%%%%%%%%%%%%%%%%%%%%%%%%%%%%%%%%%
\section{Overall Hierarchy and Recommendations}
\label{sec:hierarchy-summary}

\begin{figure}[ht]
\centering
\resizebox{\textwidth}{!}{%
\begin{tikzpicture}[node distance=1.5cm, every node/.style={draw, rectangle, align=center, minimum width=2.5cm, font=\footnotesize}]
    % One-way
    \node (MO1) {MO-1QFA};
    \node (MM1) [right=of MO1] {MM-1QFA};
    \node (LQFA) [right=of MM1] {LQFA};
    
    % Hybrid
    \node (1QFAC) [below=of MO1] {1QFAC};
    \node (CL1) [below=of MM1] {CL-1QFA};
    
    % Enhanced
    \node (EQFA) [below=of LQFA] {EQFA};
    \node (OTQFA) [right=of EQFA] {OTQFA};
    \node (AQFA) [right=of OTQFA] {A-QFA};
    
    % Two-way
    \node (2QFA) [above=of MO1] {2QFA};
    \node (2QCFA) [above=of MM1] {2QCFA};
    \node (kTQCFA) [above=of LQFA] {kTQCFA};
    
    % Interactive
    \node (QIP) [below=of EQFA, xshift=-2cm] {QIP};
    \node (QMIP) [right=of QIP] {QMIP};
    
    % Arrows
    \draw[->] (MO1) -- (MM1);
    \draw[->] (MM1) -- (LQFA);
    \draw[->] (MO1) -- (1QFAC);
    \draw[->] (MM1) -- (CL1);
    \draw[->] (LQFA) -- (EQFA);
    \draw[->] (1QFAC) -- (EQFA);
    \draw[->] (CL1) -- (OTQFA);
    \draw[->] (EQFA) -- (AQFA);
    \draw[->] (2QFA) -- (2QCFA);
    \draw[->] (2QCFA) -- (kTQCFA);
    \draw[->] (AQFA) -- (QIP);
    \draw[->] (QIP) -- (QMIP);
\end{tikzpicture}%
}
\caption{Comprehensive hierarchy of QFA models. Vertical arrows indicate increased expressiveness; horizontal arrows denote specialization. Interactive models (bottom) form a separate branch.}
\label{fig:full-hierarchy}
\end{figure}

\subsection*{Best and Worst Models by Feature}
\begin{itemize}
    \item \textbf{State Efficiency}: 
    \begin{itemize}
        \item \textbf{Best}: 1QFAC (constant quantum states for regular languages).  
        \item \textbf{Worst}: 2QFA (linear scaling quantum registers).
    \end{itemize}
    
    \item \textbf{Expressiveness}: 
    \begin{itemize}
        \item \textbf{Best}: QMIP (recognises recursively enumerable languages).  
        \item \textbf{Worst}: MO-1QFA (limited to reversible regular languages).
    \end{itemize}
    
    \item \textbf{Practicality}: 
    \begin{itemize}
        \item \textbf{Best}: MM-1QFA (bounded error, simple implementation).  
        \item \textbf{Worst}: kTQCFA (high synchronization overhead).
    \end{itemize}
\end{itemize}

%%%%%%%%%%%%%%%%%%%%%%%%%%%%%%%%%%%%%%%%%%%%%%%%%%%%%%%%%%%%%%
\section*{Conclusion}
The QFA landscape reveals a trade-off between expressiveness and practicality. While models like QMIP and 2QFA push quantum advantages to theoretical limits, hybrid models like 1QFAC and 2QCFA offer near-term viability. Future work should prioritise error correction for enhanced models and experimental validation of interactive protocols.