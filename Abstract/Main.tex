\begin{abstract} {
    As quantum computing matures, concise models are needed to connect its theoretical foundations with practical implementations.  
    Quantum finite automata fulfil this role by extending the well-known framework of classical finite automata into the quantum domain, offering a compact setting in which to study finite-memory quantum behaviour.  
    
    This thesis first revisits the essential background on both quantum information and classical automata, establishing a common language for readers from either field.  
    It then delivers a comprehensive literature review that consolidates the many quantum finite automata variants introduced over the last three decades and arranges them in a unified, consistently named taxonomy.  
    Building on that organisation, the work presents a compilation algorithm that translates the widely studied measure-once and measure-many quantum finite automata models into architecture-independent quantum-circuit templates, thereby bridging abstract automaton descriptions with executable gate-level designs.
    
    Viewed more broadly, this study contributes a computer-science perspective on the quantum landscape and offers an accessible entry point for further research in quantum software by promoting circuit-level abstraction and enabling systematic comparison of different designs.
}

\noindent\textbf{Keywords: Quantum finite automata, automata theory, quantum information, quantum computing, quantum circuits, quantum compilation.}
\end{abstract}
\newpage
