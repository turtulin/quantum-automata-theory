\begin{abstract} {
    As quantum computing matures, concise models are needed to bridge theoretical foundations and practical implementations.   
    \glspl{qfa} fulfil this role by extending the standard framework of classical finite automata into the quantum domain, providing a compact setting in which to study finite memory quantum behaviour.  
    
    This thesis first revisits the essential notions on both classical automata and quantum information, establishing a common language for readers from either field.  
    It then presents a comprehensive literature review that consolidates the many \glspl{qfa} variants introduced over the last three decades and arranges them in a unified, consistently named taxonomy.  
    Building on this organisation, the work proposes a compilation algorithm that translates the widely studied \glspl{mo-1qfa} and \glspl{mm-1qfa} models into architecture independent quantum circuit templates, thereby linking abstract automaton descriptions with executable gate level designs.
    
    Viewed more broadly, the study contributes a computer science perspective on the quantum landscape and offers an accessible entry point for further research in quantum software by encouraging circuit-level abstraction and enabling systematic comparison of different designs.
}


\noindent\textbf{Keywords: Quantum finite automata, automata theory, quantum information, quantum computing, quantum circuits, quantum compilation.}
\end{abstract}
\newpage
