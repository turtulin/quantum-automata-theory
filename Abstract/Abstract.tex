\begin{abstract} {
    Quantum automata theory investigates how principles of quantum mechanics can be integrated with classical models of computation to reveal new limits and possibilities in computational power. Although the theory offers deep insights, progress has been slowed by inconsistent notation, unclear model definitions, and fragmented comparisons across different approaches. This thesis establishes a unified framework that standardises definitions and systematically compares classical finite automata with their quantum counterparts.

    The work begins with a comprehensive review of classical finite automata, including deterministic, nondeterministic, probabilistic, and two-way models, to lay the necessary theoretical foundation. 
    It then introduces the foundational principles of quantum mechanics, such as superposition, entanglement, and measurement, and explains how these principles can be used to define quantum finite automata.
    The thesis then reviews various models of quantum finite automata, analyzing their formal definitions, computational dynamics, and language recognition properties. Through a detailed literature review and comparative analysis, this document clarifies longstanding ambiguities and identifies open research challenges, such as issues in equivalence checking and complexity trade-offs.
    
    The unified framework presented here offers clear insights into the computational capabilities and limitations of quantum automata, and it provides a systematic basis for further research in quantum computational models.
    }
\noindent\textbf{Keywords:} Quantum automata, finite automata taxonomy, computational complexity, quantum-classical hybrids, formal language theory 
\end{abstract}
\newpage
