\label{chap:chapter1}
\begin{abstract} {
Quantum automata theory merges classical computational models with quantum mechanics to explore the capabilities and limitations of quantum systems. Despite its theoretical potential, the field remains fragmented by inconsistent notation, ambiguous model definitions, and a lack of systematic comparisons between classical and quantum finite automata. This thesis addresses these gaps by establishing a unified framework for analyzing classical and quantum finite automata, emphasizing standardized definitions, structural parallels, and rigorous computational property evaluations.   

We first formalize classical finite automata—including deterministic, non-deterministic, probabilistic, and two-way variants—to establish foundational concepts. Building on this, we systematically catalog quantum finite automata models, categorizing one-way hybrids such as the one-way quantum finite automaton with classical states, two-way models such as the two-way quantum finite automaton, and enhanced variants including the enhanced quantum finite automaton and quantum finite automaton with open time evolution. For each model, we analyze formal definitions, computational dynamics, language acceptance criteria, expressive power, and closure properties, while contextualizing relationships through comparative studies.   

By synthesizing results from foundational and contemporary literature, this work resolves ambiguities in prior formulations, such as quantum-classical state hybrids, and identifies open problems in equivalence checking, pumping lemma extensions, and quantum advantage thresholds. Key contributions include a hierarchical taxonomy of models, classifications of decidable vs.~undecidable problems, and insights into size-space efficiency trade-offs. This thesis aims to serve as a foundational reference for researchers and provides a methodology to guide the development of new quantum automata models.   
}
\noindent\textbf{Keywords:} Quantum automata, finite automata taxonomy, computational complexity, quantum-classical hybrids, formal language theory 
\end{abstract}
\newpage