\begin{abstract} {
    Quantum automata theory investigates how principles of quantum mechanics can be integrated with classical models of computation to reveal new limits and possibilities in computational power. Although the theory offers deep insights, progress has been slowed by inconsistent notation, unclear model definitions, and fragmented comparisons across different approaches. This thesis establishes a unified framework that standardizes definitions and systematically compares classical finite automata with their quantum counterparts, focusing on state complexity and language recognition capabilities.

    The work begins with a comprehensive review of classical finite automata, including deterministic, nondeterministic, probabilistic, and two-way models, also specifying the formal language theory that underpins these models. This review lays the necessary theoretical foundation for the subsequent discussion of quantum finite automata.
    The thesis then introduces the foundational principles of quantum mechanics and explains how these underpin the definition of quantum automata. 
    Subsequently, the work reviews various quantum automata models, analyzing their formal definitions, computational dynamics and language recognition properties. 
    Lastly, the thesis presents a unified taxonomy that provides a systematic comparison between classical and quantum automata models, highlighting the computational capabilities and limitations of quantum automata.
    The unified framework presented here offers clear insights into the computational capabilities and limitations of quantum automata, and it provides a systematic basis for further research in quantum computational models.
}
\noindent\textbf{Keywords:} Quantum automata, finite automata taxonomy, computational complexity, hybrid quantum-classical models, formal language theory 
\end{abstract}
\newpage
