\chapter{Introduction}
\label{chap:introduction}

The accelerating progress of quantum computing has sparked a parallel evolution in theoretical models that aim to describe and harness quantum behaviour within computational frameworks \cite{deutsch1985quantum}. Among these models, the study of \glspl{qfa} offers a minimal yet expressive lens for investigating finite memory quantum systems \cite{ambainis19981,moore2000quantum}. Owing to their bounded state registers, \glspl{qfa} constitute a rigorous setting for analysing language recognition capabilities under quantum constraints and for crafting algorithms that operate within the limits of current \gls{nisq} technology \cite{Preskill2018nisq,lussi2024implementing}.

\Glspl{qfa} are particularly attractive due to their finite memory constraints, making them ideal for investigating foundational questions in quantum computational theory and exploring efficient recognisers for regular and near regular languages. Moreover, their simplicity renders them amenable to physical implementation on today's hardware, where full scale quantum algorithms remain impractical \cite{Arute2019supremacy}. Despite this promise, the diversity of \gls{qfa} models has led to a fragmented landscape. Disparate notations, inconsistent terminologies, and varied acceptance criteria have made it difficult to compare models, reason about their capabilities, or implement them as executable artefacts.

This thesis tackles the fragmentation challenge through two tightly coupled contributions. First, it provides a coherent and systematic taxonomy of \glspl{qfa}. Drawing on over three decades of research, the thesis consolidates the principal families of \glspl{qfa} into a unified nomenclature, identifies key relationships between models, and supplies the conceptual scaffolding required for rigorous analysis and comparison of expressive power, closure properties, and language recognition capabilities.

The second contribution closes the gap between abstract definitions and executable artefacts by introducing a compilation framework that converts high level descriptions of \glspl{mo-1qfa} and \glspl{mm-1qfa} into quantum circuits. The compiler leverages the taxonomy to normalise automaton specifications and then synthesises architecture independent gate templates whose parameters instantiate the original transition operators. In this way, the compilation algorithm operationalises the taxonomic unification: once models are described within a common schema, they can be mapped uniformly to circuits, enabling empirical evaluation, formal verification, and direct deployment on hardware

The remainder of the thesis is structured as follows. Chapter~\ref{chap:background} reviews the necessary background in classical automata theory and quantum information science. Chapter~\ref{chap:quantum-finite-automata} presents the unified taxonomy of \glspl{qfa}, establishing precise definitions and cataloguing formal properties. Chapter~\ref{chap:automata-to-circuits} details the circuit compilation framework, with examples illustrating how abstract automata are translated into gate level designs. Finally, Chapter~\ref{chap:conclusion} summarises the findings and outlines prospects for extending the framework to more powerful automata and for integrating \glspl{qfa} into broader quantum software stacks.


