\chapter{Introduction}
\label{chap:introduction}

The accelerating development of quantum computing has sparked a parallel evolution in theoretical models that aim to describe and harness quantum behaviour within computational frameworks. As researchers strive to reconcile physical limitations with algorithmic expressiveness, the study of \glspl{qfa} has emerged as a promising direction. These automata extend the well-established framework of classical finite automata into the quantum domain, yielding models that are minimalistic in structure yet rich in quantum phenomena.

\Glspl{qfa} are particularly attractive due to their finite memory constraints, making them ideal for investigating foundational questions in quantum computational theory and exploring efficient recognisers for regular and near-regular languages. Moreover, their simplicity renders them amenable to physical implementation on \gls{nisq} devices, where full-scale quantum algorithms remain impractical. Despite this promise, the diversity of \gls{qfa} models has led to a fragmented landscape. Disparate notations, inconsistent terminologies, and varied acceptance criteria have made it difficult to compare models, reason about their capabilities, or implement them in a unified framework.

This thesis addresses these challenges through two major contributions. First, it provides a coherent and systematic taxonomy of \glspl{qfa}. Drawing on over three decades of research, the thesis consolidates the principal families of \gls{qfa} into a unified nomenclature and identifies key relationships between models. This classification is not merely pedagogical—it lays the foundation for rigorous analysis and comparison of expressive power, closure properties, and language recognition capabilities.

Second, the thesis introduces a compilation framework that transforms high-level descriptions of \glspl{mo-1qfa} and \glspl{mm-1qfa}, into executable quantum circuits. These circuits are constructed to preserve the semantics of the original automaton, enabling formal verification and direct deployment on quantum hardware. The compilation process is structured into two phases: symbolic template generation, which defines the automaton’s logical structure for inputs of fixed length, and operator instantiation, which assigns concrete unitary transformations to each input-driven transition.

The remainder of the thesis is organised as follows. Chapter~\ref{chap:background} reviews the necessary background in classical automata theory and quantum information science. Chapter~\ref{chap:quantum-finite-automata} delves into the taxonomy of \glspl{qfa}, presenting detailed definitions and formal properties of each model. Chapter~\ref{chap:automata-to-circuits} details the circuit compilation framework, with examples illustrating how abstract automata are translated into gate-level designs. The thesis concludes in Chapter~\ref{chap:conclusion} by summarising the contributions and findings, along with a discussion of potential extensions to more powerful models and reflections on the role of \glspl{qfa} in the broader context of quantum software engineering.


