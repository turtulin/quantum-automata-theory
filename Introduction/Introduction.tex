\chapter{Introduction}  

The intersection of quantum mechanics and theoretical computer science has given rise to quantum computing, a field that reimagines computational paradigms through the lens of quantum phenomena such as superposition, entanglement, and measurement. At its core lies quantum automata theory, which seeks to understand how these principles redefine the boundaries of classical computation. Classical finite automata—\gls{dfa}, \gls{nfa}, \gls{pfa}, and two-way variants—have long served as the bedrock of formal language theory, providing mathematically rigorous frameworks for analyzing computational complexity and decidability. Yet, \glspl{qfa} introduce probabilistic and non-deterministic behaviors that transcend classical limits, necessitating a coherent framework to classify and analyze their capabilities. This thesis emerges from the recognition that the current landscape of quantum automata theory is fragmented: definitions vary across papers, notations lack standardization, and comparisons between classical and quantum models remain scattered across disjointed works. By systematically unifying these elements, this thesis aims to bridge the conceptual gap between classical and quantum computational models, offering a structured lens through which their interaction can be rigorously studied \cite{ambainis2009superiority}.  

The motivation for this work is two-fold: theoretical exploration and practical application. Theoretically, quantum automata represent the simplest quantum computational models, providing a sandbox to explore the interplay between quantum mechanics and computation. They challenge classical intuitions—for instance, quantum parallelism enables certain \glspl{qfa}, such as the \gls{mm-1qfa}, to recognize languages with exponentially fewer states than their classical counterparts \cite{moore2000quantum}. Practically, as quantum hardware advances, understanding the minimal resources required to implement \glspl{qfa} becomes critical for designing efficient algorithms and error-correcting schemes. Yet, the field’s progress has been hindered by ambiguities in model definitions. For example, early quantum automata models like the \gls{mo-1qfa} and \gls{mm-1qfa} were defined with differing acceptance criteria, leading to confusion about their relative computational power \cite{kondacs1997power}. Similarly, hybrid models such as the \gls{1qfac} introduce classical memory components, complicating direct comparisons to purely quantum or classical automata \cite{zheng2012one}. These inconsistencies obscure the true capabilities of quantum models and hinder cross-disciplinary collaboration.  

Central to this thesis is the observation that no single document systematically catalogs quantum automata models alongside their classical counterparts. Existing surveys, while valuable, often focus on specific subsets of models or lack the granularity needed to resolve nuanced differences in computational power, closure properties, or decidability. For instance, the expressive power of \gls{2qfa} surpasses that of classical two-way automata, yet the conditions under which this advantage manifests—such as the role of quantum interference in recognizing non-regular languages—remain underexplored in a comparative context \cite{yakaryilmaz2010succinctness}. In contrast, this work adopts a taxonomic approach, dissecting each model’s formal definition, acceptance criteria, and operational dynamics while contextualizing its position within the broader hierarchy of automata. This approach not only clarifies existing results but also identifies gaps where further research is needed, such as the decidability of equivalence problems for \glspl{qfa} with mixed states or the precise trade-offs between quantum entanglement and space efficiency \cite{hirvensalo2012quantum}.  

The research challenges addressed in this thesis are multifaceted. First, reconciling disparate notation and definitions requires a meticulous synthesis of foundational and contemporary literature. For example, the transition from unitary operations in \gls{mo-1qfa} to superoperator-based transitions in open quantum systems as seen in \glspl{otqfa} requires a unified formalism to compare their computational behaviors \cite{bertoni2001quantum}. Second, characterizing the relationships between classical and quantum models necessitates a framework that accounts for both their similarities (e.g., the ability of \gls{1qfac} to simulate \glspl{dfa}) and their divergences (e.g., the exponential state advantage of \gls{2qfa} over two-way probabilistic automata). Third, the absence of standardized pumping lemmas or minimization algorithms for \glspl{qfa} complicates efforts to classify their language recognition capabilities, a challenge this thesis tackles through comparative analysis of closure properties and equivalence criteria \cite{ambainis1998one}.  

To address these challenges, this thesis employs a structured methodology. It begins by grounding the discussion in classical automata theory, revisiting \gls{dfa}, \gls{nfa}, \gls{pfa}, and two-way variants to establish foundational concepts. Building on this, it systematically explores quantum models—from early variants such as \gls{mo-1qfa} \cite{moore2000quantum} and \gls{mm-1qfa} \cite{kondacs1997power} to advanced hybrids such as \gls{1qfac} and \gls{eqfa}. Each model is analyzed through multiple dimensions: formal definitions are standardized, acceptance criteria are scrutinized, and computational dynamics—such as the role of measurement timing or quantum-classical state interactions—are dissected. By juxtaposing classical and quantum models across these dimensions, the thesis uncovers patterns in their expressive power, such as the ability of certain \glspl{qfa} to recognize non-regular languages with bounded error, a feat impossible for classical finite automata \cite{ambainis2009superiority}.  

A key contribution of this work is its hierarchical taxonomy of automata models, which organizes classical and quantum automata into a coherent structure based on their computational features. This taxonomy reveals, for example, that \gls{2qfa} occupy a higher complexity class than their one-way counterparts, while quantum automata with classical states (\gls{1qfac}) occupy an intermediate position, bridging purely quantum and classical models \cite{yakaryilmaz2010succinctness}. The taxonomy also highlights open problems, such as the precise relationship between \glspl{a-qfa} and \glspl{gqfa}, or the conditions under which \glspl{qfa} outperform probabilistic models in language recognition \cite{hirvensalo2012quantum}. By mapping these relationships, the thesis provides a roadmap for future research on quantum advantage thresholds and the minimal resource requirements for quantum-enhanced computation.  

The thesis is organized to guide the reader through increasingly complex layers of analysis. Following this introduction, Chapter 2 consolidates foundational concepts from classical automata theory and quantum mechanics, providing a unified background for subsequent discussions. Chapter 3 forms the core of the work, presenting a comprehensive catalog of quantum automata models. Each model is formally defined, analyzed for computational dynamics, and compared to classical and quantum alternatives. Chapter 4 synthesizes these analyses, evaluating expressive power, closure properties, and decidability between models. Finally, Chapter 5 concludes by reflecting on the thesis’s contributions and outlining directions for future research, such as solving open questions in equivalence checking for \glspl{qfa} \cite{li2012characterizations}, extending pumping lemmas to quantum models \cite{ambainis1998one}, and developing minimization algorithms for hybrid automata.  

In essence, this thesis seeks to transform quantum automata theory from a collection of isolated results into a cohesive framework. By standardizing definitions, clarifying model relationships, and identifying open challenges, it provides both a reference for researchers and a methodology for advancing the field. As quantum computing moves from theory to practice, such systematic foundations will be essential for harnessing the full potential of quantum-enhanced computation.