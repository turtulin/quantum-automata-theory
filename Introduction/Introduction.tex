\chapter{Introduction}  
\label{chap:introduction}

The intersection of quantum mechanics and theoretical computer science has given rise to quantum computing, a field that reimagines computational paradigms through the lens of quantum phenomena such as superposition, entanglement, and measurement. At its core lies quantum automata theory, which seeks to understand how these principles redefine the boundaries of classical computation. \glspl{cfa}—\gls{dfa}, \gls{nfa}, \gls{pfa}, and two-way variants—have long served as the bedrock of formal language theory, offering mathematically rigorous frameworks for analyzing computational complexity and decidability. In contrast, \glspl{qfa} exhibit probabilistic and non-deterministic behaviors that transcend classical limits, necessitating a coherent framework to classify and analyse their capabilities. This thesis emerges from the recognition that the current landscape of quantum automata theory is fragmented: definitions vary across papers, notations lack standardization, and comparisons between classical and quantum models remain scattered across disjointed works \cite{gruska2012quantum}. By systematically unifying these elements, this thesis aims to bridge the conceptual gap between classical and quantum computational models, offering a structured lens through which their interaction can be rigorously studied \cite{ambainis2009superiority}.  

The motivation for this work is two-fold: theoretical exploration and practical application. Theoretically, quantum automata represent the simplest quantum computational models, providing a sandbox to explore the interplay between quantum mechanics and computation. They challenge classical intuitions—for instance, quantum parallelism enables certain \glspl{qfa}, such as the \gls{mm-1qfa}, to recognise languages with exponentially fewer states than their classical counterparts \cite{ambainis1998one}. Practically, as quantum hardware advances, understanding the minimal resources required to implement \glspl{qfa} becomes critical for designing efficient algorithms and robust error-correcting schemes \cite{nielsen2010quantum}. Yet, progress in the field has been hindered by ambiguities in model definitions. For example, early quantum automata models like the \gls{mo-1qfa} and \gls{mm-1qfa} were defined with differing acceptance criteria, leading to confusion about their relative computational power \cite{kondacs1997power}. Similarly, hybrid models such as the \gls{1qfac} introduce classical memory components, complicating direct comparisons to purely quantum or classical automata \cite{li2012characterizations}. These inconsistencies obscure the true capabilities of quantum models and hinder cross-disciplinary collaboration.

A central observation motivating this thesis is that no single document currently catalogs quantum automata models alongside their classical counterparts. Existing surveys, while valuable, often focus on specific subsets of models or lack the granularity needed to resolve nuanced differences in computational power, closure properties, or decidability \cite{gruska2012quantum}. For instance, although the expressive power of \gls{2qfa} surpasses that of classical two-way automata, the conditions under which this advantage manifests—such as the role of quantum interference in recognizing non-regular languages—remain underexplored in a unified context \cite{yakaryilmaz2010succinctness}. Moreover, the literature review in this thesis is dedicated to clearly identifying the specific classes of languages each automaton model accepts. In contrast, this work adopts a taxonomic approach, dissecting each model’s formal definition, acceptance criteria, and operational dynamics while contextualizing its position within the broader hierarchy of automata. This approach not only clarifies existing results but also identifies gaps where further research is needed, such as the decidability of equivalence problems for \glspl{qfa} with mixed states or the precise trade-offs between quantum entanglement and space efficiency \cite{hirvensalo2012quantum}.  

The research challenges addressed in this thesis are multifaceted. First, reconciling disparate notation and definitions requires a meticulous synthesis of foundational and contemporary literature. For example, the transition from unitary operations in \gls{mo-1qfa} to superoperator-based transitions in open quantum systems, as seen in \glspl{otqfa}, calls for a unified formalism to compare their computational behaviors \cite{bertoni2001quantum, breuer2002theory}. Second, characterizing the relationships between classical and quantum models necessitates a framework that accounts for both their similarities (e.g., the ability of \gls{1qfac} to simulate \glspl{dfa}) and their divergences (e.g., the exponential state advantage of \gls{2qfa} over two-way probabilistic automata). Third, the absence of standardised pumping lemmas or minimization algorithms for \glspl{qfa} complicates efforts to classify their language recognition capabilities, a challenge that this thesis tackles through a comparative analysis of closure properties and equivalence criteria \cite{yakaryilmaz2014quantum}.  

To address these challenges, this thesis employs a structured methodology that unfolds in several stages. It begins by grounding the discussion in classical automata theory, revisiting \gls{dfa}, \gls{nfa}, \gls{pfa}, and two-way variants to establish foundational concepts. Building on this, the thesis introduces the foundational principles of quantum mechanics—such as superposition, entanglement, and measurement—which provide the basis for defining quantum finite automata \cite{nielsen2010quantum}. With this dual background in place, the work systematically explores various quantum models, ranging from early variants like the \gls{mo-1qfa} \cite{moore2000quantum} and the \gls{mm-1qfa} \cite{kondacs1997power} to advanced hybrids such as the \gls{1qfac} and enhanced models (e.g., the \gls{eqfa}). Each model is rigorously analysed along several dimensions: its formal definition is standardised, its acceptance criteria are scrutinised, and its computational dynamics—such as the role of measurement timing and the interplay between quantum and classical states—are carefully dissected, with particular emphasis on identifying the exact classes of formal languages recognised by each model.

A significant contribution of this work is the development of a hierarchical taxonomy of automata models, which organises both classical and quantum automata into a coherent structure based on their computational features and complexity classes. For example, the analysis reveals that \glspl{2qfa} occupy a higher complexity class than their one-way counterparts, while hybrid models such as the \gls{1qfac} serve as an intermediate bridge between purely quantum and purely classical models \cite{yakaryilmaz2010succinctness}. The taxonomy further highlights open research questions, such as the precise relationships between models that employ different quantum operational frameworks (e.g., \glspl{a-qfa} versus \glspl{gqfa}) and the conditions under which quantum automata outperform probabilistic models in language recognition tasks \cite{hirvensalo2012quantum}.

The thesis is organised to guide the reader through these progressively complex layers of analysis. Following this introduction, Chapter~\ref{chap:background} consolidates foundational concepts from both classical automata theory and quantum mechanics, providing a unified background for the discussions that follow. Chapter~\ref{chap:models} presents a comprehensive catalog of quantum automata models; each model is formally defined, its computational dynamics are analysed, and its language recognition capabilities are explicitly detailed. Chapter~\ref{chap:comparative-analysis} synthesises these findings by evaluating expressive power, closure properties, and decidability issues across different models. Finally, Chapter~\ref{chap:conclusion} concludes by reflecting on the thesis’s contributions and outlining directions for future research, such as solving open questions in equivalence checking for \glspl{qfa} \cite{li2012characterizations}, extending pumping lemmas to quantum models \cite{yakaryilmaz2014quantum}, and developing minimization algorithms for hybrid automata.

In essence, this thesis seeks to transform quantum automata theory from a collection of isolated results into a cohesive and systematic framework. By standardizing definitions, clarifying the relationships between different models, and identifying critical open challenges, the work provides both a valuable reference for researchers and a solid methodological basis for future advancements in quantum computational models. As quantum computing transitions from theory to practice, such systematic foundations will be essential for harnessing the full potential of quantum-enhanced computation.