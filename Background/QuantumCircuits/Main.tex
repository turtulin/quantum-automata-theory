\section{Quantum Gates and Circuits}
\label{sec:quantum-gates-and-circuits}

The gate model formulates quantum computation as a sequence of reversible transformations that act on an ordered register of qubits. Each elementary transformation, or quantum logic gate, is represented by a unitary matrix that preserves the norm of the wavefunction and therefore the probabilistic interpretation of quantum states \cite{NCFlips}. By composing gates drawn from a finite universal set such as \gls{hadamard},  \gls{phase}, and \gls{cnot}, any unitary operator on a finite dimensional Hilbert space can be approximated to arbitrary accuracy, providing the foundation on which quantum algorithms like Shor's factoring routine and the \gls{qft} are constructed \cite{Barenco1995elementary,Shor1994}.

Beyond abstract universality, physical realisations impose architectural constraints that influence gate granularity, qubit connectivity, and measurement timing. The criteria articulated by DiVincenzo identify coherent state preparation, high fidelity single and two qubit gates, and reliable read-out as indispensable requirements for scalable devices \cite{divincenzo2000criteria}. Present \gls{nisq} processors satisfy these conditions only approximately, which motivates depth-aware decompositions, noise-adaptive compilation, and the explicit accounting of ancillary resources \cite{Preskill2018nisq}. 

In the context of this thesis, quantum circuits serve as the executable target for the compilation of \glspl{qfa}. Each transition operator of an automaton is mapped to a concrete gate sequence, and every accepting projector is realised by a register measurement that interfaces classical control with coherent evolution. The remainder of this section surveys the primitive gate library, canonical circuit families, and decomposition techniques that together supply the hardware-agnostic substrate on which the automata-to-circuit translation of Chapter~\ref{chap:automata-to-circuits} is built.

\subsection{Common Quantum Gates}
\label{sec:common-quantum-gates}

Quantum logic gates are unitary operators acting on one or more qubits.  
They constitute the elementary instruction set of the gate model of quantum computation \cite{NielsenChuang2010}.  
A single-qubit gate realises a rotation of the Bloch vector, whereas a multi-qubit gate can generate entanglement, an intrinsically non-classical resource \cite{Barenco1995elementary}.  
This subsection recalls the canonical gate library, presents the corresponding matrices, and illustrates their circuit symbols \cite{Koch2022quantikz}.

\subsubsection*{Single-qubit gates: Pauli \(X\), \(Y\), \(Z\)}

The Pauli gates \(X\), \(Y\), and \(Z\) implement rotations by \(\pi\) about the \(x\), \(y\), and \(z\) axes of the Bloch sphere, respectively \cite{NielsenChuang2010}.  
In the computational basis \(\{\ket{0},\ket{1}\}\) they take the forms
\[
X=\begin{pmatrix}0&1\\[2pt]1&0\end{pmatrix},
\qquad
Y=\begin{pmatrix}0&-i\\[2pt]i&0\end{pmatrix},
\qquad
Z=\begin{pmatrix}1&0\\[2pt]0&-1\end{pmatrix}.
\]
The gate \(X\) exchanges the computational basis states and therefore plays the role of a quantum NOT \cite{NCFlips}.  
The gate \(Z\) leaves \(\ket{0}\) invariant while mapping \(\ket{1}\) to \(-\ket{1}\); it is consequently referred to as a phase flip \cite{PhaseFlip}.  
The gate \(Y\) combines a bit flip with an intrinsic phase: \(Y\ket{0} = i\ket{1}\) and \(Y\ket{1} = -\,i\ket{0}\) \cite{NCFlips}.  
All three operators square to the identity (up to a global phase) and mutually anticommute, properties that underpin stabiliser-based error-correction protocols \cite{Gottesman1997stabilizer}.

\begin{figure}[ht]
  \centering
  \begin{quantikz}
    \lstick{$\ket{q}$} & \gate{\mathsf{X}} & \qw
  \end{quantikz}
  \caption{Circuit symbol for the Pauli \(X\) gate.  Replacing the label by \(\mathsf{Y}\) or \(\mathsf{Z}\) yields the symbols for the remaining Pauli rotations.}
  \label{fig:x-gate}
\end{figure}

\subsubsection*{\glsentrylong{hadamard}}
The \gls{hadamard} gate realises a rotation by \(\tfrac{\pi}{2}\) about the axis \((x+z)/\sqrt{2}\) on the Bloch sphere \cite{Deutsch1985}.  Its unitary matrix in the computational basis is
\[
\gls{hadamard}= \frac{1}{\sqrt{2}}
\begin{pmatrix}
1 & 1\\[2pt]
1 & -1
\end{pmatrix}.
\]
Acting on the basis states it produces the equal-weight superpositions
\(\gls{hadamard}\ket{0}= \ket{+}\) and
\(\gls{hadamard}\ket{1}= \ket{-}\), where  
\(\ket{\pm}= (\ket{0}\pm\ket{1})/\sqrt{2}\) define the Hadamard basis \cite{NielsenChuang2010}.  
A second application reverses the transformation since \(\gls{hadamard}^{2}=I\) up to global phase \cite{HadamardIteration}.  
Geometrically, the operation transfers quantum states between the \(Z\) and \(X\) axes of the Bloch sphere, sending \(\ket{0}\) and \(\ket{1}\) to the equator and vice versa \cite{Gibney2019bloch}.  
Because it satisfies \(\gls{hadamard} Z \gls{hadamard}=X\) and \(\gls{hadamard} X \gls{hadamard}=Z\), the gate exchanges the roles of bit-flip and phase-flip operators and is therefore indispensable for basis changes and interference patterns in algorithms such as Deutsch-Jozsa and Grover search \cite{Deutsch1992rapid,Grover1997fast,NCFlips}.

\begin{figure}[ht]
  \centering
  \begin{quantikz}
    \lstick{$\ket{q}$} & \gate{H} & \qw
  \end{quantikz}
  \caption{Circuit symbol for the \gls{hadamard} gate \cite{Koch2022quantikz}.}
  \label{fig:h-gate}
\end{figure}

\subsubsection*{Phase rotations: \gls{phase}, \gls{tgate} and continuous \(R_{z}\)}

Phase rotations apply a relative phase to the state $\ket{1}$ while leaving $\ket{0}$ unchanged \cite{Gottesman1997stabilizer}.  
The \gls{phase} gate introduces a phase of \(\pi/2\) and the \gls{tgate} gate a phase of \(\pi/4\); their unitaries are
\[
\gls{phase}= 
\begin{pmatrix}
1 & 0\\[4pt]
0 & i
\end{pmatrix},
\qquad
\gls{tgate}= 
\begin{pmatrix}
1 & 0\\[4pt]
0 & e^{i\pi/4}
\end{pmatrix},
\]
so that
\(\gls{phase}\ket{1}=i\ket{1}\) and \(\gls{tgate}\ket{1}=e^{i\pi/4}\ket{1}\) \cite{NCFlips}.  
Because \(\gls{phase}^{2}=Z\) and \(\gls{tgate}^{4}=Z\), the operations may be viewed as fractional powers of the Pauli \(Z\) rotation \cite{Bravyi2012magic}.  

Neither gate is involutory; instead their adjoints are
\[
  \gls{phase}^{\dagger} =
  \begin{pmatrix}
    1 & 0\\[4pt]
    0 & -i
  \end{pmatrix},
  \qquad
  \gls{tgate}^{\dagger} =
  \begin{pmatrix}
    1 & 0\\[4pt]
    0 & e^{-i\pi/4}
  \end{pmatrix}.
\]

The discrete set \(\{\gls{hadamard},\gls{phase},\mathrm{CNOT}\}\) generates the Clifford group, which is not universal for quantum computation, but adjoining the non-Clifford \gls{tgate} yields a universal gate library capable of approximating any single-qubit unitary to arbitrary precision \cite{Bravyi2012magic,Dawson2005solovay}.  
In fault-tolerant architectures the cost of magic-state distillation makes the \gls{tgate} the dominant resource, so circuit depth is often quantified by its \(T\)-count \cite{Eastin2013thesis}.

\begin{figure}[ht]
  \centering
  \begin{quantikz}
    \lstick{$\ket{q}$} & \gate{S} & \gate{T} & \qw
  \end{quantikz}
  \caption{Sequential application of an \gls{phase} gate followed by a \gls{tgate}.}
  \label{fig:st-sequence}
\end{figure}

More generally, a rotation about a Pauli axis \(\sigma_{\alpha}\in\{X,Y,Z\}\) through an angle \(\theta\) is
\[
R_{\alpha}(\theta)=\exp\bigl(-i\tfrac{\theta}{2}\sigma_{\alpha}\bigr),
\]
with explicit examples
\[
R_{x}(\theta)=
\begin{pmatrix}
\cos\bigl(\tfrac{\theta}{2}\bigr) & -i\sin\bigl(\tfrac{\theta}{2}\bigr)\\[4pt]
-i\sin\bigl(\tfrac{\theta}{2}\bigr) & \cos\bigl(\tfrac{\theta}{2}\bigr)
\end{pmatrix},
\qquad
R_{z}(\theta)=
\begin{pmatrix}
e^{-i\theta/2} & 0\\[4pt]
0 & e^{i\theta/2}
\end{pmatrix}.
\]
The parameterised family \(\{R_{y}(\theta), R_{z}(\phi)\}\) forms a convenient basis for variational circuits, and any single-qubit unitary admits the Euler decomposition  
\(U=R_{z}(\lambda)\,R_{y}(\theta)\,R_{z}(\phi)\) up to global phase \cite{NielsenChuang2010,Kandala2017hardware}.  
Specialising \(\theta=0,\phi=\pi/2\) or \(\pi/4\) recovers the \gls{phase} and \gls{tgate} operations, respectively.

\subsubsection*{Multi-qubit gates: entangling operations}

Operations acting on two or more qubits can create quantum correlations that admit no classical description \cite{Bell1964}.  
Together with arbitrary single-qubit rotations they supply universality for quantum computation \cite{Barenco1995elementary}.

\subsubsection*{\glsentrylong{cnot}}

The \gls{cnot} gate flips the target qubit conditioned on the control being in the state $\ket{1}$ \cite{NielsenChuang2010}.  
In the ordered basis \{$\ket{00}$,$\ket{01}$,$\ket{10}$,$\ket{11}$\} its unitary is
\[
\gls{cnot}=
\begin{pmatrix}
1&0&0&0\\[4pt]
0&1&0&0\\[4pt]
0&0&0&1\\[4pt]
0&0&1&0
\end{pmatrix},
\]
implementing the map \(\ket{a,b}\mapsto\ket{a,a\oplus b}\).  
Applied to a superposition it generates entanglement, for example
\[
\bigl(\tfrac{\ket{0}+\ket{1}}{\sqrt{2}}\bigr)\!\otimes\!\ket{0}
\xrightarrow{\gls{cnot}}
\tfrac{\ket{00}+\ket{11}}{\sqrt{2}},
\]
a Bell state \cite{Bell1964}.  
\gls{cnot} is self-inverse and belongs to the Clifford group \cite{Gottesman1997stabilizer}.

\begin{figure}[ht]
  \centering
  \begin{quantikz}
    \lstick{$\ket{c}$} & \ctrl{1} & \qw \\
    \lstick{$\ket{t}$} & \targ{} & \qw
  \end{quantikz}
  \caption{Circuit symbol for the \gls{cnot} gate with control \(\ket{c}\) and target \(\ket{t}\) \cite{Koch2022quantikz}.}
  \label{fig:cnot-gate}
\end{figure}

\subsubsection*{\glsentrylong{cz}}

The \gls{cz} gate applies a phase flip to the joint state $\ket{11}$; its matrix is
\[
\gls{cz}=
\begin{pmatrix}
1&0&0&0\\[4pt]
0&1&0&0\\[4pt]
0&0&1&0\\[4pt]
0&0&0&-1
\end{pmatrix}
\]
\cite{NielsenChuang2010}.
It is related to \gls{cnot} by a Hadamard conjugation on the target,
\(\gls{cz}_{(c,t)}=\gls{hadamard}_{t}\,\gls{cnot}_{(c,t)}\,\gls{hadamard}_{t}\) \cite{Barenco1995elementary},  
and arises natively in several hardware platforms where controlled phase interactions dominate \cite{Arute2019supremacy}.

\subsubsection*{\glsentrylong{swap}}

The \gls{swap} gate exchanges the states of two qubits: \(\ket{a,b}\mapsto\ket{b,a}\).  
It decomposes into three \gls{cnot} gates,
\(
\gls{swap}_{(1,2)}=
\gls{cnot}_{1,2}\,
\gls{cnot}_{2,1}\,
\gls{cnot}_{1,2}
\) \cite{Barenco1995elementary}.  
Symbolically it is drawn as two crossing lines with \(\times\) markers on the intersection \cite{Koch2022quantikz}.

\subsubsection*{\glsentrylong{toffoli} and \glsentrylong{cswap} gates}

The three-qubit \gls{toffoli} gate flips a target conditioned on two controls being in $\ket{1}$.  
It is universal for reversible classical computation \cite{Bennett1973logical} and admits a fault-tolerant decomposition using six \glspl{cnot}, seven \glspl{tgate} and single-qubit \gls{hadamard} gates \cite{Amy2013tcount}.  

\begin{figure}[ht]
  \centering
  \begin{quantikz}
    \lstick{$\ket{c_1}$} & \ctrl{2} & \qw \\
    \lstick{$\ket{c_2}$} & \ctrl{1} & \qw \\
    \lstick{$\ket{t}$}   & \targ{} & \qw
  \end{quantikz}
  \caption{Symbol for the \gls{toffoli} gate \cite{Koch2022quantikz}.}
  \label{fig:toffoli}
\end{figure}

The \gls{cswap} gate swaps two targets conditioned on a single control and can be synthesised from \glspl{toffoli} \cite{FredkinGate1982}.  

\subsubsection*{Universality}

Any multi-qubit unitary can be approximated using single-qubit rotations together with \gls{cnot} or \gls{cz} gates \cite{Barenco1995elementary}.  
Consequently sets such as \(\{\gls{hadamard},\gls{tgate},\gls{cnot}\}\) support universal quantum computation, and high-level gates are routinely compiled into the \(R_{z}/R_{x}\)+\gls{cnot} primitives of hardware back-ends like IBM’s \(U_{3}\)+\gls{cnot} basis \cite{Cross2017ibm,fedoriaka2025decomposition}.

\subsection{Types of Quantum Circuits}

Just as one can categorize classical circuits (combinational, sequential, etc.), we can distinguish various types of quantum circuits by their structure and purpose.\cite{NielsenChuang2010} Here we survey several important categories, each illustrated by a simple diagram:

\subsubsection*{Single-Qubit Circuits}

A \textbf{single-qubit circuit} operates on a single qubit (or on multiple qubits but without entangling them).\cite{NielsenChuang2010} It consists of one or more single-qubit gates in sequence.\cite{Barenco1995elementary} Such circuits are conceptually the simplest, effecting arbitrary rotations on a single qubit's state.\cite{NielsenChuang2010} While a single qubit cannot exhibit entanglement, single-qubit subcircuits appear as components of larger algorithms (for example, state preparation or individual qubit rotations in a variational ansatz).\cite{Kandala2017hardware}

An example single-qubit circuit is shown below, taking an initial state $\ket{0}$ and applying a sequence of rotations ($H$, then $T$, then $X$), followed by a measurement:

\begin{quantikz}
\lstick{$\ket{0}$} & \gate{H} & \gate{T} & \gate{X} & \meter{} & \cw \\
\end{quantikz}

\noindent This circuit prepares the state $XTH\ket{0}$ and measures it (the outcome is a probabilistic function of the applied gates).\cite{NielsenChuang2010} In general, any single-qubit unitary can be implemented by an appropriate sequence of $H$, $S$, $T$ (or other rotation) gates, as discussed above.\cite{Dawson2005solovay} Single-qubit circuits are often used to calibrate hardware or illustrate basic quantum phenomena like Bloch-sphere rotations.\cite{Barends2014superconducting}

\subsubsection*{Multi-Qubit Circuits}

A \textbf{multi-qubit circuit} involves two or more qubits with gates that act on multiple qubits (such as CNOT or other entangling gates).\cite{Barenco1995elementary} These circuits can generate entanglement and are necessary for computational tasks where qubit interactions are required.\cite{Bell1964} Multi-qubit circuits range from small entangling subroutines (like creating a Bell pair) to large circuits comprising many interacting gates.\cite{NielsenChuang2010}

As a basic example, consider a two-qubit circuit that creates a Bell state.\cite{Bell1964} Starting from $\ket{00}$, we apply a Hadamard on the first qubit and then a CNOT with the first qubit as control and second as target:

\begin{quantikz}
\lstick{$\ket{0}$} & \gate{H} & \ctrl{1} & \qw \\
\lstick{$\ket{0}$} & \qw   & \targ{} & \qw
\end{quantikz}

\noindent After these gates, the qubits are in the entangled state $\frac{1}{\sqrt{2}}(\ket{00} + \ket{11})$, one of the four Bell states.\cite{Bell1964} In general, multi-qubit circuits may involve many entangling gates.\cite{Arute2019supremacy} For instance, quantum adders, error-correcting code circuits, or oracle circuits for algorithms all involve networks of CNOTs (and related gates) spread across multiple qubits.\cite{Gottesman1997stabilizer} Multi-qubit circuits are the backbone of quantum algorithms, as they carry out the entangling operations that give quantum computing its power.\cite{Shor1994}

\subsubsection*{Parameterized (Variational) Circuits}

A \textbf{parameterized quantum circuit} (PQC) is a circuit that contains gates with continuous parameters (angles) which can be adjusted.\cite{Peruzzo2014vqe} These circuits are central to many \emph{variational quantum algorithms} and quantum machine-learning models.\cite{Cerezo2021variational} By treating gate angles as tunable parameters, one can use a classical optimization loop to iteratively adjust these parameters and minimize a cost function evaluated via quantum measurements.\cite{Peruzzo2014vqe} Such circuits are also called \emph{variational circuits} or \emph{ansatz circuits}.\cite{Kandala2017hardware}

Parameterized circuits often have a regular structure, e.g.\ layers of single-qubit rotations and entangling gates.\cite{Kandala2017hardware} An example 2-qubit parameterized circuit (one layer of a common variational ansatz) is shown below:

\begin{quantikz}
\lstick{$\ket{0}$} & \gate{R_y(\theta_1)} & \ctrl{1} & \gate{R_z(\phi_1)} & \qw \\
\lstick{$\ket{0}$} & \gate{R_y(\theta_2)} & \targ{} & \gate{R_z(\phi_2)} & \qw
\end{quantikz}

\noindent Here each qubit is rotated by some angle $\theta_i$ about $Y$ and then entangled, followed by $Z$-rotations by $\phi_i$. The angles ${\theta_i,\phi_i}$ are free parameters that can be optimized.\cite{Peruzzo2014vqe} Stacking multiple such layers increases the expressive power of the ansatz at the cost of more parameters.\cite{Sim2019expressibility} Variational circuits are used in algorithms like the Variational Quantum Eigensolver (for ground-state energies) and QAOA (Quantum Approximate Optimization Algorithm), as well as in quantum classifiers and neural networks.\cite{Farhi2014qaoa} They exemplify a hybrid quantum-classical approach: the quantum circuit provides a parameterized function, and a classical routine tunes the parameters.\cite{Cerezo2021variational}

\subsubsection*{Measurement-Based Circuits}

A \textbf{measurement-based quantum circuit} refers to a model of quantum computation (the one-way quantum computer) where computation is driven by measurements on an entangled resource state.\cite{Raussendorf2001oneway} One first prepares a large entangled state (typically a \emph{cluster state}) and then performs a sequence of single-qubit measurements.\cite{Briegel2009measurement} The measurement bases and any feed-forward adjustments can implement an arbitrary quantum computation, despite using only measurements during the “computing” phase.\cite{Raussendorf2003measurement}

For example, consider a simple three-qubit linear cluster state.\cite{Raussendorf2001oneway} We start with three qubits in $\ket{+}$ states, entangle neighbors via CZ gates, then measure qubits 1 and 2 while qubit 3 remains unmeasured:

\begin{quantikz}
\lstick{$\ket{+}$} & \ctrl{1} & \qw   & \meter{} & \cw \\
\lstick{$\ket{+}$} & \targ{} & \ctrl{1} & \meter{} & \cw \\
\lstick{$\ket{+}$} & \qw   & \targ{} & \qw
\end{quantikz}

\noindent Depending on the measurement outcomes, the state of qubit 3 corresponds to the result of a quantum computation; conditional $X$ or $Z$ corrections (feed-forward) may be applied.\cite{Raussendorf2003measurement} This model is equivalent in power to the standard gate model—any gate circuit can be translated to a measurement pattern on a cluster state.\cite{Briegel2009measurement}

\subsubsection*{Hybrid Quantum-Classical Circuits}

In the current \gls{nisq} era, many practical algorithms use a \textbf{hybrid quantum-classical} approach, wherein quantum circuits are interleaved with classical processing.\cite{Preskill2018nisq} Variational circuits described above are a prime example: the quantum processor prepares a state and performs measurements, and a classical computer processes those results to adjust parameters for the next round.\cite{Cerezo2021variational} Another example is quantum error correction with feedback, where measurement outcomes (syndrome bits) are fed to classical logic that decides further quantum operations in real-time.\cite{Kelly2015error}

Hybrid circuits leverage the strengths of both paradigms: quantum circuits for state preparation and interference on exponentially large state spaces, and classical computations for flexible control and optimization.\cite{Preskill2018nisq} 

\noindent In this flow, the quantum circuit is executed and measured to evaluate a cost function, and a classical optimizer computes new parameters for the next quantum run.\cite{Cerezo2021variational} Hybrid circuits thus are not a single static circuit but a sequence of partial circuits and classical computations.\cite{Preskill2018nisq} Nonetheless, one can represent a single iteration in an expanded quantum circuit diagram by explicitly including measurement operations mid-circuit and classically-controlled gates.\cite{Kelly2015error}

In summary, the above categories illustrate the diversity of quantum circuit styles. A given quantum algorithm may encompass several of these aspects.\cite{Shor1994} Understanding the circuit types helps in designing and optimizing quantum algorithms for real hardware.\cite{Arute2019supremacy}


\subsection{Example Quantum Algorithms as Circuits}

To see these gates and circuit paradigms in action, we now examine three foundational quantum algorithms and their circuit implementations: the Deutsch-Jozsa algorithm, Grover's search algorithm, and the Quantum Fourier Transform.\cite{NielsenChuang2010} For each, we describe the circuit and how the algorithm leverages quantum gates to achieve a speed-up or functionality beyond classical means.\cite{Preskill2018nisq}

\subsubsection*{Deutsch-Jozsa Algorithm}

The Deutsch-Jozsa (DJ) algorithm is one of the first examples of a quantum algorithm that outperforms the classical deterministic result for a specific problem.\cite{Deutsch1992rapid} The task is to determine whether a Boolean function $f:\{0,1\}^n \to \{0,1\}$ is constant or \emph{balanced}, promised that $f$ is one or the other.\cite{Deutsch1992rapid} Classically, in the worst case $2^{n-1}+1$ queries are required.\cite{Cleve1998dj} The Deutsch-Jozsa quantum algorithm solves it with a single query to an oracle $U_f$ that implements $f$.\cite{Deutsch1992rapid}

The circuit for the Deutsch-Jozsa algorithm is as follows.\cite{NielsenChuang2010} We have $n$ qubits for the input (initialized to $\ket{0\cdots0}$) and one output qubit (initialized to $\ket{1}$). First, a layer of Hadamard gates is applied to all qubits, creating a uniform superposition and putting the output qubit in $\ket{-}$.\cite{Deutsch1992rapid} Next, the oracle $U_f$ is applied, mapping $\ket{x}\ket{y}\to\ket{x}\ket{y\oplus f(x)}$.\cite{Cleve1998dj} Finally, another Hadamard layer is applied to the $n$ input qubits, and they are measured.\cite{NielsenChuang2010}

The complete circuit for $n=2$ is:

\begin{quantikz}
\lstick{$\ket{0}_{q_0}$} & \gate{H} & \gate[3,nwires=2]{U_f} & \gate{H} & \meter{} & \cw \\
\lstick{$\ket{0}_{q_1}$} & \gate{H} &                & \gate{H} & \meter{} & \cw \\
\lstick{$\ket{1}_{q_2}$} & \gate{H} &                & \qw   & \qw   &
\end{quantikz}

In this diagram, $U_f$ is a multi-qubit oracle acting on all three qubits.\cite{Cleve1998dj} The output qubit $q_2$ returns to $\ket{-}$ regardless of $f$, so it is ignored in the final measurement.\cite{NielsenChuang2010} Measuring the input qubits yields $0^n$ if $f$ is constant and a non-zero string if $f$ is balanced.\cite{Deutsch1992rapid}

This works because $U_f$ imprints $(-1)^{f(x)}$ as a phase on the input register (phase kickback).\cite{Cleve1998dj} The final Hadamards perform interference: identical phases yield constructive interference on $\ket{0\cdots0}$, while balanced phases cancel there and appear elsewhere, distinguishing the two cases with certainty in one query.\cite{NielsenChuang2010} The DJ algorithm illustrates how superposition and interference evaluate a global property of a function efficiently.\cite{Deutsch1992rapid}

\subsubsection*{Grover's Search Algorithm}

Grover's algorithm searches an unstructured list of $N=2^n$ items for a marked item, achieving a quadratic speed-up over classical search.\cite{Grover1997fast} Classically one needs $O(N)$ queries; Grover requires $O(\sqrt{N})$.\cite{Boyer1998tight} The algorithm iteratively applies the Grover operator $G = D\,U_f$, where $U_f$ flips the phase of the marked state and $D$ (diffusion) inverts amplitudes about the average.\cite{Brassard2002amplification}

A Grover iteration on $n$ qubits is:

\begin{quantikz}
 \lstick{$\ket{0}^{\otimes n}$}
   & \gate{H^{\otimes n}}
   & \gate{U_f}
   & \gate{D}
   & \meter{} \qwbundle{n}
\end{quantikz}


For $n=2$, marking $\ket{11}$ as the solution, an explicit circuit is:

\begin{quantikz}
\lstick{$\ket{0}$} & \gate{H} & \ctrl{1} & \gate{H} & \gate{X} & \ctrl{1} & \gate{X} & \gate{H} & \meter{} & \cw \\
\lstick{$\ket{0}$} & \gate{H} & \gate{Z} & \gate{H} & \gate{X} & \targ{} & \gate{X} & \gate{H} & \meter{} & \cw
\end{quantikz}

Here the first Hadamards create a uniform superposition.\cite{Grover1997fast} The oracle $U_f$ is the controlled-$Z$ on $\ket{11}$.\cite{Brassard2002amplification} The remaining gates implement diffusion: $H^{\otimes n}X^{\otimes n}Z_{\ket{0^n}}X^{\otimes n}H^{\otimes n}$.\cite{NielsenChuang2010} Repeating $r\approx\frac{\pi}{4}\sqrt{N}$ iterations maximizes success probability.\cite{Grover1997fast,Boyer1998tight} Grover thus exploits amplitude amplification to boost the marked state's probability quadratically faster than classical search.\cite{Brassard2002amplification}


\subsubsection*{Quantum Fourier Transform (QFT)}

The Quantum Fourier Transform is a quantum analogue of the discrete Fourier transform applied to the amplitudes of a quantum state.\cite{NielsenChuang2010} It is a key component in many quantum algorithms, including Shor's factoring and quantum phase estimation.\cite{Shor1994} The QFT on $n$ qubits is a unitary $U_{\mathrm{QFT}}$ that maps a basis state $\ket{j}$ to a phase-weighted superposition:
$$
\ket{j} \;\mapsto\; \frac{1}{2^{n/2}}\sum_{k=0}^{2^{n}-1} e^{2\pi i jk / 2^{n}}\,\ket{k}.
$$\cite{NielsenChuang2010}

The standard QFT circuit uses $O(n^2)$ one- and two-qubit gates: Hadamards and controlled phase rotations.\cite{Cleve1998qft} For three qubits ($n=3$), the circuit (ignoring final swaps) is:

\begin{quantikz}
\lstick{$\ket{q_0}$} & \gate{H} & \ctrl{1} & \ctrl{2} & \qw   & \qw   & \qw   & \meter{} & \cw \\
\lstick{$\ket{q_1}$} & \qw   & \gate{R_2} & \qw   & \gate{H} & \ctrl{1} & \qw   & \meter{} & \cw \\
\lstick{$\ket{q_2}$} & \qw   & \qw   & \gate{R_3} & \qw   & \gate{R_2} & \gate{H} & \meter{} & \cw
\end{quantikz}

Here $R_k$ denotes a $Z$-rotation by $2\pi/2^{k}$.\cite{NielsenChuang2010} Swapping $q_0$ and $q_2$ at the end yields the textbook output order.\cite{Cleve1998qft} In general, the QFT circuit requires $n(n+1)/2$ gates, exponentially fewer than a classical DFT over $2^{n}$ elements encoded naïvely in gates.\cite{NielsenChuang2010} After QFT, measuring the qubits in algorithms like phase estimation reveals frequency-domain information encoded in the phases.\cite{Kitaev1995phase}

An approximate QFT can omit small-angle rotations to reduce gate count further with bounded error, a common optimization on NISQ hardware.\cite{Barenco1996approxqft} Nonetheless, the exact QFT circuit remains a canonical example of an efficient, highly structured quantum circuit.\cite{Shor1994}

\subsection{Decomposition of Arbitrary Unitaries into Quantum Circuits}

Given an arbitrary unitary operation $U$ on $n$ qubits ($2^{n}\times2^{n}$ matrix), we ask how to implement it as a gate sequence from a universal set.\cite{Shende2006synthesis} This is the problem of \textbf{quantum circuit synthesis} or unitary decomposition.

A general strategy breaks $U$ into a product of simpler unitaries that affect only a two-dimensional subspace—\emph{two-level unitaries}.\cite{Reck1994optics} Fedoriaka's algorithm (2019) follows this approach, eliminating off-diagonal elements one by one with two-level operations.\cite{fedoriaka2025decomposition} Each two-level unitary acts non-trivially on basis states $\ket{i},\ket{j}$ and as identity elsewhere, analogous to a Givens rotation.\cite{Reck1994optics}

\paragraph{Two-Level Decomposition and Gray Codes.} 
Choosing the sequence so that $\ket{i}$ and $\ket{j}$ differ in exactly one qubit simplifies implementation: the two-level unitary becomes a single-qubit rotation controlled on the other $n-1$ qubits.\cite{Barenco1995elementary} Ordering basis states in a Gray-code sequence ensures consecutive pairs differ by one bit.\cite{Bullock2004gray} One conceptually permutes $U$ to Gray order with a permutation $P$ and then decomposes $PUP^T$; the permutation itself can be realized by fixed SWAP networks or absorbed into labeling.\cite{Bullock2004gray}

\paragraph{Fully-Controlled Rotation Implementation.} 
If $\ket{i}$ and $\ket{j}$ differ only in qubit $k$, the required operation is a single-qubit rotation on $k$ controlled on the other qubits matching the shared bit pattern—a fully controlled gate $C^{\,n-1}(R_y(\theta))$ or $C^{\,n-1}(R_z(\phi))$.\cite{fedoriaka2025decomposition} Such gates may be further decomposed into CNOTs plus single-qubit rotations or implemented directly if native controls are available.\cite{Barenco1995elementary}

Applying all such fully controlled rotations (and necessary $X$ flips on controls) yields a circuit of roughly $4^{n}$ basic gates, matching the asymptotic lower bound for exact synthesis of a generic $n$-qubit unitary.\cite{Shende2006synthesis}

\paragraph{Circuit Complexity and Optimizations.} 
The number of two-level unitaries required is $\tfrac12\,2^{n}(2^{n}-1)=\Theta(4^{n})$, reflecting the $4^{n}$ real degrees of freedom in a generic $2^{n}\times2^{n}$ unitary.\cite{Shende2006synthesis} Consequently, any \emph{exact} circuit for an arbitrary unitary must use at least $\Omega(4^{n})$ elementary gates; Fedoriaka's Gray-code method saturates this bound up to constant factors.\cite{fedoriaka2025decomposition}

Fedoriaka also describes practical optimizations: consecutive $X$ flips on the same qubit often cancel, reducing gate count without altering functionality.\cite{fedoriaka2025decomposition} A final global phase can be fixed with a single $R_1$ gate, so one need not track global phases throughout the decomposition.\cite{Reck1994optics} 
If $U$ happens to be sparse or block-diagonal, many off-diagonal elements are already zero and corresponding two-level rotations can be skipped, yielding circuits far shorter than the $4^{n}$ worst case.\cite{Bullock2004gray} When $U$ factors as a tensor product of single-qubit unitaries, only $n$ gates are needed.\cite{Barenco1995elementary}

Beyond Fedoriaka's elimination, other exact synthesis techniques exist. The \emph{cosine-sine decomposition} (CSD) recursively splits $U$ into smaller blocks, leading to the \emph{quantum Shannon decomposition} family of circuits with similar $\Theta(4^{n})$ size but often nicer structure (uniformly-controlled rotations).\cite{Miller2006csd} Householder-reflection methods achieve the same bound with different gate patterns.\cite{Shende2006synthesis}

If approximation suffices, the Solovay-Kitaev theorem guarantees any unitary can be approximated to error $\varepsilon$ with length $O\bigl(\log^{c}(1/\varepsilon)\bigr)$ over a fixed universal set ($c\approx3.97$), independent of $n$ once an exact synthesis for basis gates is available.\cite{Dawson2005solovay} Modern numerical compilers combine CSD back-bones with iterative \emph{KAK} or \emph{ZX-calculus} reductions to trade accuracy for shorter depth on NISQ devices.\cite{Heyfron2018zx}

\paragraph{Summary.} 
Arbitrary $n$-qubit unitaries can be decomposed exactly by a sequence of two-level operations ordered via Gray codes, each realized as a fully-controlled single-qubit rotation.\cite{fedoriaka2025decomposition} The resulting circuit uses $\Theta(4^{n})$ gates, matching information-theoretic lower bounds.\cite{Shende2006synthesis} Further optimizations—gate-cancellation, exploiting sparsity, or adopting alternative decompositions such as CSD—trim constant factors but not the exponential scaling. Hence, practical quantum advantage hinges on exploiting \emph{structured} unitaries (QFT, oracles, variational ansätze) whose circuits grow only polynomially.\cite{Preskill2018nisq}
