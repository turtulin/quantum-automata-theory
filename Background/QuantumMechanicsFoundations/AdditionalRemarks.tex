\subsection{Unitary Evolution and Quantum Dynamics}
\label{subsec:unitary_evolution}

\begin{definition}[Unitary Evolution]
In a closed quantum system, the state evolution is governed by the Schrödinger equation,
\[
i\hbar \frac{d}{dt} |\psi(t)\rangle = H |\psi(t)\rangle,
\]
where \(H\) is the Hamiltonian. The solution to this equation is given by
\[
|\psi(t)\rangle = U(t)|\psi(0)\rangle, \quad \text{with } U(t)=e^{-iHt/\hbar},
\]
where \(U(t)\) is a unitary operator.
\end{definition}

\begin{remark}
Unitary evolution is reversible and forms the basis for the operation of quantum circuits, where continuous evolution is discretised into sequences of quantum gates.
\end{remark}

\begin{theorem}[No-Cloning Theorem]
    Let \(\mathcal{H}\) be a Hilbert space with \(\dim \mathcal{H} \ge 2\). There is no unitary operator \(U\) such that for all states \(|\psi\rangle \in \mathcal{H}\) the following holds \cite{wootters1982single}:
    \[
    U\big(|\psi\rangle \otimes |0\rangle\big) = |\psi\rangle \otimes |\psi\rangle
    .\]
\end{theorem}

\begin{observation}
    The \textbf{no-cloning theorem} states that it is impossible to create an exact copy of an arbitrary unknown quantum state. This fundamental principle has significant implications for quantum information processing and quantum cryptography.
    The no-cloning theorem ensures that quantum information cannot be perfectly replicated, which underpins the security of many quantum cryptographic protocols such as BB84 \cite{bennett1984quantum}.
\end{observation}