\subsection{Decoherence and Open Systems}
\label{subsec:decoherence}

\begin{definition}[Decoherence]
\emph{Decoherence} is the process by which a quantum system loses its coherent properties due to interaction with its environment. This results in the decay of the off-diagonal elements in the system's density matrix, leading the system to behave more classically.
\end{definition}

\begin{remark}
Decoherence is a major obstacle in quantum computing because it degrades the quantum correlations needed for quantum parallelism and entanglement.
\end{remark}

\begin{definition}[Lindblad Master Equation]
The evolution of an open quantum system can be described by the \textbf{Lindblad master equation}:
\[
\frac{d\rho}{dt} = -\frac{i}{\hbar}\,[H, \rho] + \sum_k \left( L_k \rho L_k^\dagger - \frac{1}{2}\{L_k^\dagger L_k, \rho\} \right),
\]
where \(\rho\) is the density matrix, \(H\) is the Hamiltonian, and \(L_k\) are the Lindblad (noise) operators \cite{breuer2002theory}.
\end{definition}

\begin{example}[Noise Models]
    Typical noise models include:
    \begin{itemize}
        \item \textbf{Amplitude damping}: Models energy loss (e.g., spontaneous emission) \cite{nielsen2010quantum}.
        \item \textbf{Phase damping}: Represents the loss of phase coherence without energy dissipation \cite{nielsen2010quantum}.
    \end{itemize}
\end{example}

\begin{observation}
    To combat decoherence, quantum error correction codes (such as the Shor code \cite{shor1995scheme} and surface codes \cite{fowler2012surface}) and decoherence-free subspaces are employed.
\end{observation}
