\section{Quantum Mechanics Foundations}
\label{sec:qm-foundations}

Quantum theory delivers a linear-algebraic description of microscopic
systems that has withstood every experimental test to date
\cite{Dirac1930,NielsenChuang2010}. The finite-dimensional fragment of
this theory already suffices for reversible computing,
cryptography and the automata models investigated later in the thesis.
Accordingly the presentation below restricts attention to finite
Hilbert spaces and the unitary maps that act upon them.

The narrative proceeds from kinematic notions to dynamical ones.
First come pure quantum states, qubits and Dirac's bra-ket notation,
followed by the tensor product that builds composite systems and the
Bloch-sphere picture that aids visualisation of single-qubit
superposition. Entanglement is then introduced as the key non-classical
resource, together with Schmidt rank as a quantitative measure.
Measurement postulates appear next, encompassing both projective
measurements and the more general \glspl{povm} that
model noisy detectors. Density operators, partial trace and quantum
channels provide the language for open-system dynamics, while Kraus
decompositions formalise decoherence and error. Finally the Schrödinger
equation is specialised to finite-dimensional unitary evolution,
establishing the toolkit used in subsequent chapters on quantum finite
automata.

Relativistic extensions handled by quantum field theory, such as those described by the \gls{qft}, remain outside the present scope \cite{Weinberg1995}; every
example and proof will rely only on the finite-dimensional machinery
summarised here.

\subsection{Qubits and Quantum States}

Quantum information is encoded in the state of a two-level system, the
\gls{qubit}. Although physical realisations differ, ranging from
photonic polarisations and superconducting circuits to nuclear spins, the mathematical
model is uniform: a unit vector in a two-dimensional complex Hilbert
space \cite{Dirac1930,NielsenChuang2010}. Dirac's bra-ket notation
offers a compact syntax for such vectors and fixes
\(\{\ket{0},\ket{1}\}\) as the computational basis used
throughout the thesis.

\begin{definition}[Pure qubit state]\label{def:pure-qubit}
A pure state of a qubit is a unit vector
\[
 \ket{\psi}= \alpha\ket{0}+\beta\ket{1},
 \qquad
 \alpha,\beta\in\mathbb C,\;
 |\alpha|^{2}+|\beta|^{2}=1.
\]
Equivalently one may parameterise the state as
\[
 \ket{\psi}=\cos\frac{\theta}{2}\ket{0}
       +e^{\mathrm i\phi}\sin\frac{\theta}{2}\ket{1},
 \qquad
 \theta\in[0,\pi],\;\phi\in[0,2\pi),
\]
where the pair \((\theta,\phi)\) locates \(\ket{\psi}\) on the Bloch
sphere \cite{NielsenChuang2010}.
\end{definition}

The Bloch sphere embeds all pure qubit states on the surface of a unit
sphere in \(\mathbb R^{3}\); its poles represent the basis vectors while
global phases \(e^{\mathrm i\gamma}\ket{\psi}\) collapse to the same
point because overall phase has no observable consequence
\cite{Sakurai1994}. Figure~\ref{fig:bloch_sphere} displays the geometry
and the two angular coordinates that parameterise a state.

\begin{figure}[ht]
 \centering
 \begin{tikzpicture}[scale=1.5]
  \shade[ball color=blue!20, opacity=0.3] (0,0) circle (2cm);
  \draw[thick] (0,0) circle (2cm);
  \draw[dashed] (0,0) ellipse (2cm and 0.55cm);
  \draw[->, thick] (0,0) -- (2.5,0) node[right] {$y$};
  \draw[->, thick] (0,0) -- (0,2.5) node[above] {$z$};
  \draw[->, thick] (0,0) -- (-0.8,-0.8) node[right] {$x$};
  \fill (0,2) circle (0.05);
  \fill (0,-2) circle (0.05);
  \node[above, xshift=-0.4cm] at (0,2) {\(|0\rangle\)};
  \node[below, xshift=0.4cm] at (0,-2) {\(|1\rangle\)};
  \draw[red, ultra thick] (0,0) -- (0.5,1.4) node[right] {\(\;|\psi\rangle\)};
  \fill[red] (0.5,1.4) circle (0.05);
  \draw[dashed, red!40] (0.5,1.4) -- (0.5,-0.3);
  \draw[dashed, red!40] (0,0) -- (0.5,-0.3);
  \coordinate (O) at (0,0);
  \coordinate (N) at (0,2);
  \coordinate (B) at (0.5,1.4);
  \coordinate (T) at (0.5,-0.3);
  \coordinate (X) at (-0.8,-0.8);
  \pic [shift={(0,0.5)}, draw, "$\theta$", angle eccentricity=1.3, angle radius=1.2cm] {angle = B--O--N};
  \pic [draw, "$\phi$", angle eccentricity=1.5, angle radius=0.4cm] {angle = X--O--T};
 \end{tikzpicture}
 \caption{The Bloch sphere representation of a qubit.}
 \label{fig:bloch_sphere}
 \end{figure}

 Composite systems are modelled by tensor products. Two qubits \(A\) and
 \(B\) inhabit
 \(\mathcal H_{AB}=\mathcal H_{2,A}\otimes\mathcal H_{2,B}\), enabling
 entangled states such as the Bell pair
 \(\ket{\Phi^{+}}=\tfrac{1}{\sqrt2}(\ket{00}+\ket{11})\). Mixed states,
 described by density operators \(\rho\) with
 \(\rho\ge0\) and \(\operatorname{tr}\rho=1\), appear later when noise
 and measurement are introduced; for the purely kinematic discussion
 here, pure vectors suffice.
 
 The qubit therefore serves as the atomic carrier of quantum information,
 combining a continuous state space with a discrete computational basis.
 Subsequent subsections build on this representation, first to formalise
 measurement and decoherence, then to construct \glspl{qfa}
 whose internal registers are composed of a finite number of qubits.

 \subsection{Superposition and Entanglement}\label{subsec:superposition_entanglement}

 Quantum information is encoded in vectors of a finite-dimensional Hilbert space~$\mathcal H$.  
 Linearity of~$\mathcal H$ implies that a linear combination of admissible vectors is again an admissible vector.  
 
 \begin{definition}[Principle of superposition {\cite{Sakurai1994}}]
   Let $\{\ket{\psi_i}\}_{i\in I}\subset\mathcal H$ be a family of normalised state vectors and let $\{\alpha_i\}_{i\in I}\subset\mathbb C$ satisfy $\sum_{i\in I}\lvert\alpha_i\rvert^2=1$.  
   The vector $\ket{\Psi}=\sum_{i\in I}\alpha_i\ket{\psi_i}$ is a valid quantum state and is called a superposition of the~$\ket{\psi_i}$.
 \end{definition}
 
 Superposition yields interference phenomena that have no classical counterpart; amplitudes, not probabilities, add.  
 In protocols such as the Deutsch-Jozsa algorithm and amplitude amplification the relative phases between components of~$\ket{\Psi}$ are adjusted in order to concentrate probability mass on measurement outcomes that reveal global properties of the input~\cite{NielsenChuang2010}.  
 
 \begin{remark}
   For a single qubit any pure state can be written as $\ket{\psi}=\cos\frac\theta2\ket{0}+e^{i\phi}\sin\frac\theta2\ket{1}$ with $\theta\in[0,\pi]$ and $\phi\in[0,2\pi)$.  
   This parametrisation maps one-to-one to the Bloch sphere and offers geometric insight into unitary control operations.
 \end{remark}
 
 
 Composing systems is formalised by the tensor product.  
 Let $\mathcal H_A$ and $\mathcal H_B$ be Hilbert spaces for subsystems~$A$ and~$B$.  
 A joint pure state lives in $\mathcal H_{AB}=\mathcal H_A\otimes\mathcal H_B$.  
 Some joint states are separable; others are not.
 
 \begin{definition}[Entangled state {\cite{Horodecki2009}}]\label{def:entangled}
   A pure state $\ket{\Psi}_{AB}\in\mathcal H_{AB}$ is entangled if there exist no states $\ket{\psi}_A\in\mathcal H_A$ and $\ket{\phi}_B\in\mathcal H_B$ such that $\ket{\Psi}_{AB}=\ket{\psi}_A\otimes\ket{\phi}_B$.
 \end{definition}
 
 \begin{example}[Bell pair {\cite{Bell1964}}]
   The state
   \[
     \ket{\Psi^-}=\frac1{\sqrt2}\bigl(\ket{0}_A\ket{1}_B-\ket{1}_A\ket{0}_B\bigr)
   \]
   is entangled.  
   Tracing out either subsystem produces the maximally mixed single-qubit density operator $\frac12I$, confirming that no local description captures the correlations present in~$\ket{\Psi^-}$.
 \end{example}
 
 \begin{proposition}[Schmidt criterion {\cite{NielsenChuang2010}}]
   A bipartite pure state is separable if and only if its Schmidt rank equals~$1$.
 \end{proposition}
 
 \begin{proof}[Sketch]
   Write $\ket{\Psi}_{AB}=\sum_i\lambda_i\ket{u_i}_A\otimes\ket{v_i}_B$ with orthonormal families $\{\ket{u_i}\}$, $\{\ket{v_i}\}$ and $\lambda_i>0$.  
   If only one Schmidt coefficient is non-zero then $\ket{\Psi}_{AB}$ factorises; conversely, factorisation implies a single non-zero coefficient.  
 \end{proof}
 
 
 Entanglement is a resource underpinning quintessential quantum protocols.  
 It enables perfect teleportation of an unknown qubit state and doubles the classical capacity of a single qubit through superdense coding~\cite{Bennett1993}.  
 Experimentally, violations of Bell inequalities exclude local hidden-variable theories and confirm the non-classical character of entangled correlations~\cite{Aspect1982}.  
 
 \begin{remark}
   Entanglement extends beyond two parties.  
   Greenberger-Horne-Zeilinger states and graph states furnish multipartite resources for measurement-based quantum computation.  
   Resource theories quantify entanglement via monotones such as the von~Neumann entropy of the reduced state~\cite{Horodecki2009}.
 \end{remark}
 
 Entanglement appears implicitly in later chapters: composite state registers of \glspl{qfa} may evolve into non-separable configurations even when each register individually undergoes unitary dynamics.  
 Understanding superposition and entanglement is therefore prerequisite for analysing interference patterns and acceptance probabilities in \glspl{qfa} models.
 \subsection{Measurement and Probabilistic Outcomes}\label{subsec:measurement_probabilistic}

Measurement translates the abstract formalism of quantum mechanics into experimentally accessible numbers.  
The bridge is supplied by the measurement postulates, stated here for finite-dimensional systems~\cite{vonNeumann1955,NielsenChuang2010}.

\begin{definition}[Observable {\cite{vonNeumann1955}}]\label{def:observable}
	An observable is a Hermitian operator $M\in\mathcal L(\mathcal H)$ that admits the spectral decomposition
	\[
		M=\sum_{m}m\,\Pi_{m},
	\]
	with orthogonal projectors $\{\Pi_{m}\}$ satisfying $\Pi_{m}\Pi_{m'}=\delta_{m,m'}\Pi_{m}$ and $\sum_{m}\Pi_{m}=\mathbb I$.
\end{definition}

\begin{definition}[Projective measurement {\cite{Born1926}}]\label{def:projective_measurement}
	Let $M$ be an observable with eigenstates $\{\ket{m}\}$ and projectors $\Pi_{m}= \ket{m}\bra{m}$.  
	For an input state $\ket{\psi}$ the probability of obtaining outcome $m$ is
	\[
		P(m)=\braket{\psi|\Pi_{m}|\psi}=|\braket{m}{\psi}|^{2},
	\]
	and the post-measurement state is
	\[
		\ket{\psi}\longmapsto\frac{\Pi_{m}\ket{\psi}}{\sqrt{P(m)}}=\ket{m}.
	\]
\end{definition}

Projective measurements are often too restrictive: realistic detectors have finite resolution, and many protocols require non-orthogonal outcomes.  
The most general measurement allowed by quantum theory is the \gls{povm}, introduced next.

\begin{definition}[\gls{povm} {\cite{Kraus1983,NielsenChuang2010}}]\label{def:povm}
	A set of positive semi-definite operators $\{E_{k}\}\subset\mathcal L(\mathcal H)$ is a \gls{povm} if $\sum_{k}E_{k}=\mathbb I$.  
	For an input state $\ket{\psi}$ the probability of outcome $k$ is
	\[
		P(k)=\bra{\psi}E_{k}\ket{\psi}.
	\]
	There exist operators $\{M_{k}\}$, called Kraus operators, such that $E_{k}=M_{k}^{\dagger}M_{k}$ and $\sum_{k}M_{k}^{\dagger}M_{k}=\mathbb I$.
\end{definition}

\begin{proposition}[State-update rule {\cite{Kraus1983}}]\label{prop:kraus_update}
	When outcome $k$ of a \gls{povm} is observed, the (unnormalised) post-measurement state is
	$M_{k}\ket{\psi},$
	and the normalised state equals $M_{k}\ket{\psi}/\sqrt{P(k)}$.
\end{proposition}

\begin{example}[Computational-basis measurement]\label{ex:computational_basis}
	For a single qubit define $E_{0}=\ket{0}\bra{0}$ and $E_{1}=\ket{1}\bra{1}$.  
	With $\ket{\psi}=\alpha\ket{0}+\beta\ket{1}$, $|\alpha|^{2}+|\beta|^{2}=1$, one finds
	\[
		P(0)=|\alpha|^{2},\qquad P(1)=|\beta|^{2},
	\]
	in agreement with the Born rule; the post-measurement state collapses to the eigenstate corresponding to the observed outcome.
\end{example}


Linearity of quantum dynamics has a profound implication for information processing.

\begin{theorem}[No-cloning {\cite{Wootters1982,Dieks1982}}]\label{thm:nocloning}
	There exists no unitary operator $U$ and no fixed blank state $\ket{b}$ such that
	\[
		U\bigl(\ket{\psi}\otimes\ket{b}\bigr)=\ket{\psi}\otimes\ket{\psi}
	\]
	holds for every pure state $\ket{\psi}\in\mathcal H$.
\end{theorem}

\begin{proof}[Sketch]
	Assume such a unitary $U$ exists and consider two non-orthogonal states $\ket{\psi}$ and $\ket{\phi}$.  
	Because unitaries preserve inner products,
	\[
		\braket{\psi|\phi}=\bigl(\braket{\psi|\phi}\bigr)^{2},
	\]
	which implies $\braket{\psi|\phi}\in\{0,1\}$.  
	This contradicts the premise that $\ket{\psi}$ and $\ket{\phi}$ are non-orthogonal; thus no perfect cloner exists.
\end{proof}

The impossibility of perfect cloning secures quantum key distribution, forbids noiseless amplification of unknown signals, and motivates approximate or probabilistic cloning strategies analysed in later chapters~\cite{Scarani2005}.
\subsection{Decoherence and Open Systems}\label{subsec:decoherence_open}

Isolated dynamics represent an idealisation; every realistic device exchanges energy and information with uncontrolled external degrees of freedom~\cite{Breuer2002}.  
A quantum description that ignores the environment replaces pure vectors by statistical operators.

\begin{definition}[Density operator {\cite{NielsenChuang2010}}]\label{def:density}
	A density operator on a Hilbert space~$\mathcal H$ is a positive semidefinite matrix $\rho\in\mathcal L(\mathcal H)$ that satisfies $\operatorname{Tr}\rho=1$.  
	It encodes an ensemble $\{p_{i},\ket{\psi_{i}}\}$ through $\rho=\sum_{i}p_{i}\ket{\psi_{i}}\bra{\psi_{i}}$.
\end{definition}

Pure states correspond to projectors $\rho^{2}=\rho$; mixed states satisfy $\rho^{2}\neq\rho$.


Let $\mathcal H_{S}\otimes\mathcal H_{E}$ denote the Hilbert space of a system~$S$ and its environment~$E$.  
If $\ket{\Psi}_{SE}$ evolves unitarily under $U_{SE}$ the reduced state
\[
	\rho_{S}(t)=\operatorname{Tr}_{E}\bigl[\,U_{SE}(t)\,\rho_{SE}(0)\,U_{SE}^{\dagger}(t)\bigr]
\]
is generally mixed and its off-diagonal elements in a preferred basis decay, a phenomenon called decoherence~\cite{Zurek2003,Schlosshauer2005}.  

\begin{definition}[Quantum channel]\label{def:channel}
	A linear map $\mathcal E:\mathcal L(\mathcal H)\to\mathcal L(\mathcal H)$ is a quantum channel if it is completely positive and trace preserving.
\end{definition}

\begin{proposition}[Kraus representation {\cite{Kraus1983}}]\label{prop:kraus}
	Every completely positive and trace preserving map admits operators $\{L_{k}\}$ on~$\mathcal H$ such that
	\[
		\mathcal E(\rho)=\sum_{k}L_{k}\rho L_{k}^{\dagger},
		\qquad
		\sum_{k}L_{k}^{\dagger}L_{k}=\mathbb I .
	\]
\end{proposition}

\begin{example}[Amplitude-damping channel]\label{ex:amplitude_damp}
	For a qubit let $L_{0}=\begin{psmallmatrix}1&0\\0&\sqrt{1-\gamma}\end{psmallmatrix}$ and
	$L_{1}=\begin{psmallmatrix}0&\sqrt{\gamma}\\0&0\end{psmallmatrix}$ with $0\le\gamma\le1$.  
	The map $\mathcal E_{\gamma}(\rho)=L_{0}\rho L_{0}^{\dagger}+L_{1}\rho L_{1}^{\dagger}$ models spontaneous emission at rate~$\gamma$.  
	Diagonal elements remain unchanged whereas the coherence term $\rho_{01}$ decays as $\rho_{01}\mapsto\sqrt{1-\gamma}\,\rho_{01}$, illustrating decoherence.
\end{example}


When environmental correlations relax quickly (Markovian limit) the family $\{\rho(t)\}_{t\ge0}$ forms a quantum dynamical semigroup with generator in \gls{gksl} form~\cite{Gorini1976,Lindblad1976}:

\begin{definition}[Lindblad master equation]\label{def:lindblad}
	For a system Hamiltonian $H$ and Lindblad operators $\{L_{k}\}$ the Markovian evolution obeys
	\[
		\frac{\mathrm d\rho}{\mathrm dt}= -\frac{\mathrm i}{\hbar}[H,\rho]
		+\sum_{k}\Bigl(L_{k}\rho L_{k}^{\dagger}-\tfrac12\{L_{k}^{\dagger}L_{k},\rho\}\Bigr).
	\]
\end{definition}

The generator guarantees complete positivity of the channel $e^{t\mathcal L}$ for all $t\ge0$ and will serve in later chapters to model noise acting on the internal registers of \glspl{qfa}.

\subsection{Unitary Evolution and Quantum Dynamics}\label{subsec:unitary_dynamics}

In the absence of measurements and uncontrollable perturbations, a quantum system is considered closed.
Its state vector evolves according to the postulate of unitary time development~\cite{NielsenChuang2010}.

\begin{definition}[Unitary operator]\label{def:unitary}
	A linear map $U:\mathcal H\longrightarrow\mathcal H$ is unitary if $U^{\dagger}U=UU^{\dagger}=\mathbb I$.  
	Unitary operators form a group under composition and preserve inner products:
	$\braket{\psi|\phi}= \braket{U\psi|U\phi}$ for all $\ket{\psi},\ket{\phi}\in\mathcal H$.
\end{definition}


Dynamics are generated by the system Hamiltonian~$H=H^{\dagger}$ through the time-dependent Schrödinger equation~\cite{Schrodinger1926}
\[
	\mathrm i\hbar\frac{\mathrm d}{\mathrm dt}\ket{\psi(t)} = H(t)\ket{\psi(t)} .
\]

\begin{proposition}[Propagator]\label{prop:propagator}
	The formal solution is
	\[
		\ket{\psi(t)} = U(t,t_{0})\ket{\psi(t_{0})},
		\qquad
		U(t,t_{0}) = \mathcal T
		\exp\Bigl[-\frac{\mathrm i}{\hbar}\int_{t_{0}}^{t}H(t')\,\mathrm dt'\Bigr],
	\]
	where $\mathcal T$ denotes the time-ordering operator~\cite{Sakurai1994}.  
	The propagator satisfies $U(t,t)=\mathbb I$ and the composition law $U(t_{2},t_{0})=U(t_{2},t_{1})U(t_{1},t_{0})$.
\end{proposition}

\begin{remark}[Time independent Hamiltonian]
	If $H$ is constant one has the closed expression
	\[
		U(t,t_{0}) = \exp\bigl[-\mathrm iH(t-t_{0})/\hbar\bigr] ,
	\]
	which implements a one-parameter unitary group generated by $H$~\cite{Sakurai1994}.
\end{remark}

\begin{example}[Qubit rotation about the $z$ axis]\label{ex:qubit_rotation}
	For a single qubit let $H=\tfrac{\hbar\omega}{2}\sigma_{z}$.  
	The evolution operator equals
	\[
		U(t,0)=\exp\bigl[-\mathrm i\omega t\,\sigma_{z}/2\bigr]
		=\begin{psmallmatrix}
			 e^{-\mathrm i\omega t/2} & 0 \\[2pt]
			 0 & e^{\mathrm i\omega t/2}
		  \end{psmallmatrix},
	\]
	which produces a phase rotation of the computational basis and leaves probabilities invariant.
\end{example}


Unitary dynamics conserve superposition amplitudes and entanglement resources.  
Non unitary modifications originate from coupling to external degrees of freedom or from measurements, modelled in the preceding subsections.  
A rigorous understanding of unitary time development is essential for analysing quantum algorithms, control protocols and the acceptance dynamics of quantum automata.
