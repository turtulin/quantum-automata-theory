% \section{Quantum Mechanics Foundations}
% \label{sec:quantum-foundations}

% 

\subsection{Qubits and Quantum States}
\label{subsec:qubits}

\begin{definition}[Qubit]
A \emph{\gls{qubit}} is the fundamental unit of quantum information. It is represented as a normalised vector in a two-dimensional complex Hilbert space,
\[
\mathcal{H} = \mathbb{C}^2.
\]
\end{definition}

\begin{notation}[Computational Basis]
The standard (computational) basis states for a qubit are defined as
\[
|0\rangle = \begin{pmatrix} 1 \\ 0 \end{pmatrix}, \quad |1\rangle = \begin{pmatrix} 0 \\ 1 \end{pmatrix}.
\]
\end{notation}

\begin{definition}[General Qubit State]
A general state of a \gls{qubit} is given by
\[
|\psi\rangle = \alpha|0\rangle + \beta|1\rangle, \quad \text{with } |\alpha|^2 + |\beta|^2 = 1,
\]
where \(\alpha,\beta \in \mathbb{C}\) are the \emph{probability amplitudes}.
\end{definition}

%TODO: give more context to this remark
\begin{remark}
Global phase factors—i.e. multiplying the state by an overall phase \(e^{i\gamma}\)—do not affect the physical properties of the qubit.
\end{remark}

\begin{example}[Bloch Sphere Representation]
  Any pure state of a \gls{qubit} can be written in the form
  \[
  |\psi\rangle = \cos\frac{\theta}{2}|0\rangle + e^{i\phi}\sin\frac{\theta}{2}|1\rangle,
  \]
  with \(\theta \in [0,\pi]\) and \(\phi \in [0,2\pi)\). Figure~\ref{fig:bloch_sphere} illustrates the \textbf{Bloch sphere} representation of a qubit \cite{nielsen2010quantum}.
\end{example}

\begin{figure}[h]
  \centering
  \begin{tikzpicture}[scale=1.5]
    \shade[ball color=blue!20, opacity=0.3] (0,0) circle (2cm);
    \draw[thick] (0,0) circle (2cm);
    
    \draw[dashed] (0,0) ellipse (2cm and 0.55cm);
    
    \draw[->, thick] (0,0) -- (2.5,0) node[right] {$y$};
    \draw[->, thick] (0,0) -- (0,2.5) node[above] {$z$};
    \draw[->, thick] (0,0) -- (-0.8,-0.8) node[right] {$x$};
    
    \fill (0,2) circle (0.05);
    \fill (0,-2) circle (0.05);
    \node[above, xshift=-0.4cm] at (0,2) {\(|0\rangle\)};
    \node[below, xshift=0.4cm] at (0,-2) {\(|1\rangle\)};
    
    \draw[red, ultra thick] (0,0) -- (0.5,1.4) node[right] {\(\;|\psi\rangle\)};
    \fill[red] (0.5,1.4) circle (0.05);

    \draw[dashed, red!40] (0.5,1.4) -- (0.5,-0.3);
    \draw[dashed, red!40] (0,0) -- (0.5,-0.3);
    
    \coordinate (O) at (0,0);
    \coordinate (N) at (0,2);
    \coordinate (B) at (0.5,1.4);
    \coordinate (T) at (0.5,-0.3);
    \coordinate (X) at (-0.8,-0.8);
    
    \pic [shift={(0,0.5)}, draw, "$\theta$", angle eccentricity=1.3, angle radius=1.2cm] {angle = B--O--N};
    \pic [draw, "$\phi$", angle eccentricity=1.5, angle radius=0.4cm] {angle = X--O--T};

  \end{tikzpicture}
  \caption{Bloch sphere representation of a qubit.}
  \label{fig:bloch_sphere}
\end{figure}

%TODO: specify why 0 and 1
\begin{observation}
  For multi-qubit systems, the overall state space is given by the tensor product of individual qubit spaces. For instance, a two-qubit system is described by
  \[
  |\psi\rangle = \sum_{i,j \in \{0,1\}} \alpha_{ij}\, |i\rangle \otimes |j\rangle, \quad \sum_{i,j} |\alpha_{ij}|^2 = 1.
  \]
  This exponential scaling of the state space underpins the potential of quantum parallelism \cite{nielsen2010quantum}.
\end{observation}



% 
\subsection{Superposition and Entanglement}
\label{subsec:superposition}

\begin{definition}[Superposition]
\emph{Superposition} is the principle that a quantum state may exist as a linear combination of basis states. This property enables quantum systems to be in multiple configurations simultaneously and is central to the power of quantum algorithms.
\end{definition}

\begin{example}[Hadamard Transformation]
    \label{ex:hadamard}
Applying the Hadamard gate \(H\) to the basis state \(|0\rangle\) creates a uniform superposition:
\[
H|0\rangle = \frac{|0\rangle + |1\rangle}{\sqrt{2}}, \quad H|1\rangle = \frac{|0\rangle - |1\rangle}{\sqrt{2}}.
\]
\end{example}

\begin{definition}[Entanglement]
\emph{Entanglement} is a uniquely quantum phenomenon where the state of a composite system cannot be expressed as a product of the states of its individual subsystems. In such cases, the measurement outcomes on one subsystem are intrinsically correlated with those on another.
\end{definition}

\begin{example}[Bell States]
The \textbf{Bell states} are examples of maximally entangled two-qubit states:
\[
|\Phi^+\rangle = \frac{|00\rangle + |11\rangle}{\sqrt{2}}, \quad
|\Phi^-\rangle = \frac{|00\rangle - |11\rangle}{\sqrt{2}},
\]
\[
|\Psi^+\rangle = \frac{|01\rangle + |10\rangle}{\sqrt{2}}, \quad
|\Psi^-\rangle = \frac{|01\rangle - |10\rangle}{\sqrt{2}}.
\]
\end{example}

\begin{example}[Multipartite Entangled States]
    Important multipartite entangled states include the \gls{ghz} state and the W state:
    \[
    |\text{GHZ}\rangle = \frac{|000\rangle + |111\rangle}{\sqrt{2}}, \quad
    |W\rangle = \frac{|001\rangle + |010\rangle + |100\rangle}{\sqrt{3}}.
    \]
    These states are essential for applications in quantum communication and quantum error correction \cite{greenberger1990bell}.
\end{example}

\begin{observation}
    Entanglement is the resource behind many quantum protocols such as quantum teleportation \cite{bennett1993teleporting} and superdense coding, and it plays a pivotal role in the computational speedup promised by algorithms like Shor's factorization algorithm \cite{shor1994algorithms}.
\end{observation}

% \subsection{Measurement and Probabilistic Outcomes}
\label{subsec:measurement-and-probabilistic-outcomes}
% \subsection{Decoherence and Open Systems}
\label{subsec:decoherence-and-open-systems}

 
% \subsection{Unitary Evolution and Quantum Dynamics}
\label{subsec:unitary_evolution}

\begin{definition}[Unitary Evolution]
In a closed quantum system, the state evolution is governed by the Schrödinger equation,
\[
i\hbar \frac{d}{dt} |\psi(t)\rangle = H |\psi(t)\rangle,
\]
where \(H\) is the Hamiltonian. The solution to this equation is given by
\[
|\psi(t)\rangle = U(t)|\psi(0)\rangle, \quad \text{with } U(t)=e^{-iHt/\hbar},
\]
where \(U(t)\) is a unitary operator.
\end{definition}

\begin{remark}
Unitary evolution is reversible and forms the basis for the operation of quantum circuits, where continuous evolution is discretised into sequences of quantum gates.
\end{remark}

\begin{theorem}[No-Cloning Theorem]
    Let \(\mathcal{H}\) be a Hilbert space with \(\dim \mathcal{H} \ge 2\). There is no unitary operator \(U\) such that for all states \(|\psi\rangle \in \mathcal{H}\) the following holds \cite{wootters1982single}:
    \[
    U\big(|\psi\rangle \otimes |0\rangle\big) = |\psi\rangle \otimes |\psi\rangle
    .\]
\end{theorem}

\begin{observation}
    The \textbf{no-cloning theorem} states that it is impossible to create an exact copy of an arbitrary unknown quantum state. This fundamental principle has significant implications for quantum information processing and quantum cryptography.
    The no-cloning theorem ensures that quantum information cannot be perfectly replicated, which underpins the security of many quantum cryptographic protocols such as BB84 \cite{bennett1984quantum}.
\end{observation}    

\section{Quantum Mechanics Foundations}
\label{sec:qm-foundations}

Quantum mechanics offers a mathematically consistent framework that accurately predicts the behaviour of microscopic systems \cite{Dirac1930,NielsenChuang2010}.  
The present subsection recalls the concepts that will be used throughout this thesis: quantum states and qubits, superposition and entanglement, the postulates of measurement, decoherence in open systems and the unitary dynamics of closed systems.  Relativistic extensions, treated by quantum field theory (\gls{qft}), lie outside the present scope \cite{Weinberg1995}.

\subsection{Qubits and Quantum States}

An isolated physical system is represented by a unit vector $\ket{\psi}$ in a complex Hilbert space $\mathcal H$ \cite{Dirac1930}.  
For a two‐level system—the quantum bit or qubit—the state space is $\mathcal H_2\cong\mathbb C^2$ and an orthonormal computational basis $\{\ket{0},\ket{1}\}$ can always be chosen \cite{NielsenChuang2010}.  
A pure qubit state is therefore written  
\begin{equation}
  \ket{\psi}= \alpha\ket{0}+\beta\ket{1},\qquad 
  \alpha,\beta\in\mathbb C,\;|\alpha|^{2}+|\beta|^{2}=1
\end{equation}\cite{NielsenChuang2010}.

Any pure state of a \gls{qubit} can be written in the form
\[
|\psi\rangle = \cos\frac{\theta}{2}|0\rangle + e^{i\phi}\sin\frac{\theta}{2}|1\rangle,
\]
with \(\theta \in [0,\pi]\) and \(\phi \in [0,2\pi)\). Figure~\ref{fig:bloch_sphere} illustrates the \textbf{Bloch sphere} representation of a qubit.

\begin{figure}[ht]
    \centering
    \begin{tikzpicture}[scale=1.5]
      \shade[ball color=blue!20, opacity=0.3] (0,0) circle (2cm);
      \draw[thick] (0,0) circle (2cm);
      \draw[dashed] (0,0) ellipse (2cm and 0.55cm);
      \draw[->, thick] (0,0) -- (2.5,0) node[right] {$y$};
      \draw[->, thick] (0,0) -- (0,2.5) node[above] {$z$};
      \draw[->, thick] (0,0) -- (-0.8,-0.8) node[right] {$x$};
      \fill (0,2) circle (0.05);
      \fill (0,-2) circle (0.05);
      \node[above, xshift=-0.4cm] at (0,2) {\(|0\rangle\)};
      \node[below, xshift=0.4cm] at (0,-2) {\(|1\rangle\)};
      \draw[red, ultra thick] (0,0) -- (0.5,1.4) node[right] {\(\;|\psi\rangle\)};
      \fill[red] (0.5,1.4) circle (0.05);
      \draw[dashed, red!40] (0.5,1.4) -- (0.5,-0.3);
      \draw[dashed, red!40] (0,0) -- (0.5,-0.3);
      \coordinate (O) at (0,0);
      \coordinate (N) at (0,2);
      \coordinate (B) at (0.5,1.4);
      \coordinate (T) at (0.5,-0.3);
      \coordinate (X) at (-0.8,-0.8);
      \pic [shift={(0,0.5)}, draw, "$\theta$", angle eccentricity=1.3, angle radius=1.2cm] {angle = B--O--N};
      \pic [draw, "$\phi$", angle eccentricity=1.5, angle radius=0.4cm] {angle = X--O--T};
    \end{tikzpicture}
    \caption{The Bloch sphere representation of a qubit.}
    \label{fig:bloch_sphere}
  \end{figure}

The global phase $e^{\mathrm i\phi}\ket{\psi}$ leaves all physical predictions unchanged \cite{Sakurai1994}.  
For a composite system the total state space is the tensor product of the subsystems, e.g.\ two qubits $A$ and $B$ live in $\mathcal H_{AB}=\mathcal H_{2,A}\otimes\mathcal H_{2,B}$ \cite{NielsenChuang2010}.  

\subsection{Superposition and Entanglement}

Because $\mathcal H$ is linear, any normalised linear combination of allowed states is again a valid state, a fact known as the principle of superposition \cite{Sakurai1994}.  
In multi‐partite systems some states cannot be written as tensor products of subsystem states; such non‐separable states are entangled \cite{Einstein1935,Bell1964,Horodecki2009}.  
A paradigmatic two‐qubit entangled state is
\begin{equation}
  \ket{\Psi^{-}}=\frac{1}{\sqrt2}\bigl(\ket{0}_A\ket{1}_B-\ket{1}_A\ket{0}_B\bigr) 
\end{equation}\cite{Bell1964}.

\begin{definition}[Entangled state {\cite{Horodecki2009}}]
A pure state $\ket{\Psi}_{AB}$ is entangled if no vectors $\ket{\psi}_A\in\mathcal H_A$ and $\ket{\phi}_B\in\mathcal H_B$ exist such that $\ket{\Psi}_{AB}=\ket{\psi}_A\otimes\ket{\phi}_B$.
\end{definition}

Entanglement enables protocols such as quantum teleportation and superdense coding \cite{Bennett1993,NielsenChuang2010} and its non-local correlations have been confirmed experimentally \cite{Aspect1982}.  

\subsection{Measurement and Probabilistic Outcomes}

Every observable is represented by a Hermitian operator $M$ acting on $\mathcal H$ \cite{vonNeumann1955}.  
A projective (von Neumann) measurement of $M$ with eigenstates $\{\ket{m}\}$ returns outcome $m$ with probability 
\begin{equation}
  P(m)=|\braket{m}{\psi}|^{2}
\end{equation} \cite{Born1926},
after which the post‐measurement state collapses to $\ket{m}$ \cite{vonNeumann1955}.  
Generalised measurements are described by a set of operators $\{M_k\}$ satisfying $\sum_k M_k^{\dagger}M_k=\mathbb I$; the probability of outcome $k$ is $\bra{\psi}M_k^{\dagger}M_k\ket{\psi}$ and the (unnormalised) post‐measurement state is $M_k\ket{\psi}$ \cite{NielsenChuang2010}.  

A fundamental consequence of linearity is the no‐cloning theorem: no physical process can map an arbitrary unknown state onto two identical copies \cite{Wootters1982,Dieks1982}.  

\subsection{Decoherence and Open Systems}

Realistic systems interact with uncontrolled environments and are therefore open \cite{Breuer2002}.  
The state of an open system is described by a density matrix $\rho$, a positive semidefinite operator with $\operatorname{Tr}\rho=1$ \cite{NielsenChuang2010}.  
Tracing out environmental degrees of freedom generally converts a pure state into a mixed one and suppresses off-diagonal coherences, a process known as decoherence \cite{Zurek2003,Schlosshauer2005}.  
In the Markovian limit the time evolution of $\rho$ obeys the Gorini–Kossakowski–Sudarshan–Lindblad master equation  
\begin{equation}
  \frac{\mathrm d\rho}{\mathrm dt}=-\frac{\mathrm i}{\hbar}[H,\rho]
  +\sum_k\Bigl(L_k\rho L_k^{\dagger}-\tfrac12\{L_k^{\dagger}L_k,\rho\}\Bigr)
\end{equation} \cite{Gorini1976,Lindblad1976}.

\subsection{Unitary Evolution and Quantum Dynamics}

When external perturbations and measurements are absent, the evolution of a closed system is unitary \cite{NielsenChuang2010}.  
The Schrödinger equation  
\begin{equation}
  \mathrm i\hbar\frac{\mathrm d}{\mathrm dt}\ket{\psi(t)} = H\ket{\psi(t)}
\end{equation} \cite{Schrodinger1926}
has the formal solution $\ket{\psi(t)}=U(t,t_0)\ket{\psi(t_0)}$ with
\begin{equation}
  U(t,t_0)=\mathcal T\exp\!\Bigl[-\frac{\mathrm i}{\hbar}\int_{t_0}^{t} H(t')\,\mathrm dt'\Bigr]
\end{equation} \cite{Sakurai1994}.
Unitarity preserves inner products and thus total probability \cite{NielsenChuang2010}.  
For time-independent $H$ one simply has $U(t,t_0)=\exp[-\mathrm iH(t-t_0)/\hbar]$ \cite{Sakurai1994}.  
Under unitary dynamics superposition and entanglement are maintained, while measurements or environmental couplings introduce non-unitary changes as discussed above \cite{NielsenChuang2010}.  
