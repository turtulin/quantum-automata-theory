
\subsection{Decoherence and Open Systems}
\label{subsec:decoherence}

In an ideal (closed) quantum system, evolution is unitary. However, in practice, quantum systems interact with their environments, leading to \textbf{decoherence}. Decoherence describes the loss of quantum coherence (i.e. the decay of off-diagonal elements in the density matrix) and causes the system to behave more classically.

The dynamics of an open quantum system are often modeled by the \textbf{Lindblad master equation}:
\[
\frac{d\rho}{dt} = -\frac{i}{\hbar}\,[H, \rho] + \sum_k \left( L_k \rho L_k^\dagger - \frac{1}{2}\{L_k^\dagger L_k, \rho\} \right),
\]
where \(\rho\) is the density matrix of the system, \(H\) is the system Hamiltonian, and \(L_k\) are the Lindblad (noise) operators \cite{breuer2002theory}. Typical noise models include:
\begin{itemize}
    \item \textbf{Amplitude damping:} Models energy loss (e.g., spontaneous emission).
    \item \textbf{Phase damping:} Represents loss of phase coherence without energy dissipation.
\end{itemize}

Decoherence is a central challenge in quantum computation and communication. To mitigate its effects, one employs \textbf{quantum error correction} codes (e.g., Shor code \cite{shor1995scheme}) and develops decoherence–free subspaces.
