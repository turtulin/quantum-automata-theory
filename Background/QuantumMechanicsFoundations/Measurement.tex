
\subsection{Measurement and Probabilistic Outcomes}
\label{subsec:measurement}

Measurement in quantum mechanics is a fundamentally probabilistic process. When a quantum system in state
\[
|\psi\rangle = \sum_i \alpha_i|i\rangle
\]
is measured in the orthonormal basis \(\{|i\rangle\}\), the \textbf{Born rule} states that the outcome corresponding to \(|i\rangle\) is observed with probability
\[
P(i) = |\alpha_i|^2.
\]
This process is typically described as a \emph{projective measurement}, after which the state collapses to the observed eigenstate.

More generally, measurements can be described by \textbf{Positive Operator-Valued Measures (POVMs)}, which provide a framework for describing generalized measurements that are not necessarily projective. In a POVM, each measurement outcome is associated with a positive operator \(E_i\) satisfying \(\sum_i E_i = I\). The probability of outcome \(i\) is then given by
\[
P(i) = \langle\psi| E_i |\psi\rangle.
\]

For example, measuring the Bell state \( |\Phi^+\rangle = \frac{|00\rangle + |11\rangle}{\sqrt{2}} \) in the computational basis yields the outcomes \( |00\rangle \) or \( |11\rangle \) with 50\% probability each, as summarized in Table~\ref{tab:bell_measurement}.

\begin{table}[h]
\centering
\caption{Measurement outcomes for \( |\Phi^+\rangle \).}
\label{tab:bell_measurement}
\begin{tabular}{|c|c|}
\hline
\textbf{Outcome} & \textbf{Probability} \\ \hline
\( |00\rangle \)      & 50\%                \\ \hline
\( |11\rangle \)      & 50\%                \\ \hline
\end{tabular}
\end{table}

Measurement is an irreversible process and plays a critical role in quantum algorithms as well as in quantum automata theory, where it provides the means to extract classical information from quantum computations.

Mixed states, which describe statistical ensembles of quantum states, are represented by density matrices:
\[
\rho = \sum_i p_i |\psi_i\rangle\langle\psi_i|,
\]
with \(p_i \ge 0\) and \(\sum_i p_i = 1\). This formalism is essential when considering open systems subject to decoherence.
