
\subsection{Measurement and Probabilistic Outcomes}
\label{subsec:measurement}
\begin{definition}[Projective Measurement]
    When a quantum system in state
    \[
    |\psi\rangle = \sum_i \alpha_i |i\rangle
    \]
    is measured in an orthonormal basis \(\{|i\rangle\}\), the \textbf{Born rule} states that the outcome corresponding to \(|i\rangle\) is observed with probability
    \[
    P(i) = |\alpha_i|^2.
    \]
    Such a measurement is called a \emph{projective measurement}, after which the state collapses to the observed eigenstate \cite{nielsen2010quantum}.
\end{definition}

\begin{remark}
Projective measurements are irreversible and disturb the quantum state. This irreversibility is central to quantum algorithms and quantum automata, where measurement is the mechanism for extracting classical information.
\end{remark}

\begin{definition}[POVM Measurement]
    A more general framework is provided by Positive Operator-Valued Measures (POVMs). In a POVM, each measurement outcome \(i\) is associated with a positive operator \(E_i\) satisfying \(\sum_i E_i = I\). The probability of outcome \(i\) is \(\langle \psi | E_i | \psi \rangle\) \cite{nielsen2010quantum}.
\end{definition}

\begin{example}[Measurement of a Bell State]
Consider the Bell state
\[
|\Phi^+\rangle = \frac{|00\rangle + |11\rangle}{\sqrt{2}}.
\]
Measuring this state in the computational basis yields the outcomes \(|00\rangle\) and \(|11\rangle\) with probability 50\% each (see Table~\ref{tab:bell_measurement}).
\end{example}

\begin{table}[h]
\centering
\caption{Measurement outcomes for \(|\Phi^+\rangle\).}
\label{tab:bell_measurement}
\begin{tabular}{|c|c|}
\hline
\textbf{Outcome} & \textbf{Probability} \\ \hline
\( |00\rangle \) & 50\% \\ \hline
\( |11\rangle \) & 50\% \\ \hline
\end{tabular}
\end{table}

\begin{observation}
Measurement is a critical process in quantum computation and quantum automata theory as it converts quantum information into classical data.
\end{observation}

\begin{definition}[Mixed State]
    A \emph{mixed state} describes a statistical ensemble of quantum states (pure or mixed) and is represented by a density matrix
    \[
    \rho = \sum_i p_i |\psi_i\rangle \langle\psi_i|,
    \]
    where \(p_i \ge 0\) and \(\sum_i p_i = 1\). This representation is essential for modeling open quantum systems affected by decoherence \cite{nielsen2010quantum}.
\end{definition}