
\subsection{Superposition and Entanglement}
\label{subsec:superposition}

\textbf{Superposition} is the principle that a quantum state can exist as a linear combination of basis states. This underlies the quantum parallelism exploited by quantum algorithms. For example, applying the Hadamard gate \(H\) to \(|0\rangle\) creates a uniform superposition:
\[
H|0\rangle = \frac{|0\rangle + |1\rangle}{\sqrt{2}}, \quad H|1\rangle = \frac{|0\rangle - |1\rangle}{\sqrt{2}}.
\]

\textbf{Entanglement} is a uniquely quantum correlation between subsystems. An entangled state is one that cannot be factored into a product of individual states. The \textbf{Bell states} are classic examples of maximally entangled two-qubit states:
\[
|\Phi^+\rangle = \frac{|00\rangle + |11\rangle}{\sqrt{2}}, \quad
|\Phi^-\rangle = \frac{|00\rangle - |11\rangle}{\sqrt{2}},
\]
\[
|\Psi^+\rangle = \frac{|01\rangle + |10\rangle}{\sqrt{2}}, \quad
|\Psi^-\rangle = \frac{|01\rangle - |10\rangle}{\sqrt{2}}.
\]
Beyond Bell states, multipartite entangled states such as the Greenberger–Horne–Zeilinger (GHZ) state
\[
|\text{GHZ}\rangle = \frac{|000\rangle + |111\rangle}{\sqrt{2}}
\]
and the W state
\[
|W\rangle = \frac{|001\rangle + |010\rangle + |100\rangle}{\sqrt{3}}
\]
play critical roles in quantum communication and error correction.

Entanglement is essential for phenomena like quantum teleportation \cite{bennett1993teleporting}, superdense coding, and it contributes to the potential exponential speedup in algorithms such as Shor's factorization algorithm \cite{shor1999polynomial}.
