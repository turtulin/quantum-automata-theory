%%%%%%%%%%%%%%%%%%%%%%%%%%%%%%%%%%%%%%%%%%%%%%%%%%%%%%%%%%%%%%%
% Two-Way Finite Automata Variants
%%%%%%%%%%%%%%%%%%%%%%%%%%%%%%%%%%%%%%%%%%%%%%%%%%%%%%%%%%%%%%%

\subsection{Two-Way Finite Automata Variants}
\label{subsec:two-way-variants}

Two-way finite automata extend the classical one‐way model by allowing the read head to move in both directions over the input. Although this extra power does not increase the class of recognizable languages (both one‐way and two‐way automata recognize exactly the regular languages), two‐way models can be exponentially more succinct than one‐way models and naturally lend themselves to algorithms in several contexts (e.g., in complexity analysis and even quantum models).

%%%%%%%%%%%%%%%%%%%%%%%%%%%%%%%%%%%%%%%%%%%%%%%%%%%%%%%%%%%%%%%
% Two-Way Deterministic Finite Automata (2DFA)
%%%%%%%%%%%%%%%%%%%%%%%%%%%%%%%%%%%%%%%%%%%%%%%%%%%%%%%%%%%%%%%

\subsubsection{\acrfull{2dfa}}
\label{subsubsec:2dfa}

\begin{definition}[Two-Way Deterministic Finite Automaton]
A \gls{2dfa} is formally defined as an 8-tuple 
\[
M = (Q, \Sigma, L, R, \delta, s, t, r),
\]
where:
\begin{itemize}
  \item \(Q\) is a finite set of states,
  \item \(\Sigma\) is a finite input alphabet,
  \item \(L\) and \(R\) are special symbols called the left and right endmarkers, respectively (with \(L,R \notin \Sigma\)),
  \item \(\delta: Q \times (\Sigma \cup \{L, R\}) \to Q \times \{L, R\}\) is the transition function,
  \item \(s\in Q\) is the start state,
  \item \(t\in Q\) is the (unique) accept state, and
  \item \(r\in Q\) (with \(r\neq t\)) is the (unique) reject state.
\end{itemize}
In addition, the transition function is assumed to satisfy:
\begin{itemize}
  \item For every state \(q\in Q\), when reading the left endmarker \(L\), the head always moves to the right; that is, \(\delta(q,L) = (q', R)\) for some \(q'\in Q\).
  \item Similarly, when reading the right endmarker \(R\), the head always moves to the left: \(\delta(q,R) = (q', L)\).
  \item Once the machine reaches the accept state \(t\) (or the reject state \(r\)), it remains there (the transition always maps back to itself) while moving in a fixed direction.
\end{itemize}
\end{definition}

\begin{remark}
The two-way motion allows the automaton to perform multiple passes over the input, which can result in an exponential reduction in the number of states compared to one-way automata, though at the expense of increased operational complexity.
\end{remark}

\begin{example}[First and Last Symbol Equality]
Consider the language 
\[
L = \{ w\in \{0,1\}^* \mid \text{the first symbol of } w \text{ equals the last symbol} \}.
\]
A \gls{2dfa} for \(L\) operates as follows:
\begin{enumerate}
  \item Start at the left endmarker \(L\) and immediately move right to read the first symbol; store it in the control.
  \item Continue scanning right until the right endmarker \(R\) is reached.
  \item Upon reading \(R\), reverse direction (move left) and, in the process, skip any blank moves until the last symbol is reached.
  \item Compare the stored first symbol with the last symbol. If they are equal, move to the accept state \(t\); otherwise, move to the reject state \(r\).
\end{enumerate}
A simplified state diagram illustrating this process is provided in Figure~\ref{fig:2dfa-example}.
\end{example}

\begin{observation}
Although every \gls{2dfa} can be simulated by a one-way \gls{dfa}, such a simulation may require an exponential increase in the number of states.
\end{observation}

\paragraph{Operational Mechanics}
The two-way motion enables the automaton to make multiple passes over the input, which is particularly useful for verifying properties that depend on both the prefix and the suffix of the input string.

\paragraph{State Complexity and Conversion Algorithms}
For some families of regular languages, \glspl{2dfa} can be exponentially more succinct than their one-way counterparts. Conversion algorithms—such as those proposed by Shepherdson and Kozen—use crossing sequences to simulate two-way behavior in a one-way \gls{dfa}, typically at the cost of exponential state blow-up.

%%%%%%%%%%%%%%%%%%%%%%%%%%%%%%%%%%%%%%%%%%%%%%%%%%%%%%%%%%%%%%%
% Two-Way Nondeterministic Finite Automata (2NFA)
%%%%%%%%%%%%%%%%%%%%%%%%%%%%%%%%%%%%%%%%%%%%%%%%%%%%%%%%%%%%%%%

\subsubsection{\acrfull{2nfa}}
\label{subsubsec:2nfa}

\begin{definition}[Two-Way Nondeterministic Finite Automaton]
A \gls{2nfa} is defined similarly to a \gls{2dfa} but with a nondeterministic transition function. Formally, a \gls{2nfa} is an 8-tuple
\[
M = (Q, \Sigma, L, R, \delta, s, t, r),
\]
where:
\begin{itemize}
    \item \(Q\) is a finite set of states,
    \item \(\Sigma\) is a finite input alphabet,
    \item \(L\) and \(R\) are the left and right endmarkers (with \(L,R\notin\Sigma\)),
    \item \(\delta: Q \times (\Sigma \cup \{L,R\}) \to 2^{\,Q \times \{L,R\}}\) is the nondeterministic transition function,
    \item \(s\in Q\) is the start state,
    \item \(t\in Q\) is the unique accept state, and
    \item \(r\in Q\) (with \(r\neq t\)) is the unique reject state.
\end{itemize}
The transition function obeys similar boundary conditions as in the \gls{2dfa} case.
\end{definition}

\begin{remark}
The nondeterminism in a \gls{2nfa} allows it to "guess" important positions within the input and verify them via bidirectional traversal, which can lead to significant state savings compared to deterministic models.
\end{remark}

\begin{example}[Symmetry Check (Toy Version)]
Consider the language 
\[
L_{sym} = \{ w \in \{0,1\}^* \mid \text{the first two symbols equal the last two symbols} \}.
\]
A high-level description of a \gls{2nfa} for \(L_{sym}\) is:
\begin{enumerate}
    \item Scan right from the left endmarker \(L\) while nondeterministically guessing the point where comparison will occur.
    \item Upon reaching the right endmarker \(R\), reverse direction.
    \item While moving left, nondeterministically check that the stored first two symbols match the corresponding symbols at the end.
    \item If both comparisons succeed, transition to the accept state \(t\); otherwise, transition to the reject state \(r\).
\end{enumerate}
Figure~\ref{fig:2nfa-example} schematically illustrates this guess-and-check mechanism.
\end{example}

\begin{observation}
\glspl{2nfa} can be exponentially more succinct than one-way \glspl{dfa}, even though the class of languages they recognize remains the same (i.e., the regular languages).
\end{observation}

%%%%%%%%%%%%%%%%%%%%%%%%%%%%%%%%%%%%%%%%%%%%%%%%%%%%%%%%%%%%%%%
% Two-Way Probabilistic Finite Automata (2PFA)
%%%%%%%%%%%%%%%%%%%%%%%%%%%%%%%%%%%%%%%%%%%%%%%%%%%%%%%%%%%%%%%

\subsubsection{\acrfull{2pfa}}
\label{subsubsec:2pfa}

\begin{definition}[Two-Way Probabilistic Finite Automaton]
A \gls{2pfa} is an 8-tuple
\[
M = (Q, \Sigma, L, R, \delta, s, t, r),
\]
where:
\begin{itemize}
    \item \(Q\) is a finite set of states,
    \item \(\Sigma\) is a finite input alphabet,
    \item \(L\) and \(R\) are the left and right endmarkers (with \(L,R \notin \Sigma\)),
    \item \(\delta: Q \times (\Sigma \cup \{L,R\}) \to \mathbb{R}_{\ge 0}^{\,Q \times \{L,R\}}\) is a probabilistic transition function such that
    \[
    \sum_{(q',d)\in Q\times\{L,R\}} \delta(q,a,q',d) = 1 \quad \text{for all } q \in Q \text{ and } a \in \Sigma \cup \{L,R\},
    \]
    \item \(s\in Q\) is the start state,
    \item \(t\in Q\) is the unique accept state, and
    \item \(r\in Q\) (with \(r\neq t\)) is the unique reject state.
\end{itemize}
\end{definition}

\begin{remark}
A \gls{2pfa} extends the probabilistic finite automaton by allowing bidirectional head movement. Its transitions are governed by probability distributions, and acceptance is determined by whether the cumulative probability of reaching the accept state exceeds a predetermined cut-point.
\end{remark}

\begin{example}[Majority Language]
Consider the language 
\[
L_{maj} = \{ w \in \{a,b\}^* \mid \#a(w) > \#b(w) \}.
\]
A \gls{2pfa} for \(L_{maj}\) operates by making probabilistic passes over the input, updating state probabilities, and eventually halting in the accept state \(t\) if the acceptance probability is high enough. Figure~\ref{fig:2pfa-example} provides a schematic illustration of such a machine.
\end{example}

\begin{theorem}[Rabin's Theorem for \glspl{pfa}]
\label{thm:2pfa-rabin}
A \gls{pfa} with an isolated cut-point recognizes exactly the class of regular languages. This result extends to \glspl{2pfa} under analogous conditions.
\end{theorem}

\begin{proposition}
If a \gls{2pfa} employs a non-isolated cut-point (e.g., \(\lambda = 0\)), it may recognize languages beyond the regular class.
\end{proposition}

\begin{corollary}
For a \gls{2pfa} with a strict cut-point (\(\lambda = 1\)), the recognized language is equivalent to that of a \gls{dfa}.
\end{corollary}

%%%%%%%%%%%%%%%%%%%%%%%%%%%%%%%%%%%%%%%%%%%%%%%%%%%%%%%%%%%%%%%
% Comparative Analysis of Two-Way Models
%%%%%%%%%%%%%%%%%%%%%%%%%%%%%%%%%%%%%%%%%%%%%%%%%%%%%%%%%%%%%%%

\subsubsection{Comparative Analysis of Two-Way Models}
\label{subsubsec:two-way-comparison}

\begin{table}[h]
    \centering
    \begin{adjustbox}{max width=\textwidth}
    \begin{tabular}{|l|c|c|c|c|l|}
        \hline
        \textbf{Model} & \textbf{Language Class} & \textbf{Time Complexity} & \textbf{Space Complexity} & \textbf{State Complexity} & \textbf{Key Reference} \\
        \hline
        \gls{2dfa}  & REG & \(O(n^2)\) & \(O(1)\) & May be exponentially smaller than 1DFA & \cite{hopcroft2006introduction} \\
        \gls{2nfa}  & REG & \(O(n)\) & \(O(1)\) & Can be exponentially more succinct than 1DFA & \cite{yakaryilmaz2010succinctness} \\
        \gls{2pfa}  & \( \mbox{REG} \subset L \subseteq \mbox{P} \) & \(O(n^3)\) & \(O(\log n)\) & Varies with error bounds & \cite{freivalds1981probabilistic} \\
        \hline
    \end{tabular}
    \end{adjustbox}
    \caption{Comparison of classical two-way automata models (using explicit endmarkers).}
    \label{tab:two-way-comparison}
\end{table}

\begin{remark}
The comparative analysis illustrates that while all two-way automata recognize only regular languages, the two-way models often achieve significant advantages in state complexity and, in some cases, time complexity, compared to their one-way counterparts.
\end{remark}

\begin{center}
\begin{tikzpicture}[shorten >=1pt,node distance=2.5cm,on grid,auto]
  \node[state,initial,accepting] (q0) {\(q_0\)};
  \node[state] (q1) [right=of q0] {\(q_1\)};
  \node[state] (q2) [below=of q1] {\(q_2\)};
  \node[state,accepting] (t) [left=of q2] {\(t\)};
  \node[state,rejecting] (r) [below=of q0] {\(r\)};

  \path[->]
    (q0) edge node {read first symbol, store \(a\)} (q1)
    (q1) edge [bend left] node {scan right until \(R\)} (q2)
    (q2) edge [bend left] node {reverse; on \(R\) switch to leftward} (q1)
    (q2) edge node {at last symbol, compare with \(a\)} (t)
    (q2) edge [bend right] node [swap] {if mismatch} (r);
\end{tikzpicture}
\end{center}

\begin{figure}[h]
    \centering  
    \begin{tikzpicture}[shorten >=1pt, node distance=2cm, on grid, auto]
        \node[state, initial] (q0) {\(q_0\)};
        \node[state] (q1) [right=of q0] {\(q_1\)};
        \node[state, accepting] (q2) [right=of q1] {\(q_2\)};
        \node[state, rejecting] (qr) [below=of q1] {\(r\)};
        \path[->]
            (q0) edge [loop above] node {\(0,1,R\)} (q0)
            (q0) edge node {\(\triangleright\)} (q1)
            (q1) edge [loop above] node {\(0,1,L\)} (q1)
            (q1) edge [bend left] node {\(0/1\)} (q2)
            (q1) edge [bend right] node[swap] {\(0/1\)} (qr);
    \end{tikzpicture}
    \caption{Schematic 2NFA illustrating a guess-and-check mechanism (for a toy symmetry language).}
    \label{fig:2nfa-example}
\end{figure}

\begin{figure}[h]
    \centering  
    \begin{tikzpicture}[shorten >=1pt, node distance=2.5cm, on grid, auto]
        \node[state, initial] (q0) {\(q_0\)};
        \node[state] (q1) [right=of q0] {\(q_1\)};
        \node[state, accepting] (q2) [right=of q1] {\(q_2\)};
        \path[->]
            (q0) edge [loop above] node[align=center] {\(a:0.6R\)\\\(b:0.4R\)} (q0)
            (q0) edge node {\(\triangleright:0.5\)} (q1)
            (q1) edge [loop above] node[align=center] {\(a:0.3L\)\\\(b:0.7L\)} (q1)
            (q1) edge node {\(\triangleleft:0.7\)} (q2);
    \end{tikzpicture}
    \caption{Schematic 2PFA for a majority language (with sample probabilistic transitions).}
    \label{fig:2pfa-example}
\end{figure}
