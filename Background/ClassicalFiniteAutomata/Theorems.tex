
% \subsection{Key Theorems}
% \label{subsec:key-theorems} 

% \begin{enumerate}
%     \item \textit{Kleene's Theorem}: A language is regular if and only if it is recognized by a DFA/NFA or described by a regular expression \cite{hopcroft2006introduction}.
%     \item \textit{Subset Construction Theorem}: Every NFA can be converted to an equivalent DFA, with up to $2^n$ states \cite{hopcroft2006introduction}.
%     \item \textit{Myhill-Nerode Theorem}: Characterizes $\text{REG}$ via string indistinguishability, forming the basis for DFA minimization \cite{hopcroft2006introduction}.
%     \item \textit{Rabin's Theorem}: PFAs recognize stochastic languages, a strict superset of $\text{REG}$ \cite{rabin1963probabilistic}.
%     \item \textit{Sipser's Theorem}: 2PFAs recognize $\text{REG}$ in logarithmic space but require exponential time for non-regular languages \cite{sipser1980halting}.
% \end{enumerate} 

% These theorems collectively delineate the boundaries of classical finite automata, setting the stage for quantum extensions in subsequent chapters. 
