
\subsection{\acrfull{nfa}}
\label{subsec:nfa}

\begin{definition}[\gls{nfa}]
A \gls{nfa} is a quintuple 
\[
M = (Q, \Sigma, \delta, q_0, F)
\]
where:
\begin{itemize}
    \item \( Q \) is a finite set of states,
    \item \( \Sigma \) is an input alphabet,
    \item \( \delta: Q \times (\Sigma \cup \{\epsilon\}) \rightarrow 2^Q \) is a nondeterministic transition function,
    \item \( q_0 \in Q \) is the initial state, and
    \item \( F \subseteq Q \) is the set of accepting states.
\end{itemize}
\end{definition}

\begin{remark}
Unlike \glspl{dfa}, a \gls{nfa} may have multiple transitions for a given state and input symbol, including transitions on the empty string \(\epsilon\). This nondeterminism allows for multiple computational paths.
\end{remark}

\begin{example}
Figure~\ref{fig:nfa-example} depicts a \gls{nfa} that recognises the language 
\[
L = \{ w \in \{a,b\}^* \mid w \text{ contains the substring } ab \}.
\]
\end{example}

\begin{algorithm}[Subset Construction for \glspl{nfa}]
\label{alg:subset}
To convert an \gls{nfa} \( N = (Q, \Sigma, \delta, q_0, F) \) into an equivalent \gls{dfa}:
\begin{enumerate}
    \item Compute the \(\epsilon\)-closure of the initial state: \( S_0 = \epsilon\text{-closure}(\{q_0\}) \).
    \item For each \gls{dfa} state \( S \subseteq Q \) and each input symbol \(\sigma \in \Sigma\), define 
    \[
    \delta_{\text{\gls{dfa}}}(S, \sigma) = \epsilon\text{-closure}\Big(\bigcup_{q \in S} \delta(q, \sigma)\Big).
    \]
    \item Mark \( S \) as accepting if \( S \cap F \neq \emptyset \).
    \item Repeat until no new states are produced.
\end{enumerate}
\end{algorithm}

\begin{observation}
The subset construction algorithm may produce up to \(2^{|Q|}\) states in the worst case, illustrating a potential state explosion when converting an \gls{nfa} to a \gls{dfa}.
\end{observation}

\begin{figure}[h]
    \centering  
    \begin{tikzpicture}[shorten >=1pt, node distance=2.5cm, on grid, auto]
        \node[state, initial, initial text={}] (q0) {$q_0$};
        \node[state] (q1) [right=of q0] {$q_1$};
        \node[state, accepting] (q2) [right=of q1] {$q_2$};

        \path[->]
        (q0) edge [loop above] node {$a,b$} (q0)
        (q0) edge node {$a$} (q1)
        (q1) edge node {$b$} (q2);
    \end{tikzpicture}
    \caption{NFA recognizing \( L = \Sigma^*ab \)}
    \label{fig:nfa-example}
\end{figure}

\begin{figure}[h]
    \centering  
    \begin{tikzpicture}[shorten >=1pt, node distance=3cm, on grid, auto]
        \node[state, initial, initial text={}] (A) {$\{q_0\}$};
        \node[state] (B) [right=of A] {$\{q_0, q_1\}$};
        \node[state, accepting] (C) [right=of B] {$\{q_0, q_2\}$};

        \path[->]
        (A) edge [loop below] node {$b$} (A)
        (A) edge node {$a$} (B)
        (B) edge [loop below] node {$a$} (B)
        (B) edge node {$b$} (C)
        (C) edge [bend left] node {$a$} (B)
        (C) edge [bend right] node[swap] {$b$} (A);
    \end{tikzpicture}
    \caption{Equivalent DFA for NFA in Figure~\ref{fig:nfa-example} \cite{hopcroft2006introduction}}
    \label{fig:dfa-conversion}
\end{figure}