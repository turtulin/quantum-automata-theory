\subsection{\glsentrylong{dfa}}
\label{subsec:dfa}

\begin{definition}[\gls{dfa}]
A \gls{dfa} is a quintuple 
\[
M = (Q, \Sigma, \delta, q_0, F)
\]
where:
\begin{itemize}
    \item \( Q \) is a finite set of states,
    \item \( \Sigma \) is a finite input alphabet,
    \item \( \delta: Q \times \Sigma \rightarrow Q \) is a total transition function (defined for all \( (q, \sigma) \in Q \times \Sigma \)),
    \item \( q_0 \in Q \) is the initial state,
    \item \( F \subseteq Q \) is the set of accept states.
\end{itemize}
\end{definition}

\begin{theorem}[Myhill-Nerode Theorem]
    \label{thm:myhill-nerode}
    A language \( L \subseteq \Sigma^* \) is regular if and only if the equivalence relation \( \sim_L \), defined by
    \[
    x \sim_L y \iff \forall z \in \Sigma^*,\, xz \in L \Longleftrightarrow yz \in L,
    \]
    has finitely many equivalence classes. The number of equivalence classes equals the number of states in the minimal \gls{dfa} for \( L \) \cite{sipser2013introduction, myhill1957finite, nerode1958linear}.
\end{theorem}

\begin{example}
Consider the language 
\[
L = \{ w \in \{0,1\}^* \mid \text{the number of 0's and 1's in \( w \) are both even} \}.
\]
Figure~\ref{fig:dfa-example} shows a minimal \gls{dfa} for \( L \). Each state encodes a parity pair \((p_0, p_1)\), where \( p_0 \) and \( p_1 \) represent the parity (even/odd) of 0's and 1's, respectively.
\end{example}

\begin{figure}[ht]
    \centering  
    \begin{tikzpicture}[shorten >=1pt, node distance=3cm, on grid, auto]
        \node[state, initial, accepting, initial text={}] (q00) {$(0,0)$};
        \node[state] (q01) [right=of q00] {$(0,1)$};
        \node[state] (q10) [below=of q01] {$(1,0)$};
        \node[state] (q11) [left=of q10] {$(1,1)$};

        \path[->]
        (q00) edge [bend left=15] node {1} (q01)
              edge [bend right=15] node[swap] {0} (q10)
        (q01) edge [bend left=15] node {0} (q11)
              edge [bend right=15] node[swap] {1} (q00)
        (q10) edge [bend left=15] node {1} (q11)
              edge [bend right=15] node[swap] {0} (q00)
        (q11) edge [bend left=15] node {0} (q01)
              edge [bend right=15] node[swap] {1} (q10);
    \end{tikzpicture}
    \caption{Minimal \gls{dfa} for the language of strings with even numbers of 0's and 1's. States are labeled as (parity of 0's, parity of 1's).}
    \label{fig:dfa-example}
\end{figure}

\begin{observation}
The Myhill-Nerode Theorem~\ref{thm:myhill-nerode} explains why this \gls{dfa} is minimal: the four states correspond to the four equivalence classes of \( \sim_L \). No smaller \gls{dfa} can recognise \( L \) \cite{sipser2013introduction}.
\end{observation}