
\subsection{Deterministic Finite Automata (\gls{dfa})}
\label{subsec:dfa}

\begin{definition}[Deterministic Finite Automaton]
A \gls{dfa} is a quintuple 
\[
M = (Q, \Sigma, \delta, q_0, F)
\]
where:
\begin{itemize}
    \item \( Q \) is a finite set of states,
    \item \( \Sigma \) is a finite input alphabet,
    \item \( \delta: Q \times \Sigma \rightarrow Q \) is a total transition function,
    \item \( q_0 \in Q \) is the unique initial state, and
    \item \( F \subseteq Q \) is the set of accepting states.
\end{itemize}
\end{definition}

\begin{remark}
A \gls{dfa} has the property that for every state \( q \in Q \) and every input symbol \( \sigma \in \Sigma \), there is exactly one transition defined, ensuring deterministic behavior.
\end{remark}

\begin{theorem}[Myhill-Nerode Theorem]
\label{thm:myhill-nerode}
A language \( L \subseteq \Sigma^* \) is regular if and only if the equivalence relation defined by 
\[
x \sim_L y \iff \forall z \in \Sigma^*,\, xz \in L \Longleftrightarrow yz \in L
\]
has finitely many equivalence classes. Consequently, every regular language has a unique minimal \gls{dfa} up to isomorphism.
\end{theorem}

\begin{example}
Consider the language 
\[
L = \{ w \in \{0,1\}^* \mid \text{the number of \(0\)'s and \(1\)'s in \(w\) are both even} \}.
\]
Figure~\ref{fig:dfa-example} shows a \gls{dfa} recognizing \( L \).
\end{example}

\begin{figure}[h]
    \centering  
    \begin{tikzpicture}[shorten >=1pt, node distance=3cm, on grid, auto]
        \node[state, initial, accepting, initial text={}] (q0) {$q_0$};
        \node[state] (q1) [right=of q0] {$q_1$};
        \node[state] (q2) [below=of q1] {$q_2$};
        \node[state] (q3) [left=of q2] {$q_3$};

        \path[->]
        (q0) edge [bend left] node {1} (q1)
        (q0) edge [bend right] node[swap] {0} (q3)
        (q1) edge [bend right] node[swap] {0} (q2)
        (q1) edge [bend left] node {1} (q0)
        (q2) edge [bend left] node {1} (q3)
        (q2) edge [bend right] node[swap] {0} (q1)
        (q3) edge [bend right] node[swap] {0} (q0)
        (q3) edge [bend left] node {1} (q2);
    \end{tikzpicture}
    \caption{DFA recognizing even numbers of \texttt{0}s and \texttt{1}s.}
    \label{fig:dfa-example}
\end{figure}

\begin{observation}
    Graphical representations (see Figure~\ref{fig:dfa-example}) provide an intuitive view of how the \gls{dfa} tracks the parity of \(0\)'s and \(1\)'s, reinforcing its deterministic nature.
\end{observation}