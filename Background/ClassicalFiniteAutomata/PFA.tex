\subsection{\acrfull{pfa}}
\label{subsec:pfa}

\begin{definition}[\gls{pfa}]
    A \gls{pfa} is a quintuple 
    \[
    M = (Q, \Sigma, \delta, \pi, F)
    \]
    where:
    \begin{itemize}
        \item \( Q \) is a finite set of states,
        \item \( \Sigma \) is a finite input alphabet,
        \item \( \delta: Q \times \Sigma \times Q \rightarrow [0,1] \) is a probabilistic transition function such that 
        \[
        \sum_{q' \in Q} \delta(q, \sigma, q') = 1 \quad \text{for all } q \in Q \text{ and } \sigma \in \Sigma
        \]\cite{rabin1963probabilistic},
        \item \( \pi \in \mathbb{R}^{|Q|} \) is an initial state distribution vector with 
        \[
        \sum_{q \in Q} \pi_q = 1,
        \]
        \item \( F \subseteq Q \) is the set of accepting states.
    \end{itemize}
\end{definition}

\begin{remark}
In a \gls{pfa}, transitions are probabilistic. The acceptance of an input string is determined by whether the cumulative probability of ending in an accepting state exceeds a chosen cut-point.
\end{remark}

\begin{example}
Figure~\ref{fig:pfa-example} illustrates a \gls{pfa} that recognises the language 
\[
L_{\text{maj}} = \{ w \in \{a,b\}^* \mid |w|_a > |w|_b \},
\]
where the acceptance probability is at least \( \frac{2}{3} \).
\end{example}

\begin{theorem}[Rabin's Theorem for \glspl{pfa}]
    \label{thm:rabin}
    A \gls{pfa} with an isolated cut-point recognises exactly the class of regular languages \cite{rabin1963probabilistic}.
\end{theorem}

\begin{proposition}
    If a \gls{pfa} employs a non-isolated cut-point (e.g., \(\lambda = 0\)), it may recognise languages beyond the regular class, including some context-sensitive languages \cite{paz1971introduction}.
\end{proposition}

\begin{corollary}
For a \gls{pfa} with a strict cut-point (\(\lambda = 1\)), the recognised language is equivalent to that of a \gls{dfa}.
\end{corollary}

\begin{observation}
    The closure properties of \glspl{pfa} differ from those of classical finite automata. In particular, complementation is not directly achievable unless an isolated cut-point is used \cite{droste2009handbook}.
\end{observation}

\begin{figure}[h]
    \centering  
    \begin{tikzpicture}[shorten >=1pt, node distance=3cm, on grid, auto]
        \node[state, initial, initial text={$\pi=1$}] (q0) {$q_0$};
        \node[state, accepting] (q1) [right=of q0] {$q_1$};
        
        \path[->]
        (q0) edge [loop above] node {a(0.6), b(0.4)} (q0)
        (q0) edge [bend left] node {a(0.4), b(0.6)} (q1)
        (q1) edge [loop above] node {a(0.3), b(0.7)} (q1)
        (q1) edge [bend left] node {a(0.7), b(0.3)} (q0);
    \end{tikzpicture}
    \caption{PFA for majority language with probabilistic transitions}
    \label{fig:pfa-example}
\end{figure}
