\section{Detailed Models of Quantum Finite Automata}
\label{sec:detailed-models}
Among the numerous variants of \glspl{qfa}, the measure-once and measure-many models stand out as two of the most foundational and widely studied frameworks. Respectively known as \gls{mo-1qfa} and \gls{mm-1qfa}, these automata provide a minimalistic yet powerful theoretical playground for investigating the principles of quantum computation with finite memory. Their significance lies not only in their historical development as some of the earliest quantum models for language recognition \cite{moore2000quantum, kondacs1997power}, but also in their role as archetypal examples in the study of quantum-classical computational boundaries.

A \gls{mo-1qfa} performs unitary operations throughout the input reading process and conducts a single measurement only at the end of the computation. This model, introduced by Moore and Crutchfield \cite{moore2000quantum}, can recognize only a restricted class of \glspl{reg}, such as the so-called group languages \cite{brodsky2002characterizations}. In contrast, the \gls{mm-1qfa}, introduced by Kondacs and Watrous \cite{kondacs1997power}, allows measurements after each symbol is processed, enabling it to recognize a strictly larger class of \glspl{reg}, though still not the full class.

The primary importance of \gls{mo-1qfa} and \gls{mm-1qfa} is not just theoretical. These models are particularly suitable for demonstrating techniques for the compilation of quantum automata into quantum circuits, as we will explore in Chapter ~\ref{chap:automata-to-circuits}. Due to their simpler architecture—one-way movement, discrete time steps, and finite-dimensional Hilbert spaces—these automata provide an ideal framework for illustrating how abstract automaton transitions can be implemented using quantum gates and projective measurements. In this thesis, they will serve as canonical models to illustrate the compilation strategy, setting a baseline for comparison with more advanced or generalized \gls{qfa} variants.


\subsection{\glsentrylong{mo-1qfa}}
\label{sec:moqfa}

\subsubsection{Introduction}
\glspl{mo-1qfa} represent one of the simplest models of quantum computation in the realm of automata theory. Introduced by Moore and Crutchfield in 2000~\cite{moore2000quantum}, \glspl{mo-1qfa} evolve solely through unitary transformations corresponding to the input symbols and perform a single measurement at the end of the computation. This model has been further characterized by Brodsky and Pippenger~\cite{brodsky2002characterizations}, and it is known for its conceptual simplicity as well as for its limitations. Notably, when restricted to bounded error, \glspl{mo-1qfa} recognize exactly the class of group languages—a proper subset of the regular languages. In contrast, the measure-many variant ~\cite{kondacs1997power} employs intermediate measurements and exhibits different acceptance capabilities.

\subsubsection{Formal Definition}
An \gls{mo-1qfa} is formally defined as a 5-tuple 
\[
M = (Q,\Sigma,\delta,q_0,F),
\]
where:
\begin{itemize}
    \item $Q$ is a finite set of states,
    \item $\Sigma$ is a finite input alphabet, typically augmented with a designated end-marker (e.g., \$),
    \item $\delta : Q \times \Sigma \times Q \to \mathbb{C}$ is the transition function such that, for every symbol $\sigma\in\Sigma$, the matrix 
    \[
    U_\sigma,\quad \text{with} \quad (U_\sigma)_{q,q'} = \delta(q,\sigma,q'),
    \]
    is unitary~\cite{moore2000quantum},
    \item $q_0 \in Q$ is the initial state, and
    \item $F\subseteq Q$ is the set of accepting states.
\end{itemize}
For an input string $x=x_1x_2\cdots x_n$, the computation proceeds by applying the corresponding unitary matrices sequentially:
\[
|\Psi_x\rangle = U_{x_n}U_{x_{n-1}}\cdots U_{x_1}|q_0\rangle.
\]
After reading the entire input, a measurement is performed using the projection operator
\[
P=\sum_{q\in F} |q\rangle\langle q|,
\]
so that the acceptance probability is defined as
\[
p_M(x)=\|P\,|\Psi_x\rangle\|^2.
\]
Alternative characterizations, including formulations using the Heisenberg picture, have been discussed in~\cite{qiu2004characterizations,piazza2022mirrors}.

\subsubsection{Strings Acceptance}
A string $x$ is accepted by an \gls{mo-1qfa} if the acceptance probability $p_M(x)$ exceeds a predetermined cut-point $\lambda$. In a bounded error setting, there exists a margin $\epsilon > 0$ such that:
\[
\forall x\in L:\quad p_M(x) \ge \lambda + \epsilon,
\]
\[
\forall x\notin L:\quad p_M(x) \le \lambda - \epsilon.
\]
Under the unbounded error regime, \glspl{mo-1qfa} can accept some nonregular languages (for instance, solving the word problem over the free group)~\cite{brodsky2002characterizations}. The precise acceptance behavior thus depends on whether a cut-point or a bounded error framework is adopted.

\subsubsection{Set of Languages Accepted}
When \glspl{mo-1qfa} are restricted to bounded error acceptance, they recognize exactly the class of \emph{group languages}—languages whose syntactic semigroups form groups~\cite{brodsky2002characterizations}. This class forms a strict subset of the regular languages, emphasizing the inherent limitation of the \gls{mo-1qfa} model. In contrast, by relaxing the error bounds, one may design \glspl{mo-1qfa} that accept a broader range of languages, albeit often at the cost of increased computational complexity.

\subsubsection{Closure Properties}
The class of languages accepted by \glspl{mo-1qfa} under bounded error exhibits robust closure properties. Specifically, this class is closed under:
\begin{itemize}
    \item Inverse Homomorphisms~\cite{brodsky2002characterizations},
    \item Word Quotients~\cite{brodsky2002characterizations},
    \item Boolean Operations (union, intersection, and complement)~\cite{freivalds2005languages,bertoni2003quantum}.
\end{itemize}
Additional algebraic properties, including aspects related to the pumping lemma and the structure of the accepting probabilities, have been further elaborated in works such as Ambainis et al.~\cite{ambainis1999probabilities} and Xi et al.~\cite{xi2008some}.

\subsubsection{Summary of Advantages and Limitations}
\glspl{mo-1qfa} are praised for their simplicity. Since the quantum state evolves unitarily until a single measurement is made, the model avoids the complications associated with intermediate state collapses. This simplicity has practical implications; for example, recent experimental work has shown that custom control pulses can significantly reduce error rates in IBM-Q implementations~\cite{lussi2024implementing}, and photonic implementations have further demonstrated the feasibility of \glspl{mo-1qfa} in optical setups~\cite{candeloro2021enhanced}. On the downside, the acceptance power of \glspl{mo-1qfa} is limited—when operating under bounded error, they recognize only the group languages, which form a strict subset of the regular languages. In contrast, \glspl{mm-1qfa} offer greater acceptance power but at the cost of increased model complexity~\cite{kondacs1997power,berzicna2001ambainis}.

\subsubsection{Example}
Consider a simple \gls{mo-1qfa} defined over the unary alphabet $\Sigma=\{a\}$ with state set $Q=\{q_0,q_1\}$, initial state $q_0$, and accepting state set $F=\{q_1\}$. Let the unitary operator corresponding to the symbol $a$ be defined by the rotation matrix:
\[
U_a = \begin{pmatrix}
\cos\theta & -\sin\theta \\
\sin\theta & \cos\theta
\end{pmatrix},
\]
for a fixed angle $\theta$. For an input string $a^n$, the state evolves as:
\[
|\Psi_{a^n}\rangle = U_a^n |q_0\rangle.
\]
The acceptance probability is computed as:
\[
p_M(a^n)=\|P\,|\Psi_{a^n}\rangle\|^2,
\]
where the projection operator $P$ is given by $P=|q_1\rangle\langle q_1|$. By appropriately choosing $\theta$, the automaton can be tuned so that $p_M(a^n)$ exceeds the cut-point $\lambda$ (for example, $\lambda=\frac{1}{2}$) if and only if $a^n$ belongs to the target language. This example illustrates the essential mechanism of \glspl{mo-1qfa}, as described in~\cite{moore2000quantum,brodsky2002characterizations}.

\subsubsection{Additional Topics}
\paragraph{Learning and Optimization:} Recent work by Chu et al.~\cite{chu2023approximately} has introduced methods that combine active learning with non-linear optimization to approximately learn the parameters of \glspl{mo-1qfa}. These techniques provide insights into how one can recover the unitary transformations and state structure from observed data.

\paragraph{Complexity and Minimization:} The problem of minimizing the number of states in an \gls{mo-1qfa} was originally posed by Moore and Crutchfield~\cite{moore2000quantum}. Subsequent work by Mateus, Qiu, and Li~\cite{mateus2012complexity} has established an EXPSPACE upper bound for the minimization problem, framing it as a challenge in solving systems of algebraic polynomial (in)equations.

\paragraph{Experimental Implementations:} Experimental realizations of \glspl{mo-1qfa} have also been explored. Lussi et al.~\cite{lussi2024implementing} demonstrated an implementation on IBM-Q devices using custom control pulses, while photonic approaches have been reported in~\cite{candeloro2021enhanced}. These works highlight both the practical challenges and the potential advantages of implementing \glspl{mo-1qfa} on current quantum hardware.

\paragraph{Future Directions:} Future research may focus on further enhancing experimental implementations, developing more robust learning algorithms for \glspl{mo-1qfa}, and exploring new minimization techniques that could lead to more efficient automata. Extensions that combine features of \glspl{mo-1qfa} and \gls{mm-1qfa}s may also provide richer language recognition capabilities and deepen our understanding of quantum computational models.


\subsection{\glsentrylong{mm-1qfa}}
\label{sec:mmqfa}

\subsubsection{Introduction}
\glspl{mm-1qfa} are a variant of quantum finite automata in which a measurement is performed after reading each input symbol. Introduced by Kondacs and Watrous in 1997~\cite{kondacs1997power}, \glspl{mm-1qfa} allow the automaton to collapse its quantum state at intermediate steps, thereby potentially influencing the computation dynamically. Although this mechanism can enhance the detection of accepting or rejecting conditions during the run, under the bounded error regime \glspl{mm-1qfa} are known to recognize only a proper subset of the regular languages~\cite{brodsky2002characterizations}. Recent work, such as by Lin~\cite{lin2012another}, has provided elegant methods for addressing the equivalence problem of \glspl{mm-1qfa}, further enriching our understanding of their computational properties.

\subsubsection{Formal Definition}
A \glsentryfull{mm-1qfa} is defined as a 6-tuple
\[
M = (Q,\Sigma,\delta,q_0,Q_{acc},Q_{rej}),
\]
where:
\begin{itemize}
    \item $Q$ is a finite set of states,
    \item $\Sigma$ is a finite input alphabet, typically augmented with an end-marker (e.g., \$),
    \item $\delta : Q \times \Sigma \times Q \to \mathbb{C}$ is the transition function, where for each symbol $\sigma\in\Sigma$ the corresponding matrix 
    \[
    U_\sigma,\quad \text{with} \quad (U_\sigma)_{q,q'}=\delta(q,\sigma,q'),
    \]
    is unitary~\cite{kondacs1997power},
    \item $q_0 \in Q$ is the initial state,
    \item $Q_{acc} \subseteq Q$ is the set of accepting (halting) states, and
    \item $Q_{rej} \subseteq Q$ is the set of rejecting (halting) states.
\end{itemize}
After each symbol is read, the automaton's current state is measured with respect to the decomposition
\[
E_{acc} = \text{span}\{|q\rangle : q \in Q_{acc}\},\quad
E_{rej} = \text{span}\{|q\rangle : q \in Q_{rej}\},\quad
E_{non} = \text{span}\{|q\rangle : q \in Q \setminus (Q_{acc}\cup Q_{rej})\}.
\]
If the measurement outcome lies in $E_{acc}$ or $E_{rej}$, the computation halts immediately with acceptance or rejection, respectively. This definition, adapted from Kondacs and Watrous~\cite{kondacs1997power} and refined in Lin~\cite{lin2012another}, forms the basis of the \gls{mm-1qfa} model.

\subsubsection{Strings Acceptance}
For an input string $x=x_1x_2\cdots x_n$, the \gls{mm-1qfa} processes each symbol sequentially. At each step $i$, the unitary operator $U_{x_i}$ is applied, followed by a measurement:
\begin{itemize}
    \item If the measurement result falls in $E_{acc}$, the automaton immediately accepts $x$.
    \item If it falls in $E_{rej}$, the automaton rejects $x$.
    \item If the result lies in $E_{non}$, the computation continues with the next symbol.
\end{itemize}
The overall acceptance probability of $x$ is the cumulative probability of all computation paths that eventually lead to an accepting state. In a bounded error framework, there exists a margin $\epsilon > 0$ such that for every $x\in L$, the acceptance probability satisfies
\[
p_M(x) \ge \lambda + \epsilon,
\]
and for every $x\notin L$, 
\[
p_M(x) \le \lambda - \epsilon,
\]
where $\lambda$ is a predetermined cut-point (commonly set to $\frac{1}{2}$)~\cite{kondacs1997power,brodsky2002characterizations}.

\subsubsection{Set of Languages Accepted}
Under the bounded error constraint, \glspl{mm-1qfa} recognize a proper subset of the regular languages. In particular, the languages accepted by \glspl{mm-1qfa} must satisfy specific algebraic properties that restrict their expressive power. Although \glspl{mm-1qfa} can, in some cases, recognize nonregular languages when allowed unbounded error, the bounded error condition confines them to a class that is comparable to that of group languages~\cite{brodsky2002characterizations,kondacs1997power}. This limitation underscores the trade-off between the increased measurement frequency and the resultant reduction in language recognition capability.

\subsubsection{Closure Properties}
The language class recognized by \glspl{mm-1qfa} with bounded error is known to enjoy several closure properties:
\begin{itemize}
    \item It is closed under complement and inverse homomorphisms~\cite{brodsky2002characterizations}.
    \item It is closed under word quotients~\cite{brodsky2002characterizations}.
    \item However, the class is not closed under arbitrary homomorphisms~\cite{kondacs1997power,bertoni2003quantum}.
\end{itemize}
Recent work by Lin~\cite{lin2012another} further refines our understanding of these closure properties by addressing the equivalence problem for \glspl{mm-1qfa}, thereby linking the structural properties of the recognized languages to the underlying automata.

\subsubsection{Summary of Advantages and Limitations}
\glspl{mm-1qfa} offer notable advantages:
\begin{itemize}
    \item The use of intermediate measurements can enable earlier detection of acceptance or rejection, potentially reducing the average computation time.
    \item The dynamic collapse of the quantum state provides a different balance between quantum coherence and classical decision-making.
\end{itemize}
Nevertheless, there are significant limitations:
\begin{itemize}
    \item The frequent measurements interrupt the quantum evolution, which can limit the automaton’s ability to harness quantum interference effectively.
    \item As a result, under bounded error conditions, \glspl{mm-1qfa} recognize only a restricted subset of the regular languages.
    \item The complexity of analyzing and minimizing \glspl{mm-1qfa} remains high, with state minimization posing an EXPSPACE challenge~\cite{mateus2012complexity} and lower bound results highlighting the inherent state complexity~\cite{ablayev2000lower}.
\end{itemize}
Moreover, when compared to \gls{mo-1qfa}s, \glspl{mm-1qfa} may offer greater recognition power in some unbounded error scenarios but at the cost of increased computational and implementation complexity~\cite{kondacs1997power,berzicna2001ambainis}.

\subsubsection{Example}
Consider an \gls{mm-1qfa} defined over the alphabet $\Sigma=\{a\}$ with the state set $Q=\{q_0,q_1,q_2\}$, where $q_0$ is the initial state, $Q_{acc}=\{q_2\}$, and $Q_{rej}=\{q_1\}$. Let the unitary operator for the symbol $a$ be given by:
\[
U_a = \begin{pmatrix}
\frac{1}{\sqrt{2}} & \frac{1}{\sqrt{2}} & 0 \\[1mm]
\frac{1}{\sqrt{2}} & -\frac{1}{\sqrt{2}} & 0 \\[1mm]
0 & 0 & 1
\end{pmatrix}.
\]
The \gls{mm-1qfa} processes an input string such as $aa$ as follows:
\begin{enumerate}
    \item Starting in state $|q_0\rangle$, the operator $U_a$ is applied and a measurement is performed. The measurement may collapse the state into:
    \begin{itemize}
        \item $E_{acc}$ (state $q_2$) – leading to immediate acceptance,
        \item $E_{rej}$ (state $q_1$) – leading to immediate rejection, or
        \item $E_{non}$ (state $q_0$, in this example) – allowing the computation to continupe.
    \end{itemize}
    \item If the first measurement yields a non-halting result, the second symbol is processed in a similar manner. The overall acceptance probability is the sum of the probabilities of all computation paths that eventually result in an outcome within $E_{acc}$.
\end{enumerate}
This example demonstrates the stepwise measurement process that characterizes \glspl{mm-1qfa}~\cite{kondacs1997power,lin2012another}.

\subsubsection{Additional Topics}
\paragraph{Equivalence and Decision Problems:} Lin~\cite{lin2012another} presents a simplified approach for deciding the equivalence of two \glspl{mm-1qfa} by reducing the problem to comparing initial vectors, thereby streamlining the decision process.

\paragraph{State Complexity and Lower Bounds:} Lower bound results for \gls{1qfa}, such as those by Ablayev and Gainutdinova~\cite{ablayev2000lower}, provide insights into the inherent state complexity challenges that also impact \glspl{mm-1qfa}.

\paragraph{Experimental Considerations:} While experimental implementations have predominantly focused on \gls{mo-1qfa}s due to their relative simplicity, future work may explore the adaptation of techniques (e.g., custom pulse shaping as demonstrated in~\cite{lussi2024implementing}) to the more complex \gls{mm-1qfa} framework.

