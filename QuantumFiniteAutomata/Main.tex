\chapter{Quantum Finite Automata}
\label{chap:quantum-finite-automata}

The study of \glspl{qfa} arose from the question of how far the
finite-memory paradigm of \glspl{cfa} extends when the transition
map is replaced by symbol-controlled unitary operators followed by
quantum measurement. Since the seminal formulations of measure-once and
measure-many one-way machines \cite{kondacs1997power,
moore2000quantum}, this minimalist model has revealed capabilities
that distinguish it sharply from its deterministic counterpart; examples
include exponential state savings in promise-problem recognition
\cite{ambainis19981} and interactive proof power that eludes
classical finite verifiers of identical size \cite{nishimura2009application}.

This chapter offers a concise taxonomy that tracks the historical
broadening of the model while keeping a uniform notation. We begin with
the foundational one-way variants, then investigate how two-way head
motion, hybrid quantum-classical control or bounded counters enlarge the
language classes recognised, occasionally reaching beyond the regular
languages \cite{brodsky2002characterizations,ambainis2002two,zheng2012one}. The survey closes with less common
frameworks, such as postselection, promise settings and $\omega$-word machines, 
that continue to challenge classical intuitions
\cite{aaronson2005quantum,scegulnaja2010postselection,bhatia2019quantum}.

Throughout the discussion each device is specified as a fixed-size tuple
whose components highlight the role of the quantum state space, the
observable structure and the head movement policy. Key results on
expressive power, closure properties and decidability are presented, culminating in a comparative
table that juxtaposes state complexity bounds and closure behaviour
across all surveyed models. This structured overview equips the reader
with the precise vocabulary and conceptual tools required for the later
translation of \glspl{qfa} into executable quantum circuits developed in
Chapter\,\ref{chap:automata-to-circuits}.

\section{Detailed Models of Quantum Finite Automata}
\label{sec:detailed-models}
Among the numerous variants of \glspl{qfa}, the measure-once and measure-many models stand out as two of the most foundational and widely studied frameworks. Respectively known as \gls{mo-1qfa} and \gls{mm-1qfa}, these automata provide a minimalistic yet powerful theoretical playground for investigating the principles of quantum computation with finite memory. Their significance lies not only in their historical development as some of the earliest quantum models for language recognition \cite{moore2000quantum, kondacs1997power}, but also in their role as archetypal examples in the study of quantum-classical computational boundaries.

A \gls{mo-1qfa} performs unitary operations throughout the input reading process and conducts a single measurement only at the end of the computation. This model, introduced by Moore and Crutchfield \cite{moore2000quantum}, can recognise only a restricted class of \glspl{reg}, such as the so-called group languages \cite{brodsky2002characterizations}. In contrast, the \gls{mm-1qfa}, introduced by Kondacs and Watrous \cite{kondacs1997power}, allows measurements after each symbol is processed, enabling it to recognise a strictly larger class of \glspl{reg}, though still not the full class.

The primary importance of \gls{mo-1qfa} and \gls{mm-1qfa} is not just theoretical. These models are particularly suitable for demonstrating techniques for the compilation of quantum automata into quantum circuits, as we will explore in Chapter ~\ref{chap:automata-to-circuits}. Due to their simpler architecture—one-way movement, discrete time steps, and finite-dimensional Hilbert spaces—these automata provide an ideal framework for illustrating how abstract automaton transitions can be implemented using quantum gates and projective measurements. In this thesis, they will serve as canonical models to illustrate the compilation strategy, setting a baseline for comparison with more advanced or generalised \gls{qfa} variants.


\subsection{\glsentrylong{mo-1qfa}}
\label{sec:moqfa}

\subsubsection{Introduction}
\glspl{mo-1qfa} represent one of the simplest models of quantum computation in the realm of automata theory. Introduced by Moore and Crutchfield in 2000~\cite{moore2000quantum}, \glspl{mo-1qfa} evolve solely through unitary transformations corresponding to the input symbols and perform a single measurement at the end of the computation. This model has been further characterized by Brodsky and Pippenger~\cite{brodsky2002characterizations}, and it is known for its conceptual simplicity as well as for its limitations. Notably, when restricted to bounded error, \glspl{mo-1qfa} recognize exactly the class of group languages—a proper subset of the regular languages. In contrast, the measure-many variant ~\cite{kondacs1997power} employs intermediate measurements and exhibits different acceptance capabilities.

\subsubsection{Formal Definition}
An \gls{mo-1qfa} is formally defined as a 5-tuple 
\[
M = (Q,\Sigma,\delta,q_0,F),
\]
where:
\begin{itemize}
    \item $Q$ is a finite set of states,
    \item $\Sigma$ is a finite input alphabet, typically augmented with a designated end-marker (e.g., \$),
    \item $\delta : Q \times \Sigma \times Q \to \mathbb{C}$ is the transition function such that, for every symbol $\sigma\in\Sigma$, the matrix 
    \[
    U_\sigma,\quad \text{with} \quad (U_\sigma)_{q,q'} = \delta(q,\sigma,q'),
    \]
    is unitary~\cite{moore2000quantum},
    \item $q_0 \in Q$ is the initial state, and
    \item $F\subseteq Q$ is the set of accepting states.
\end{itemize}
For an input string $x=x_1x_2\cdots x_n$, the computation proceeds by applying the corresponding unitary matrices sequentially:
\[
|\Psi_x\rangle = U_{x_n}U_{x_{n-1}}\cdots U_{x_1}|q_0\rangle.
\]
After reading the entire input, a measurement is performed using the projection operator
\[
P=\sum_{q\in F} |q\rangle\langle q|,
\]
so that the acceptance probability is defined as
\[
p_M(x)=\|P\,|\Psi_x\rangle\|^2.
\]
Alternative characterizations, including formulations using the Heisenberg picture, have been discussed in~\cite{qiu2004characterizations,piazza2022mirrors}.

\subsubsection{Strings Acceptance}
A string $x$ is accepted by an \gls{mo-1qfa} if the acceptance probability $p_M(x)$ exceeds a predetermined cut-point $\lambda$. In a bounded error setting, there exists a margin $\epsilon > 0$ such that:
\[
\forall x\in L:\quad p_M(x) \ge \lambda + \epsilon,
\]
\[
\forall x\notin L:\quad p_M(x) \le \lambda - \epsilon.
\]
Under the unbounded error regime, \glspl{mo-1qfa} can accept some nonregular languages (for instance, solving the word problem over the free group)~\cite{brodsky2002characterizations}. The precise acceptance behavior thus depends on whether a cut-point or a bounded error framework is adopted.

\subsubsection{Set of Languages Accepted}
When \glspl{mo-1qfa} are restricted to bounded error acceptance, they recognize exactly the class of \emph{group languages}—languages whose syntactic semigroups form groups~\cite{brodsky2002characterizations}. This class forms a strict subset of the regular languages, emphasizing the inherent limitation of the \gls{mo-1qfa} model. In contrast, by relaxing the error bounds, one may design \glspl{mo-1qfa} that accept a broader range of languages, albeit often at the cost of increased computational complexity.

\subsubsection{Closure Properties}
The class of languages accepted by \glspl{mo-1qfa} under bounded error exhibits robust closure properties. Specifically, this class is closed under:
\begin{itemize}
    \item Inverse Homomorphisms~\cite{brodsky2002characterizations},
    \item Word Quotients~\cite{brodsky2002characterizations},
    \item Boolean Operations (union, intersection, and complement)~\cite{freivalds2005languages,bertoni2003quantum}.
\end{itemize}
Additional algebraic properties, including aspects related to the pumping lemma and the structure of the accepting probabilities, have been further elaborated in works such as Ambainis et al.~\cite{ambainis1999probabilities} and Xi et al.~\cite{xi2008some}.

\subsubsection{Summary of Advantages and Limitations}
\glspl{mo-1qfa} are praised for their simplicity. Since the quantum state evolves unitarily until a single measurement is made, the model avoids the complications associated with intermediate state collapses. This simplicity has practical implications; for example, recent experimental work has shown that custom control pulses can significantly reduce error rates in IBM-Q implementations~\cite{lussi2024implementing}, and photonic implementations have further demonstrated the feasibility of \glspl{mo-1qfa} in optical setups~\cite{candeloro2021enhanced}. On the downside, the acceptance power of \glspl{mo-1qfa} is limited—when operating under bounded error, they recognize only the group languages, which form a strict subset of the regular languages. In contrast, \glspl{mm-1qfa} offer greater acceptance power but at the cost of increased model complexity~\cite{kondacs1997power,berzicna2001ambainis}.

\subsubsection{Example}
Consider a simple \gls{mo-1qfa} defined over the unary alphabet $\Sigma=\{a\}$ with state set $Q=\{q_0,q_1\}$, initial state $q_0$, and accepting state set $F=\{q_1\}$. Let the unitary operator corresponding to the symbol $a$ be defined by the rotation matrix:
\[
U_a = \begin{pmatrix}
\cos\theta & -\sin\theta \\
\sin\theta & \cos\theta
\end{pmatrix},
\]
for a fixed angle $\theta$. For an input string $a^n$, the state evolves as:
\[
|\Psi_{a^n}\rangle = U_a^n |q_0\rangle.
\]
The acceptance probability is computed as:
\[
p_M(a^n)=\|P\,|\Psi_{a^n}\rangle\|^2,
\]
where the projection operator $P$ is given by $P=|q_1\rangle\langle q_1|$. By appropriately choosing $\theta$, the automaton can be tuned so that $p_M(a^n)$ exceeds the cut-point $\lambda$ (for example, $\lambda=\frac{1}{2}$) if and only if $a^n$ belongs to the target language. This example illustrates the essential mechanism of \glspl{mo-1qfa}, as described in~\cite{moore2000quantum,brodsky2002characterizations}.

\subsubsection{Additional Topics}
\paragraph{Learning and Optimization:} Recent work by Chu et al.~\cite{chu2023approximately} has introduced methods that combine active learning with non-linear optimization to approximately learn the parameters of \glspl{mo-1qfa}. These techniques provide insights into how one can recover the unitary transformations and state structure from observed data.

\paragraph{Complexity and Minimization:} The problem of minimizing the number of states in an \gls{mo-1qfa} was originally posed by Moore and Crutchfield~\cite{moore2000quantum}. Subsequent work by Mateus, Qiu, and Li~\cite{mateus2012complexity} has established an EXPSPACE upper bound for the minimization problem, framing it as a challenge in solving systems of algebraic polynomial (in)equations.

\paragraph{Experimental Implementations:} Experimental realizations of \glspl{mo-1qfa} have also been explored. Lussi et al.~\cite{lussi2024implementing} demonstrated an implementation on IBM-Q devices using custom control pulses, while photonic approaches have been reported in~\cite{candeloro2021enhanced}. These works highlight both the practical challenges and the potential advantages of implementing \glspl{mo-1qfa} on current quantum hardware.

\paragraph{Future Directions:} Future research may focus on further enhancing experimental implementations, developing more robust learning algorithms for \glspl{mo-1qfa}, and exploring new minimization techniques that could lead to more efficient automata. Extensions that combine features of \glspl{mo-1qfa} and \gls{mm-1qfa}s may also provide richer language recognition capabilities and deepen our understanding of quantum computational models.


\subsection{\glsentrylong{mm-1qfa}}
\label{sec:mmqfa}
\glspl{mm-1qfa} are a variant of \glspl{qfa} in which a measurement is performed after reading each input symbol. Introduced by Kondacs and Watrous in 1997~\cite{kondacs1997power}, \glspl{mm-1qfa} allow the automaton to collapse its quantum state at intermediate steps, thereby potentially influencing the computation dynamically. Although this mechanism can enhance the detection of accepting or rejecting conditions during the run, under the bounded error regime \glspl{mm-1qfa} are known to recognise only a proper subset of the \glspl{reg}~\cite{brodsky2002characterizations}. Recent work, such as by Lin~\cite{lin2012another}, has provided elegant methods for addressing the equivalence problem of \glspl{mm-1qfa}, further enriching our understanding of their computational properties.

\begin{definition}[\gls{mm-1qfa}]
A \glsentryfull{mm-1qfa} is defined as a 6-tuple
\[
M = (Q,\Sigma,\delta,q_0,Q_{acc},Q_{rej}),
\]
where:
\begin{itemize}
    \item $Q$ is a finite set of states,
    \item $\Sigma$ is a finite input alphabet, typically augmented with an end-marker (e.g., \$),
    \item $\delta : Q \times \Sigma \times Q \to \mathbb{C}$ is the transition function, where for each symbol $\sigma\in\Sigma$ the corresponding matrix 
    \[
    U_\sigma,\quad \text{with} \quad (U_\sigma)_{q,q'}=\delta(q,\sigma,q'),
    \]
    is unitary~\cite{kondacs1997power},
    \item $q_0 \in Q$ is the initial state,
    \item $Q_{acc} \subseteq Q$ is the set of accepting (halting) states, and
    \item $Q_{rej} \subseteq Q$ is the set of rejecting (halting) states.
\end{itemize}
\end{definition}

After each symbol is read, the automaton's current state is measured with respect to the decomposition
\[
E_{acc} = \text{span}\{\ket{q} : q \in Q_{acc}\},\quad
E_{rej} = \text{span}\{\ket{q} : q \in Q_{rej}\},\quad
E_{non} = \text{span}\{\ket{q} : q \in Q \setminus (Q_{acc}\cup Q_{rej})\}.
\]
If the measurement outcome lies in $E_{acc}$ or $E_{rej}$, the computation halts immediately with acceptance or rejection, respectively. This definition, adapted from Kondacs and Watrous~\cite{kondacs1997power} and refined in Lin~\cite{lin2012another}, forms the basis of the \gls{mm-1qfa} model.

\subsubsection{Strings Acceptance}
For an input string $x=x_1x_2\cdots x_n$, the \gls{mm-1qfa} processes each symbol sequentially. At each step $i$, the unitary operator $U_{x_i}$ is applied, followed by a measurement:
\begin{itemize}
    \item If the measurement result falls in $E_{acc}$, the automaton immediately accepts $x$.
    \item If it falls in $E_{rej}$, the automaton rejects $x$.
    \item If the result lies in $E_{non}$, the computation continues with the next symbol.
\end{itemize}
The overall acceptance probability of $x$ is the cumulative probability of all computation paths that eventually lead to an accepting state. In a bounded error framework, there exists a margin $\epsilon > 0$ such that for every $x\in L$, the acceptance probability satisfies
\[
p_M(x) \ge \lambda + \epsilon,
\]
and for every $x\notin L$, 
\[
p_M(x) \le \lambda - \epsilon,
\]
where $\lambda$ is a predetermined cut-point (commonly set to $\frac{1}{2}$)~\cite{kondacs1997power,brodsky2002characterizations}.

\subsubsection{Set of Languages Accepted}
Under the bounded error constraint, \glspl{mm-1qfa} recognise a proper subset of the \glspl{reg}. In particular, the languages accepted by \glspl{mm-1qfa} must satisfy specific algebraic properties that restrict their expressive power. Although \glspl{mm-1qfa} can, in some cases, recognise non\glspl{reg} when allowed unbounded error, the bounded error condition confines them to a class that is comparable to that of group languages~\cite{brodsky2002characterizations,kondacs1997power}. This limitation underscores the trade-off between the increased measurement frequency and the resultant reduction in language recognition capability.

\subsubsection{Closure Properties}
The language class recognised by \glspl{mm-1qfa} with bounded error is known to enjoy several closure properties:
\begin{itemize}
    \item It is closed under complement and inverse homomorphisms~\cite{brodsky2002characterizations}.
    \item It is closed under word quotients~\cite{brodsky2002characterizations}.
    \item However, the class is not closed under arbitrary homomorphisms~\cite{kondacs1997power,bertoni2003quantum}.
\end{itemize}
Recent work by Lin~\cite{lin2012another} further refines our understanding of these closure properties by addressing the equivalence problem for \glspl{mm-1qfa}, thereby linking the structural properties of the recognised languages to the underlying automata.

\subsubsection{Summary of Advantages and Limitations}
\glspl{mm-1qfa} offer notable advantages:
\begin{itemize}
    \item The use of intermediate measurements can enable earlier detection of acceptance or rejection, potentially reducing the average computation time.
    \item The dynamic collapse of the quantum state provides a different balance between quantum coherence and classical decision-making.
\end{itemize}
Nevertheless, there are significant limitations:
\begin{itemize}
    \item The frequent measurements interrupt the quantum evolution, which can limit the automaton's ability to harness quantum interference effectively.
    \item As a result, under bounded error conditions, \glspl{mm-1qfa} recognise only a restricted subset of the \glspl{reg}.
    \item The complexity of analyzing and minimizing \glspl{mm-1qfa} remains high, with state minimization posing an EXPSPACE challenge~\cite{mateus2012complexity} and lower bound results highlighting the inherent state complexity~\cite{ablayev2000lower}.
\end{itemize}
Moreover, when compared to \gls{mo-1qfa}s, \glspl{mm-1qfa} may offer greater recognition power in some unbounded error scenarios but at the cost of increased computational and implementation complexity~\cite{kondacs1997power,berzicna2001ambainis}.


\subsubsection{Additional Topics}
\paragraph{Equivalence and Decision Problems.} Lin~\cite{lin2012another} presents a simplified approach for deciding the equivalence of two \glspl{mm-1qfa} by reducing the problem to comparing initial vectors, thereby streamlining the decision process.

\paragraph{State Complexity and Lower Bounds.} Lower bound results for \gls{1qfa}, such as those by Ablayev and Gainutdinova~\cite{ablayev2000lower}, provide insights into the inherent state complexity challenges that also impact \glspl{mm-1qfa}.
 
\paragraph{Experimental Considerations.} While experimental implementations have predominantly focused on \gls{mo-1qfa}s due to their relative simplicity, future work may explore the adaptation of techniques (e.g., custom pulse shaping as demonstrated in~\cite{lussi2024implementing}) to the more complex \gls{mm-1qfa} framework.

\begin{example}[Example of \gls{mm-1qfa} Processing]
    Consider an \gls{mm-1qfa} defined over the alphabet $\Sigma=\{a\}$ with the state set $Q=\{q_0,q_1,q_2\}$, where $q_0$ is the initial state, $Q_{acc}=\{q_2\}$, and $Q_{rej}=\{q_1\}$. Let the unitary operator for the symbol $a$ be given by:
    \[
    U_a = \begin{pmatrix}
    \frac{1}{\sqrt{2}} & \frac{1}{\sqrt{2}} & 0 \\[1mm]
    \frac{1}{\sqrt{2}} & -\frac{1}{\sqrt{2}} & 0 \\[1mm]
    0 & 0 & 1
    \end{pmatrix}.
    \]
    The \gls{mm-1qfa} processes an input string such as $aa$ as follows:
    \begin{enumerate}
        \item Starting in state $\ket{q}$, the operator $U_a$ is applied and a measurement is performed. The measurement may collapse the state into:
        \begin{itemize}
            \item $E_{acc}$ (state $q_2$) - leading to immediate acceptance,
            \item $E_{rej}$ (state $q_1$) - leading to immediate rejection, or
            \item $E_{non}$ (state $q_0$, in this example) - allowing the computation to continupe.
        \end{itemize}
        \item If the first measurement yields a non-halting result, the second symbol is processed in a similar manner. The overall acceptance probability is the sum of the probabilities of all computation paths that eventually result in an outcome within $E_{acc}$.
    \end{enumerate}
    This example demonstrates the stepwise measurement process that characterises \glspl{mm-1qfa}~\cite{kondacs1997power,lin2012another}.
\end{example}
\section{Main Models of Quantum Finite Automata}
\label{sec:main-models}

\subsection{\glsentrylong{1qfa}}
\label{sec:1-way-qfa}

\gls{1qfa} are the quantum analog of classical one-way finite automata where the input tape is read from left to right without revisiting symbols. They provide a model for finite quantum computation that is more restricted than two-way quantum finite automata but often simpler to implement and analyze.

Among the most studied variants of \gls{1qfa} are the \gls{mo-1qfa} and the \gls{mm-1qfa}, which are addressed in detail in Section~\ref{sec:detailed-models}. These automata differ primarily in the timing of their measurements: \gls{mo-1qfa} perform a single measurement at the end of the computation, while \gls{mm-1qfa} perform measurements after reading each input symbol.

Beyond these foundational models, several other types of \gls{1qfa} have been proposed, each offering unique computational perspectives or enhancements. We present a comprehensive overview of such models, each in its own subsubsection.

\subsubsection{\gls{mon-1qfa}}
The \gls{mon-1qfa} is a special model in which only measurement operations are used for computation, with no intermediate unitary evolutions. This model simplifies quantum computation by relying solely on projective measurements.

\paragraph{Formal Definition}
A \gls{mon-1qfa} is defined as a tuple \( A = (Q, \Sigma, \rho_0, \{P_{\sigma}\}_{\sigma \in \Sigma}, Q_{acc}) \) where:
\begin{itemize}
    \item \( Q \) is a finite set of states,
    \item \( \Sigma \) is a finite input alphabet,
    \item \( \rho_0 \) is the initial quantum state (density matrix),
    \item \( P_{\sigma} \) is the measurement operator for each input symbol \( \sigma \),
    \item \( Q_{acc} \subseteq Q \) is the set of accepting states.
\end{itemize}

\paragraph{Strings Acceptance}
Acceptance is determined by applying the appropriate projective measurement after each symbol and measuring the final state. Acceptance can be defined with a bounded-error threshold.

\paragraph{Sets of Languages Accepted}
The class of languages accepted by \gls{mon-1qfa} is strictly less powerful than the class of regular languages. It corresponds to a particular class of regular languages known as literally idempotent piecewise testable languages \cite{bertoni2010trace}.

\paragraph{Closure Properties}
The languages recognized by \gls{mon-1qfa} are not closed under union or complementation \cite{bertoni2010trace}.

\paragraph{Advantages and Limitations}
They offer a hardware-friendly model due to the absence of unitaries, but are strictly less powerful than \gls{mo-1qfa} and \gls{mm-1qfa}.

\paragraph{Comparison}
Compared to \gls{mo-1qfa} and \gls{mm-1qfa}, \gls{mon-1qfa} are less powerful due to the absence of unitary evolution.

\paragraph{Example}
An example is the language of all strings over \{a, b\} with an even number of a's. It can be recognized by a suitable choice of projective measurements.

\paragraph{Additional Topics}
Measure-only models relate to trace monoids with idempotent generators and have been used in language algebraic characterizations \cite{comin2013extended, bertoni2010trace}.

\subsubsection{One-Way Quantum Finite Automata with Two Observables (1QFA(2))}
\paragraph{Introduction}
The \gls{1qfa2} model introduces a second observable to the \gls{mo-1qfa} framework, enhancing the capacity to distinguish input strings. It can be seen as an intermediate enhancement over classical \gls{mo-1qfa}.

\paragraph{Formal Definition}
A \gls{1qfa2} is defined similarly to a standard \gls{mo-1qfa}, but it includes two projective measurements applied in alternation during computation:
\[
A = (Q, \Sigma, \rho_0, \{U_{\sigma}\}, \{P_1, P_2\}, Q_{acc})
\]
where the two measurements \( P_1 \) and \( P_2 \) alternate throughout the computation.

\paragraph{Strings Acceptance}
A string is accepted based on the final outcome after alternating between the two measurements. Acceptance is typically defined with bounded error.

\paragraph{Sets of Languages Accepted}
The class of languages recognized is still a proper subset of regular languages, although strictly more than \gls{mo-1qfa}.

\paragraph{Closure Properties}
These automata do not have known closure under union or intersection.

\paragraph{Advantages and Limitations}
The addition of a second observable enables more refined discrimination of inputs, albeit still with limited computational power.

\paragraph{Comparison}
The model is more expressive than \gls{mo-1qfa} but remains less powerful than general \gls{mm-1qfa}.

\paragraph{Additional Topics}
This line of work aligns with the broader research aim of incrementally extending the expressive power of \gls{1qfa} models, as seen in the approach of \cite{ciamarra2001quantum}.

\subsubsection{Two-Tape One-Way Quantum Finite Automata with Two Heads (2tQFA(2))}
\paragraph{Introduction}
The \gls{2t1qfa2} model incorporates two tapes and two heads, enabling cross-comparison between symbols of the input and reference tape, enhancing the recognition of complex patterns.

\paragraph{Formal Definition}
Formally, a \gls{2t1qfa2} is a 7-tuple \( A = (Q, \Sigma, \delta, q_0, Q_{acc}, Q_{rej}, \mathbb{T}) \), where \( \mathbb{T} \) denotes the tape set. The quantum evolution occurs in tandem over both tapes with corresponding heads.

\paragraph{Strings Acceptance}
Acceptance is determined through measurement at halting, typically with bounded error.

\paragraph{Sets of Languages Accepted}
\gls{2t1qfa2} can recognize some nonregular languages, exceeding the capabilities of classical one-way automata and standard \gls{1qfa} models \cite{ganguly20162}.

\paragraph{Closure Properties}
Closure properties remain largely unexplored, with no known results under union or complement.

\paragraph{Advantages and Limitations}
The addition of a second tape expands the computational power substantially, though it also introduces practical complexity.

\paragraph{Comparison}
Outperforms classical and most one-way quantum models in expressive power.

\paragraph{Additional Topics}
This model exemplifies a practical direction in extending \gls{1qfa} expressivity by architectural enhancement.

\subsubsection{Nondeterministic Quantum Finite Automata (NQFA)}
\paragraph{Introduction}
\gls{nqfa} introduce nondeterminism in the quantum setting. Unlike probabilistic nondeterminism, here the nondeterminism arises from quantum measurement outcomes and amplitude branches.

\paragraph{Formal Definition}
An \gls{nqfa} is a 5-tuple \( A = (Q, \Sigma, \psi_0, \{U_\sigma\}, Q_{acc}) \), and uses a cutpoint acceptance mode, where a string \( x \) is accepted if \( \mathbb{P}(x) > \lambda \) for some threshold \( \lambda \).

\paragraph{Strings Acceptance}
\gls{nqfa} recognize strings based on acceptance with nonzero amplitude, allowing acceptance of nonregular languages with bounded error \cite{yakaryilmaz2009languages}.

\paragraph{Sets of Languages Accepted}
This model strictly recognizes more than regular languages, offering power comparable to or exceeding classical nondeterministic finite automata.

\paragraph{Closure Properties}
\gls{nqfa} are not closed under complement or union.

\paragraph{Advantages and Limitations}
The model is powerful yet lacks constructive methods for deterministic acceptance, making some verification tasks harder.

\paragraph{Comparison}
More expressive than \gls{mo-1qfa}, \gls{mm-1qfa}, and even probabilistic finite automata in some cases.

\subsubsection{Reversible Quantum Finite Automata (RevQFA)}
\paragraph{Introduction}
\gls{revqfa} enforce reversibility in state transitions, in line with quantum mechanics principles, where computation steps are invertible.

\paragraph{Formal Definition}
Defined similarly to \gls{mo-1qfa}, but all transitions are reversible, and state evolution is enforced to be unitary across all paths. Measurement occurs only at the end.

\paragraph{Strings Acceptance}
A string is accepted based on the final state post unitary evolution.

\paragraph{Sets of Languages Accepted}
Recognizes all regular languages \cite{yamakami2014one}, differing from many other \gls{1qfa} models.

\paragraph{Closure Properties}
Closed under Boolean operations due to equivalence with classical deterministic automata.

\paragraph{Advantages and Limitations}
They enjoy a strong correspondence to classical reversible automata with quantum efficiency benefits, though do not surpass regular languages.

\paragraph{Comparison}
Unlike most \gls{1qfa}, \gls{revqfa} can simulate any DFA, thus closing the gap between quantum and classical finite automata in expressiveness.

\paragraph{Additional Topics}
The interest in \gls{revqfa} stems from the desire to harness quantum reversibility for computation, a core direction explored by \cite{ciamarra2001quantum}.

\subsection{Two-Way Quantum Finite Automata}
\label{sec:2-way-qfa}

\glspl{qfa} can be classified based on the measurement policy and head movement direction. In this subsection, we focus exclusively on \glspl{2qfa} models that operate under pure quantum evolution: namely, the \gls{mo-2qfa} and \gls{mm-2qfa}. These automata extend the capabilities of their one-way counterparts by allowing bidirectional movement of the tape head, while differing in how often measurements are performed during computation.

\subsubsection{\glsentryfull{mo-2qfa}}

The \gls{mo-2qfa} was formally introduced by Xi et al. \cite{xi2008some} as the natural two-way extension of the (\gls{mo-1qfa}). It performs unitary transformations while scanning the input in both directions but conducts a projective measurement only once—at the end of the input.

\begin{definition}[\gls{mo-2qfa}]
     An \gls{mo-2qfa} is defined as a 5-tuple $M = (Q, \Sigma, \delta, q_0, Q_a)$ where:
\begin{itemize}
    \item $Q$ is a finite set of quantum states,
    \item $\Sigma$ is a finite input alphabet,
    \item $\delta: Q \times \Gamma \times Q \times D \to \mathbb{C}$ is the transition function with $\Gamma = \Sigma \cup \{\#, \$\}$ and $D = \{-1, 0, +1\}$ indicating tape head movement,
    \item $q_0 \in Q$ is the initial state,
    \item $Q_a \subseteq Q$ is the set of accepting states.
\end{itemize}
\end{definition}

The unitary evolution is enforced through conditions of orthogonality and separability as outlined in \cite{xi2008some}.

\paragraph{Strings Acceptance} Acceptance is defined via a single projective measurement after the entire input string, delimited by endmarkers, is processed. A string is accepted with a probability computed from the projection onto the accepting subspace. The model supports acceptance with bounded error, cutpoint, and exact acceptance depending on configuration \cite{xi2008some}.

\paragraph{Set of Languages Recognised} The class of languages recognised by \gls{mo-2qfa} strictly contains the languages recognised by \gls{mo-1qfa}, and it includes some non\glspl{reg}. In fact, \gls{mo-2qfa} can recognise proper supersets of group languages and supports complex operations such as intersection and reversal \cite{xi2008some}.

\paragraph{Closure Properties} The languages recognised by \gls{mo-2qfa} are closed under union, intersection, complement, and reversal. Notably, the closure under intersection and union can be achieved by direct sum and tensor product constructions, respectively \cite{xi2008some}.

\paragraph{Advantages and Limitations} The main advantage of \gls{mo-2qfa} lies in its improved recognition power over one-way models. However, the measurement at the end implies potentially high computational complexity for simulating classical behaviour. Its limitation includes the reliance on exact unitary constraints and nontrivial implementation challenges \cite{xi2008some}.

\paragraph{Comparison} Compared to \gls{mo-1qfa}, the \gls{mo-2qfa} is strictly more powerful due to its bidirectional scanning ability. It is less expressive than \gls{mm-2qfa} in general, since the latter allows more frequent measurements and thus a richer computational structure.

\paragraph{Additional Topics} The authors suggest future directions include studying the closure under concatenation in more general terms and optimizing state complexity for specific regular operations \cite{xi2008some}.

\begin{example}
An example from \cite{xi2008some} shows that \gls{mo-2qfa} can recognise the language $L = \{ a^nb^n \mid n \geq 1 \}$ with bounded error—something not possible with any one-way QFA model.
\end{example}

\subsubsection{\glsentryfull{mm-2qfa}}

The \gls{mm-2qfa} was introduced by Kondacs and Watrous \cite{kondacs1997power} and represents the most powerful pure QFA model in terms of language recognition. It performs a measurement at each computational step and allows head movement in both directions.

\begin{definition}[\gls{mm-2qfa}]
An \gls{mm-2qfa} is a 6-tuple $M = (Q, \Sigma, \delta, q_0, Q_a, Q_r)$ where:
\begin{itemize}
    \item $Q$ is the finite set of states partitioned into $Q_n$ (non-halting), $Q_a$ (accepting), and $Q_r$ (rejecting) states,
    \item $\delta: Q \times \Gamma \times Q \times D \to \mathbb{C}$ is the transition function,
    \item The machine performs a projective measurement at each step to determine whether to accept, reject, or continue.
\end{itemize}
\end{definition}
Well-formedness constraints are necessary to ensure unitarity and validity of transition amplitudes \cite{kondacs1997power}.

\paragraph{Strings Acceptance} A string is accepted if, during any computational step, the machine transitions into an accepting state. This supports acceptance with bounded error, cutpoint, and exact modes depending on the configuration of the measurement \cite{kondacs1997power}.

\paragraph{Set of Languages Recognised} The \gls{mm-2qfa} can recognise non\glspl{reg} such as $L_{eq} = \{ a^n b^n \mid n \geq 0 \}$ with bounded error in linear time. This is a significant enhancement over any classical 2PFA or 1QFA model \cite{kondacs1997power}.

\paragraph{Closure Properties} The class of languages recognised by \gls{mm-2qfa} is not known to be closed under union or intersection. This is a key limitation of the model despite its high expressive power.

\paragraph{Advantages and Limitations} The model can recognise non\glspl{reg} efficiently, offering exponential advantages in space over classical automata. However, it is more complex to analyse due to frequent measurements and lacks closure under basic operations \cite{kondacs1997power, qiu2008state}.

\paragraph{Comparison} \gls{mm-2qfa} is strictly more powerful than all one-way models (\gls{mo-1qfa}, \gls{mm-1qfa}) and more powerful than \gls{mo-2qfa} in general. However, its structure is harder to simulate and analyse, limiting its practical implementation.

\paragraph{Additional Topics} Extensions to hybrid models such as \glspl{2qcfa} and \glsentryfull{qip} with \gls{2qfa} verifiers have been proposed to harness the strengths of \gls{mm-2qfa} while addressing its limitations \cite{pani2011empowering, qiu2008state}.

\begin{example}
    Kondacs and Watrous \cite{kondacs1997power} present a \gls{mm-2qfa} that recognises $L_{eq}$ in linear time with bounded error—a feat requiring exponential time for classical two-way probabilistic automata.
\end{example}
\subsection{Hybrid Quantum Finite Automata}
\label{sec:hybrid-qfa}

Hybrid quantum finite automata (HQFA) are finite-state machines that combine a quantum state component with a classical state component. In essence, an HQFA consists of a quantum system and a classical finite automaton that operate together, communicating information between them during the computation \cite{li2015hybrid}. The classical part typically controls the movement of the input tape head (which, in the models discussed here, will be one-way unless otherwise specified) and can also influenc...

In all these models, the goal is to leverage a small quantum memory together with classical states to recognize languages, potentially with far fewer states than a purely classical automaton would require \cite{zheng2012one}. At the same time, these models are usually easier to implement physically than fully quantum automata, since the classical control can simplify certain tasks (e.g., tracking the head position) \cite{qiu2009one}. Future realistic quantum computers are expected to be hybrid systems wi...

Several important hybrid QFA models have been introduced. Ambainis and Watrous \cite{ambainis2002two} first proposed the two-way quantum finite automaton with classical states (2QCFA), which augments a two-way deterministic finite automaton with a constant-size quantum register. One-way restrictions and variations of this idea were later explored. Bertoni, Mereghetti, and Palano introduced the model of QFA with control language (CL-1QFA) \cite{mereghetti2006quantum}, where a classical DFA filters the out...

More advanced hybrids include multi-tape extensions like the 2TQCFA and $k$TQCFA, which increase computational power by leveraging multiple input tapes \cite{zheng2011two}. All one-way hybrid QFA models (1QCFA, 1QFAC, CL-1QFA) recognize exactly the regular languages \cite{li2015hybrid, zheng2012one}. However, they can be significantly more \textit{succinct} in terms of state complexity than classical models \cite{xiao2021state}.

\subsubsection{1QCFA (One-Way Quantum Finite Automata with Classical States)}

\paragraph{Introduction:}  
The 1QCFA model, introduced by Zheng, Qiu, Li, and Gruska \cite{zheng2012one}, augments a classical one-way finite automaton with a quantum component. It can be seen as a one-way restriction of the 2QCFA model. The 1QCFA features two-way communication between classical and quantum parts: the classical state influences the quantum operation applied, and the quantum measurement outcome affects the next classical state. This model allows real-time probabilistic quantum experiments on input symbols and is us...

\paragraph{Formal Definition:}  
A 1QCFA is a 9-tuple:
\[
A = (Q, S, \Sigma, C, q_1, s_1, \{\Theta_{s,\sigma}\}, \delta, S_a)
\]
where:
\begin{itemize}
    \item $Q$: finite set of quantum basis states,
    \item $S$: finite set of classical states,
    \item $\Sigma$: input alphabet,
    \item $C$: measurement outcomes,
    \item $q_1 \in Q$, $s_1 \in S$: initial quantum and classical states,
    \item $\Theta_{s,\sigma}$: quantum operation with outcomes in $C$,
    \item $\delta: S \times \Sigma \times C \to S$: classical transition function,
    \item $S_a \subseteq S$: accepting states.
\end{itemize}
Computation begins in $(s_1, |q_1\rangle)$, proceeds symbol-by-symbol. On input $x = x_1x_2\cdots x_n$, the machine updates classical and quantum states based on outcomes $c_i \in C$ generated by each $\Theta_{s_i,x_i}$, applying $\delta(s_i,x_i,c_i)$ at every step. Acceptance is determined by whether the final classical state is in $S_a$ \cite{li2015hybrid}.

\paragraph{Strings Accepted:}  
A 1QCFA accepts string $x$ with probability based on outcome paths $c_1c_2\cdots c_n$. If the classical state ends in $S_a$, it accepts. Languages are recognized with bounded error $\varepsilon < 1/2$ if acceptance probability is $\geq 1-\varepsilon$ for all $x \in L$ and $\leq \varepsilon$ for all $x \notin L$ \cite{li2015hybrid, zheng2012one}.

\paragraph{Sets of Languages Recognized:}  
1QCFA recognize exactly the class of regular languages \cite{zheng2012one}. Any regular language can be recognized by some 1QCFA with certainty. Moreover, they can be exponentially more succinct than DFA for some regular languages \cite{xiao2021state}.

\paragraph{Closure Properties:}  
The class of languages recognized by 1QCFA is closed under union, intersection, and complement, as it coincides with the regular languages \cite{li2015hybrid}.

\paragraph{Advantages and Limitations:}  
While 1QCFA do not exceed DFA in language power, they are often exponentially more state-efficient. For instance, certain periodic languages can be recognized with a single qubit: rotating the quantum state by $2\pi/p$ for each $a$ and measuring at the end to detect if the state returned to its initial position (full rotation) \cite{xiao2021state, bianchi2014size}.

\paragraph{Comparison Between Models:}  
1QCFA generalize CL-1QFA and 1QFAC. The main difference lies in bidirectional communication. In comparison to 2QCFA, 1QCFA are weaker, since they cannot move backward on the input or recognize non-regular languages \cite{li2015hybrid}.

\paragraph{Example:}  
A 1QCFA can recognize the language $L = \{ a^n : n \equiv 0 \mod p \}$ using a single qubit: rotating the quantum state by $2\pi/p$ for each $a$ and measuring at the end to detect if the state returned to its initial position (full rotation) \cite{bianchi2014power}.

\paragraph{Additional Topics:}  
1QCFA equivalence is decidable \cite{li2015hybrid}. State trade-offs between classical and quantum resources have been studied extensively \cite{qiu2009equivalence, xiao2021state}. Future work includes refining state complexity bounds and exploring minimal configurations.


\subsubsection{CL-1QFA (One-Way Quantum Finite Automata with Control Language)}

\paragraph{Introduction:}  
The CL-1QFA (Quantum Finite Automata with Control Language) model, introduced by Bertoni, Mereghetti, and Palano \cite{mereghetti2006quantum}, consists of a quantum component responsible for unitary operations and measurements, and a classical DFA that processes the sequence of measurement outcomes. The role of the control language is to guide acceptance: a word is accepted if and only if the sequence of measurement outcomes belongs to a regular language defined by the control automaton.

\paragraph{Formal Definition:}  
A CL-1QFA is a 6-tuple:
\[
A = (Q, \Sigma, \{U_\sigma\}_{\sigma \in \Sigma}, q_0, M, L)
\]
where:
\begin{itemize}
    \item $Q$: finite set of quantum basis states,
    \item $\Sigma$: input alphabet,
    \item $U_\sigma$: unitary transformation applied upon reading symbol $\sigma$,
    \item $q_0 \in Q$: initial quantum state,
    \item $M$: projective measurement with outcomes in a finite set $\Gamma$,
    \item $L \subseteq \Gamma^*$: regular control language recognized by a classical DFA.
\end{itemize}
On input $x = x_1x_2\cdots x_n$, the automaton applies $U_{x_i}$ and measures after each step, producing an output string $y \in \Gamma^n$. The word $x$ is accepted if $y \in L$.

\paragraph{Strings Accepted:}  
Acceptance depends entirely on whether the sequence of quantum measurement results belongs to the control language $L$. This model generally uses bounded-error acceptance. However, exact acceptance is possible for certain languages, depending on the control DFA and quantum measurements \cite{mereghetti2006quantum}.

\paragraph{Sets of Languages Recognized:}  
CL-1QFA recognize exactly the class of regular languages \cite{li2015hybrid}. Though they do not surpass the regular class in power, their structure allows for different expressive strategies by decoupling quantum operations from classical verification.

\paragraph{Closure Properties:}  
The class of languages recognized is closed under union, intersection, and complement because the control language is regular and the measurement outcomes are deterministic modulo quantum probabilities \cite{li2015hybrid}.

\paragraph{Advantages and Limitations:}  
Advantages include clear separation between quantum processing and classical control, which simplifies modular design and analysis. The main limitation is that the model cannot recognize non-regular languages and does not allow dynamic feedback between quantum and classical components \cite{li2015hybrid}.

\paragraph{Comparison Between Models:}  
CL-1QFA can be simulated by 1QCFA \cite{li2015hybrid}, but not vice versa. Unlike 1QCFA, CL-1QFA do not allow two-way communication: measurement outcomes do not affect the ongoing quantum state. This makes CL-1QFA structurally simpler, but less expressive in practice.

\paragraph{Example:}  
To recognize $L = (ab)^*$, a CL-1QFA can measure each input symbol’s quantum effect and produce a binary output: `0` for $a$, `1` for $b$. The control DFA can then accept only if the output string alternates properly and has even length \cite{mereghetti2006quantum}.

\paragraph{Additional Topics:}  
Variants of CL-1QFA using unary control languages have been recently proposed to explore succinctness and unary acceptance conditions \cite{mereghetti2024unary}. These help analyze minimal state configurations and potential hardware implementations.


\subsubsection{2QCFA (Two-Way Quantum Finite Automata with Classical States)}

\paragraph{Introduction:}  
The 2QCFA model, introduced by Ambainis and Watrous \cite{ambainis2002two}, is a hybrid automaton that consists of a classical two-way deterministic finite control and a constant-size quantum register. It was designed to exploit quantum computation while maintaining classical control over input movement. This makes the model both powerful and physically realizable, allowing the classical part to handle head movement and state tracking, while the quantum component processes information probabilistically.

\paragraph{Formal Definition:}  
A 2QCFA is a 9-tuple:
\[
A = (Q, S, \Sigma, \Theta, \delta, q_0, s_0, S_a, S_r)
\]
where:
\begin{itemize}
    \item $Q$: finite set of quantum basis states,
    \item $S$: finite set of classical states,
    \item $\Sigma$: input alphabet,
    \item $\Theta$: quantum transition function defining unitary operators or measurements,
    \item $\delta$: classical transition function based on current state, symbol, and measurement outcome,
    \item $q_0 \in Q$, $s_0 \in S$: initial quantum and classical states,
    \item $S_a$, $S_r$: sets of accepting and rejecting classical states.
\end{itemize}
The classical control can move the tape head both left and right. The quantum state is manipulated via unitary transformations or measurements, which are determined by the classical state and scanned symbol. Decisions are made based on both the classical and quantum information.

\paragraph{Strings Accepted:}  
2QCFA accept strings using bounded error or with one-sided error. For example, languages like $L_{eq} = \{ a^n b^n \mid n \geq 1 \}$ can be accepted with one-sided bounded error in expected polynomial time \cite{ambainis2002two}.

\paragraph{Sets of Languages Recognized:}  
2QCFA can recognize certain non-regular languages, including $L_{eq}$ and palindromes over unary alphabets, which makes them strictly more powerful than classical DFA or one-way QFA \cite{ambainis2002two, li2015hybrid}.

\paragraph{Closure Properties:}  
The class of languages recognized by 2QCFA is not closed under union or intersection, due to the constraints of the probabilistic error bounds and two-way head movement. However, they maintain closure under reversal and concatenation in specific cases \cite{li2015hybrid}.

\paragraph{Advantages and Limitations:}  
Advantages include greater recognition power than 1QCFA and succinctness for certain problems. For example, 2QCFA can recognize $L_{eq}$ with only a constant-size quantum register and logarithmic classical states \cite{remscrim2020power}. However, they are generally limited to languages where probabilistic techniques suffice, and their runtime is often polynomial in the worst case \cite{remscrim2020lower}.

\paragraph{Comparison Between Models:}  
2QCFA generalize 1QCFA by allowing two-way head movement, which significantly increases computational power. In contrast to CL-1QFA, they use dynamic feedback from the quantum measurements to the classical state transitions. 2QCFA are more expressive but harder to analyze due to interaction complexity \cite{zheng2013state}.

\paragraph{Example:}  
The language $L_{eq} = \{ a^n b^n \mid n \geq 1 \}$ can be recognized by a 2QCFA by using the quantum register to randomly check positions and probabilistically verify balance between $a$'s and $b$'s through repeated subroutines \cite{ambainis2002two}.

\paragraph{Additional Topics:}  
Future work includes better understanding the time complexity of 2QCFA algorithms and developing minimization techniques. Variants include alternating 2QCFA and state-succinct encodings \cite{zheng2013state, remscrim2020lower}.

\subsubsection{2TQCFA (Two-Tape Quantum Finite Automata with Classical States)}

\paragraph{Introduction:}  
The 2TQCFA model (Two-Tape Quantum-Classical Finite Automata) extends the 2QCFA by using two input tapes instead of one. Introduced by Zheng, Li, and Qiu \cite{zheng2011two}, this model enhances computational power by enabling comparisons and synchronized traversal of two input strings. The quantum component remains fixed in size, while the classical controller can move the heads on both tapes and perform transitions based on measurements.

\paragraph{Formal Definition:}  
A 2TQCFA is formally a tuple similar to a 2QCFA but with two input tapes:
\[
A = (Q, S, \Sigma, \Theta, \delta, q_0, s_0, S_a, S_r)
\]
with the following distinctions:
\begin{itemize}
    \item Two input heads, each reading a separate string from $\Sigma^*$,
    \item Classical state $s \in S$ determines movement and operation on each tape head,
    \item Quantum operations $\Theta$ depend on the symbols scanned by both heads and classical state.
\end{itemize}
As in 2QCFA, the automaton evolves through interactions between classical and quantum transitions, but the two-tape structure allows for cross-input comparisons.

\paragraph{Strings Accepted:}  
2TQCFA can accept languages using bounded-error acceptance, typically with one-sided error. A notable example includes the language $L = \{ w \# w \mid w \in \{a, b\}^* \}$, which is non-regular and not recognizable by 2QCFA, but accepted by 2TQCFA using synchronous traversal of both input halves \cite{zheng2011two}.

\paragraph{Sets of Languages Recognized:}  
The language recognition power of 2TQCFA includes certain context-free and non-regular languages not recognizable by 2QCFA. Thus, 2TQCFA strictly extends the power of 2QCFA under bounded-error acceptance \cite{zheng2011two}.

\paragraph{Closure Properties:}  
Due to the added complexity of two-tape processing, closure properties are less well-defined. However, the model is still limited by finite memory and cannot recognize arbitrary context-free languages \cite{li2015hybrid}.

\paragraph{Advantages and Limitations:}  
The primary advantage is an increased ability to perform input comparisons, useful for palindromes or equality checks. The main limitations include increased implementation complexity and difficulties in analyzing language classes and performance bounds.

\paragraph{Comparison Between Models:}  
2TQCFA extend 2QCFA in power by enabling comparisons across two tapes. Unlike kTQCFA (which generalize even further), 2TQCFA remain practical for checking mirrored or related substrings. Compared to 1QCFA and CL-1QFA, they are significantly more powerful in terms of language recognition.

\paragraph{Example:}  
To recognize $L = \{ w \# w \}$, a 2TQCFA reads $w$ on the first tape and stores information in the quantum register. It then compares this with the second half of the input on the second tape. Probabilistic subroutines are used to ensure correctness with bounded error \cite{zheng2011two}.

\paragraph{Additional Topics:}  
Variants of multi-tape quantum automata have been studied to explore even richer classes. Open problems include characterizing all non-regular languages recognizable by 2TQCFA with polynomial expected runtime.


\subsubsection{kTQCFA (k-Tape Quantum Finite Automata with Classical States)}

\paragraph{Introduction:}  
The $k$TQCFA model generalizes the two-tape quantum-classical automaton to an arbitrary finite number $k$ of input tapes. This model, proposed in subsequent works building upon the 2TQCFA model \cite{zheng2011two}, enhances the automaton’s ability to process complex language patterns by allowing simultaneous access to multiple strings. Each tape is read by an independent head, all coordinated by a classical control unit and a constant-size quantum register.

\paragraph{Formal Definition:}  
A $k$TQCFA is defined similarly to a 2TQCFA but with $k$ tapes and $k$ input heads. The formal components include:
\[
A = (Q, S, \Sigma, \Theta, \delta, q_0, s_0, S_a, S_r)
\]
with modifications:
\begin{itemize}
    \item $k$ input tapes, each with its own head,
    \item Classical state transitions $\delta$ depend on the symbols read from all $k$ heads and outcomes of quantum operations,
    \item Quantum transitions $\Theta$ may vary based on any combination of input symbols and classical state.
\end{itemize}
The automaton reads the tapes simultaneously and updates its classical and quantum states accordingly, with acceptance determined by reaching a state in $S_a$.

\paragraph{Strings Accepted:}  
Acceptance is based on bounded-error criteria. This model can accept languages requiring coordinated comparisons across multiple strings, such as interleaving or mirror structures, which are beyond the capability of 2QCFA or 2TQCFA.

\paragraph{Sets of Languages Recognized:}  
$k$TQCFA can recognize certain languages that lie outside the class of regular and some context-free languages. It provides a hierarchical extension in power with increasing $k$, where $k=1$ corresponds to 1QCFA and $k=2$ to 2TQCFA \cite{li2015hybrid}.

\paragraph{Closure Properties:}  
Due to increasing complexity with larger $k$, closure properties are less explored. They inherit the limited closure of 2TQCFA but allow more expressive constructions for language families.

\paragraph{Advantages and Limitations:}  
The main advantage of $k$TQCFA is scalability of pattern comparison and cross-tape logic. However, this comes at a cost: managing multiple heads and quantum-classical interactions becomes increasingly complex, both analytically and in potential physical realization.

\paragraph{Comparison Between Models:}  
$k$TQCFA generalize all previously discussed models. While more powerful, they are also less practical for current quantum computing technologies. Unlike 1QCFA or CL-1QFA which are implementable with simpler setups, $k$TQCFA require sophisticated synchronization mechanisms.

\paragraph{Example:}  
A 3TQCFA can accept a language like $L = \{ (x, y, z) \mid x = y = z \}$ by comparing the three inputs simultaneously, performing probabilistic checks using quantum subroutines and classical tracking over each input position.

\paragraph{Additional Topics:}  
Future work may include classification of languages based on minimal $k$ required, complexity of simulations by smaller models, and physical feasibility of multi-tape implementations in quantum automata.
\subsection{Quantum Finite Automata with Counters}
\label{sec:qfa-with-counters}

Quantum finite automata with counters extend the computational power of \gls{qfa} by incorporating classical or quantum counters into the system. These models provide hybrid computational capabilities where quantum state transitions are influenced by the counter value and vice versa, enabling the recognition of certain non-\glspl{reg} with bounded error which classical counterparts fail to recognize.

In the literature, various models of \gls{qfa} with counters have been proposed. This subsection explores the prominent models including \gls{qf1ca}, \gls{2qf1ca}, \gls{1qfkca}, and \gls{rtq1ca}, detailing their structure, properties, capabilities, and limitations based on foundational work such as \cite{bonner2001quantum, kravtsev1999quantum, pani2011empowering, cem2012quantum}.

\subsubsection{Quantum Finite One-Counter Automata (QF1CA)}

A \gls{qf1ca} is a one-way \gls{qfa} that uses a classical counter, capable of incrementing or decrementing its value and testing for zero. This model merges quantum transitions with classical counter logic, providing a new pathway for recognizing languages beyond the regular set \cite{kravtsev1999quantum}.

\paragraph{Formal Definition} 
Formally, a \gls{qf1ca} consists of a finite set of states $Q$, an input alphabet $\Sigma$, a classical counter with values in $\mathbb{Z}$, and a transition function $\delta: Q \times \Sigma \times \{0,1\} \times Q \times \{-1,0,1\} \rightarrow \mathbb{C}$ where $\{0,1\}$ indicates whether the counter is zero or not. The system operates unitarily, with counter updates contingent on the current state and symbol read.

\paragraph{Strings Acceptance} 
\gls{qf1ca} can accept strings using bounded-error probabilistic acceptance. They can recognize non-\glspl{reg} such as $L_1 = \{ w \in \Sigma^* : \text{equal number of 0's and 1's in } x \}$ when augmented with additional structure in the input \cite{bonner2001quantum}.

\paragraph{Sets of Languages Accepted} 
The class of languages accepted by \gls{qf1ca} with bounded error properly includes the class of languages accepted by classical deterministic and probabilistic one-counter automata \cite{bonner2001quantum}.

\paragraph{Closure Properties} 
Closure properties are limited and not thoroughly investigated; however, \gls{qf1ca} do not maintain closure under union or intersection due to non-closure in the classical probabilistic case.

\paragraph{Advantages and Limitations} 
A notable advantage is the ability to recognize certain context-free languages with bounded error. However, limitations stem from counter-based non-reversibility and measurement-induced collapses which reduce robustness.

\paragraph{Comparison} 
Compared to \gls{1qfa} or \gls{mo-1qfa}, \gls{qf1ca} exhibit significantly higher computational power due to the counter's memory augmentation.

\paragraph{Example} 
A \gls{qf1ca} recognizing the language $L_1$ as shown above was constructed in \cite{bonner2001quantum}, showing correct acceptance probabilities distinguishing it from deterministic models.

\paragraph{Additional Topics} 
Current research investigates the influence of quantum control on counter updates and the simulation of classical pushdown automata using counters in hybrid quantum settings.

\subsubsection{Two-Way Quantum Finite One-Counter Automata (2QF1CA)}

\paragraph{Introduction}
\gls{2qf1ca} enhances the \gls{qf1ca} model by allowing two-way head movement on the input tape, significantly expanding computational capabilities. This flexibility enables the automaton to reprocess information with context, analogous to two-way classical finite automata but equipped with quantum transitions and a counter.

\paragraph{Formal Definition}
A \gls{2qf1ca} is defined by a tuple $(Q, \Sigma, \delta, q_0, Q_a, Q_r)$, where $\delta$ maps configurations including direction: $\delta: Q \times \Sigma \times \{0,1\} \times Q \times \{-1,0,1\} \times \{-1,0,1\} \rightarrow \mathbb{C}$. The last component indicates the head movement (-1 for left, 0 for stay, 1 for right), and the counter updates accordingly.

\paragraph{Strings Acceptance}
\gls{2qf1ca} can recognize more complex languages such as $L_2$ from \cite{bonner2001quantum}, composed of multiple $L_1$ segments demarcated by control symbols. These languages are not recognizable by 1QF1CA or classical probabilistic variants.

\paragraph{Sets of Languages Accepted}
These automata can accept languages outside deterministic and probabilistic one-counter automata capabilities, establishing a broader language class, including some context-sensitive languages under bounded error.

\paragraph{Closure Properties}
Closure under complement and intersection is not generally guaranteed due to quantum nondeterminism and measurement dependencies. Formal closure results remain limited.

\paragraph{Advantages and Limitations}
The key strength of \gls{2qf1ca} lies in its bidirectional input scanning which provides significant advantages in language parsing. However, unitarity and interference management become more complex.

\paragraph{Comparison}
Compared to \gls{qf1ca}, the two-way model shows enhanced language recognition at the cost of more complex design and verification.

\paragraph{Example}
Recognition of $L_2$ involving interleaved structures demonstrates the superiority of \gls{2qf1ca} over classical and one-way quantum models as outlined in \cite{bonner2001quantum}.

\paragraph{Additional Topics}
Further topics include automaton minimization, real-time simulation constraints, and efficient quantum algorithm implementation.

\subsubsection{One-Way Quantum Finite $k$-Counter Automata (1QFkCA)}

\paragraph{Introduction}
The \gls{1qfkca} model generalizes the \gls{qf1ca} by including $k$ classical counters. Each counter is independently incremented, decremented, or checked against zero, enabling multi-dimensional memory augmentation in the quantum control logic \cite{cem2012quantum}.

\paragraph{Formal Definition}
Formally, a \gls{1qfkca} is given by a transition function $\delta: Q \times \Sigma \times \{0,1\}^k \times Q \times \{-1,0,1\}^k \rightarrow \mathbb{C}$ with the counter vector defining current zero/non-zero statuses and updates.

\paragraph{Strings Acceptance}
Languages involving multiple numeric relationships, such as $L = \{ a^n b^n c^n \mid n \geq 1 \}$, can be recognized in bounded error by appropriately configured \gls{1qfkca}.

\paragraph{Sets of Languages Accepted}
These automata recognize a subset of context-sensitive languages and are more powerful than all one-counter automata, quantum or classical.

\paragraph{Closure Properties}
Closure under intersection and union becomes feasible with $k$ counters, particularly when structured synchronization is used in parallel counters.

\paragraph{Advantages and Limitations}
Their capability to recognize complex dependencies is advantageous, but the exponential state complexity and entangled counter management are practical limitations.

\paragraph{Comparison}
Compared to \gls{qf1ca}, this model is exponentially more powerful but with higher operational complexity.

\paragraph{Example}
In \cite{cem2012quantum}, a \gls{1qfkca} was shown to recognize the language $a^n b^n c^n$ via three synchronized counters incremented and decremented according to the current segment of the input.

\paragraph{Additional Topics}
Potential topics include quantum counter compression, fault tolerance in counters, and counter sharing protocols in hybrid quantum-classical systems.

\subsubsection{Realtime Quantum One-Counter Automata (rtQ1CA)}

\paragraph{Introduction}
\gls{rtq1ca} represents a restricted subclass of \gls{qf1ca} in which the input head moves strictly right at each step, processing the input in real-time. This model explores trade-offs between real-time operation and computational power \cite{cem2012quantum}.

\paragraph{Formal Definition}
Defined similarly to \gls{qf1ca} but with a strict constraint on the transition direction (right only). The transition function thus omits head direction: $\delta: Q \times \Sigma \times \{0,1\} \times Q \times \{-1,0,1\} \rightarrow \mathbb{C}$.

\paragraph{Strings Acceptance}
Though more limited, \gls{rtq1ca} can still recognize several non-\glspl{reg} with carefully crafted transition amplitudes and counter updates.

\paragraph{Sets of Languages Accepted}
Their accepted languages lie strictly between those of \gls{mo-1qfa} and \gls{qf1ca} due to the real-time restriction.

\paragraph{Closure Properties}
Due to strict real-time behavior and interference effects, closure properties are even more restricted.

\paragraph{Advantages and Limitations}
The main advantage is speed and simplicity in implementation, but at a significant cost to recognition power compared to \gls{qf1ca} or \gls{2qf1ca}.

\paragraph{Comparison}
\gls{rtq1ca} are less powerful than general \gls{qf1ca}, but more powerful than classical real-time automata due to quantum parallelism.

\paragraph{Example}
An \gls{rtq1ca} can probabilistically accept strings with a balanced number of 0's and 1's using only real-time passes and interference.

\paragraph{Additional Topics}
Real-time simulation fidelity and circuit-based implementations of \gls{rtq1ca} models are open areas of study.

\subsection{Generalised Quantum Finite Automata}
\label{sec:generalised-qfa}
\glspl{1gqfa} extend the standard \glspl{qfa} by replacing the usual unitary-based state transitions with the most general physically admissible maps—namely, trace-preserving quantum operations. This modification permits non-unitary evolution, allowing the automata to simulate probabilistic and classical automata while still operating with finite memory. Nevertheless, it has been shown that both the measure-once and measure-many versions of \gls{1gqfa} recognize exactly the \glspl{reg} (with bounded error) \cite{li2012characterizations}.

\subsubsection{\glsentrylong{mo-1gqfa}}
A \gls{mo-1gqfa} generalizes the traditional \gls{mo-1qfa} by allowing each input symbol to trigger a trace-preserving quantum operation (instead of a unitary transformation) on the system. In this model, no measurement is performed during the reading of the input; a single projective measurement is executed only at the end to decide acceptance or rejection \cite{li2012characterizations}.

\paragraph{Formal Definition}
An \gls{mo-1gqfa} is defined as the quintuple
\[
M = \{ \mathcal{H},\Sigma,\rho_0,\{\mathcal{E}_\sigma\}_{\sigma\in\Sigma},P_{acc}\},
\]
where
\begin{itemize}
  \item $\mathcal{H}$ is a finite-dimensional Hilbert space,
  \item $\Sigma$ is a finite input alphabet,
  \item $\rho_0\in D(\mathcal{H})$ is the initial density operator,
  \item For each $\sigma\in\Sigma$, the state transition is given by the trace-preserving quantum operation 
  \[
  \mathcal{E}_\sigma(\rho)=\sum_{k} \mathcal{E}_{\sigma,k}\,\rho\,\mathcal{E}_{\sigma,k}^\dagger,\quad \text{with } \sum_{k} \mathcal{E}_{\sigma,k}^\dagger \mathcal{E}_{\sigma,k}=I,
  \]
  \item $P_{acc}$ is a projector on the accepting subspace of $\mathcal{H}$ (with the complementary projector $P_{rej}=I-P_{acc}$).
\end{itemize}
On an input string $x=\sigma_1\sigma_2\cdots\sigma_n\in\Sigma^*$ the automaton evolves as
\[
\rho_x = \mathcal{E}_{\sigma_n}\circ \mathcal{E}_{\sigma_{n-1}}\circ\cdots\circ \mathcal{E}_{\sigma_1}(\rho_0),
\]
and a final measurement in the basis $\{P_{acc},P_{rej}\}$ is performed. The acceptance probability is defined by
\[
f_M(x)=\mathrm{Tr}\bigl(P_{acc}\,\rho_x\bigr).
\]

\paragraph{Strings Acceptance} 
A string $x\in\Sigma^*$ is accepted by $M$ if the acceptance probability meets the specified criterion. Common acceptance criteria include:
\begin{enumerate}
  \item \textbf{Bounded Error:} There exist a threshold $\lambda\in(0,1]$ and an error margin $\epsilon>0$ such that
  \[
  \begin{aligned}
  f_M(x) &\ge \lambda+\epsilon &&\text{if } x\in L,\\[1mm]
  f_M(x) &\le \lambda-\epsilon &&\text{if } x\notin L.
  \end{aligned}
  \]
  \item \textbf{Cutpoint Acceptance:} $x$ is accepted if $f_M(x)>\lambda$, where $\lambda$ is an isolated cutpoint.
  \item \textbf{Exact Acceptance:} In certain constructions (e.g., when simulating a deterministic finite automaton) one has $f_M(x)=1$ for accepted strings and $f_M(x)=0$ for rejected strings.
\end{enumerate}

\paragraph{Set of Languages Accepted}  
It has been proved that under the bounded error criterion, \gls{mo-1gqfa} recognize precisely the class of \glspl{reg}. That is, for every \gls{reg} there exists an \gls{mo-1gqfa} recognizing it, and every language recognized by an \gls{mo-1gqfa} is regular \cite{li2012characterizations}.

\paragraph{Closure Properties}  
The class of languages recognized by \gls{mo-1gqfa} is closed under several standard operations:
\begin{itemize}
  \item \textbf{Union and Intersection:} By suitable constructions (e.g., via direct sums and tensor products), if $L_1$ and $L_2$ are recognized by \gls{mo-1gqfa} then so are $L_1\cup L_2$ and $L_1\cap L_2$.
  \item \textbf{Complementation:} Replacing $P_{acc}$ with its complement $I-P_{acc}$ yields an automaton for the complement language.
  \item \textbf{Inverse Homomorphism and Concatenation with \glspl{reg}:} These operations preserve the regularity of the language.
\end{itemize}

\paragraph{Summary of Advantages and Limitations}  
The \gls{mo-1gqfa} model is advantageous due to its structural simplicity—requiring only a final measurement—and its ability to simulate classical probabilistic automata exactly via general trace-preserving operations. However, despite the broadened operational framework, its computational power remains confined to recognizing \glspl{reg} (with bounded error). Furthermore, the state minimization problem for \gls{mo-1gqfa} is known to be EXPSPACE-hard \cite{mateus2012complexity}.

\paragraph{Example}  
An example is provided by the simulation of a deterministic finite automaton (DFA) for the language
\[
L=a^*b^*.
\]
Here, one chooses 
\[
\mathcal{H}=\mathrm{span}\{\ket{q_n},\ket{q_n},\dots,\ket{q_n}\},
\]
sets the initial state as $\rho_0=\sum_{i}\pi_i\,\ket{q_n}\bra{q_i}$ (with $\{\pi_i\}$ given by the DFA's initial distribution), and defines each operation $\mathcal{E}_\sigma$ so that for each basis state $\ket{q_n}$,
\[
\mathcal{E}_\sigma\bigl(\ket{q_n}\bra{q_i}\bigr)=\sum_{j}A(\sigma)_{ij}\,\ket{q_n}\bra{q_j},
\]
where $A(\sigma)$ is the stochastic matrix corresponding to the DFA's transition function. The final measurement is performed using 
\[
P_{acc}=\sum_{q_i\in F}\ket{q_n}\bra{q_i},
\]
where $F$ is the set of accepting states. This construction ensures that the acceptance probability $f_M(x)$ replicates the behavior of the DFA \cite{li2012characterizations}.

\paragraph{Additional Topics}  
Further research on \gls{mo-1gqfa} includes the equivalence problem, where necessary and sufficient conditions are derived based on the linear span of the reachable density operators. In addition, advanced state minimization techniques have been developed, reducing the minimization problem to solving systems of polynomial inequalities with an EXPSPACE upper bound \cite{mercer2008lower}.

\subsubsection{\glsentrylong{mm-1gqfa}}
\gls{mm-1gqfa} extend the \gls{mo-1gqfa} model by performing a measurement after processing each input symbol. In this model, after each trace-preserving quantum operation corresponding to a symbol, a projective measurement is executed that partitions the state space into three mutually orthogonal subspaces—namely, the accepting subspace, the rejecting subspace, and the non-halting subspace. If the outcome lies in the accepting or rejecting subspace, the computation halts immediately; otherwise, it continues with the next symbol \cite{li2012characterizations}.

\paragraph{Formal Definition} 
An \gls{mm-1gqfa} is defined as the 6-tuple
\[
M = \{ \mathcal{H},\Sigma,\rho_0,\{\mathcal{E}_\sigma\}_{\sigma\in\Sigma\cup\{\cent,\$\}},\mathcal{H}_{acc},\mathcal{H}_{rej}\},
\]
where
\begin{itemize}
  \item $\mathcal{H}$ is a finite-dimensional Hilbert space that decomposes as
  \[
  \mathcal{H}=\mathcal{H}_{acc}\oplus \mathcal{H}_{rej}\oplus \mathcal{H}_{non},
  \]
  with $\mathcal{H}_{acc}$ and $\mathcal{H}_{rej}$ denoting the accepting and rejecting subspaces, and $\mathcal{H}_{non}$ the non-halting subspace;
  \item $\Sigma$ is a finite input alphabet, and the symbols $\cent$ and $\$$ serve as the left and right end-markers, respectively;
  \item $\rho_0\in D(\mathcal{H})$ is the initial state with $\mathrm{supp}(\rho_0)\subseteq \mathcal{H}_{non}$;
  \item For each $\sigma\in\Sigma\cup\{\cent,\$\}$, the state transition is given by the trace-preserving quantum operation 
  \[
  \mathcal{E}_\sigma(\rho)=\sum_{k} \mathcal{E}_{\sigma,k}\,\rho\,\mathcal{E}_{\sigma,k}^\dagger,\quad \text{with } \sum_{k} \mathcal{E}_{\sigma,k}^\dagger \mathcal{E}_{\sigma,k}=I;
  \]
  \item After each $\mathcal{E}_\sigma$, a projective measurement is performed with respect to the orthogonal projectors $\{P_{non},P_{acc},P_{rej}\}$ onto $\mathcal{H}_{non}$, $\mathcal{H}_{acc}$, and $\mathcal{H}_{rej}$, respectively.
\end{itemize}
For an input string $x\in\Sigma^*$ (presented as $\cent\,x\,\$$), the automaton processes each symbol sequentially. If, at any step, the measurement projects onto $\mathcal{H}_{acc}$ (or $\mathcal{H}_{rej}$), the computation halts with acceptance (or rejection). Otherwise, if the outcome is in $\mathcal{H}_{non}$, the automaton continues processing the next symbol.

\paragraph{Strings Acceptance}  
The acceptance of an input string $x$ is defined by the cumulative probability that the automaton halts in an accepting configuration. The common acceptance criteria include:
\begin{enumerate}
  \item \textbf{Bounded Error:} There exist $\lambda\in(0,1]$ and $\epsilon>0$ such that
  \[
  \begin{aligned}
  \text{if } x\in L,&\quad \text{cumulative acceptance probability} \ge \lambda+\epsilon,\\[1mm]
  \text{if } x\notin L,&\quad \text{cumulative acceptance probability} \le \lambda-\epsilon.
  \end{aligned}
  \]
  \item \textbf{Cutpoint/Exact Acceptance:} As in the \gls{mo-1gqfa} model, acceptance may also be defined via an isolated cutpoint or by requiring exact acceptance.
\end{enumerate}

\paragraph{Set of Languages Accepted} 
It has been established that \gls{mm-1gqfa}, despite the intermediate measurements after each symbol, recognize exactly the class of \glspl{reg} (with bounded error). Thus, the frequency of measurements does not extend the language recognition power beyond that of \gls{mo-1gqfa} \cite{li2012characterizations}.

\paragraph{Closure Properties}  
\gls{mm-1gqfa} are closed under standard operations. In particular, if $L_1$ and $L_2$ are recognized by \gls{mm-1gqfa} then:
\begin{itemize}
  \item They are closed under union and intersection (by appropriate constructions using direct sums or tensor products),
  \item They are closed under complementation (by swapping the roles of $\mathcal{H}_{acc}$ and $\mathcal{H}_{rej}$),
  \item And they are closed under other operations such as inverse homomorphism.
\end{itemize}

\paragraph{Summary of Advantages and Limitations} 
The \gls{mm-1gqfa} model offers the flexibility of making intermediate measurements, which may simplify the design of some automata. However, like \gls{mo-1gqfa}, its computational power remains limited to \glspl{reg} under the bounded error regime. Furthermore, the state minimization problem for \gls{mm-1gqfa} is EXPSPACE-hard \cite{mateus2012complexity}.

\paragraph{Example}  
For example, consider an \gls{mm-1gqfa} designed to recognize 
\[
L=\{w\in\{a,b\}^* \mid \text{the last symbol of }w\text{ is }a\}.
\]
In this automaton, after each input symbol the machine performs a measurement. If a measurement outcome projects onto $\mathcal{H}_{acc}$ (indicating that the current configuration is accepting) and no previous measurement forced a rejection, the automaton eventually halts with acceptance. This construction guarantees that the cumulative acceptance probability meets the bounded error condition exactly when the input ends with an $a$ \cite{li2012characterizations}.

\paragraph{Additional Topics}  
Recent research on \gls{1gqfa} has addressed the equivalence problem, providing necessary and sufficient conditions based on the linear span of the reachable density operators. Moreover, advanced state minimization techniques have been developed, reducing the minimization problem to solving systems of polynomial inequalities with an EXPSPACE upper bound \cite{mercer2008lower}. Future directions include exploring further generalizations and their potential applications in modeling noisy quantum systems.


%TODO: missing comparisons

\subsection{Interactive Automata Based on Quantum Interactive Proof Systems}
\label{sec:interactive-automata}

%%%%%%%%%%%%%%%%%%%%%%%%%%%%%%%%%%
% Introduction
%%%%%%%%%%%%%%%%%%%%%%%%%%%%%%%%%%
Interactive automata based on quantum interactive proof systems offer a striking demonstration of how even extremely resource‐limited verifiers—modeled by quantum finite automata (qfa’s)—can, through interaction with a powerful prover, recognize nontrivial languages. Two principal models have been developed in this area:
  
\begin{itemize}
  \item \textbf{Quantum Interactive Proof (\gls{qip}) systems}, in which the verifier’s internal moves remain hidden (private-coin), and
  \item \textbf{Quantum Arthur–Merlin (\gls{qam}) systems}, where the verifier publicly announces his next move (public-coin).
\end{itemize}

In these models, the verifier is typically a two-way qfa, though one-way variants have also been considered. The seminal works by Nishimura and Yamakami \cite{nishimura2009application, nishimura2015interactive} and Zheng, Qiu, and Gruska \cite{zheng2015power} have established detailed protocols and complexity separations that reveal the potential of interactive proofs even when the verifier possesses only finite-dimensional quantum memory.

%%%%%%%%%%%%%%%%%%%%%%%%%%%%%%%%%%
% Formal Definition
%%%%%%%%%%%%%%%%%%%%%%%%%%%%%%%%%%
\paragraph{Formal Definition.}
A general \gls{qip} system with a qfa verifier is defined as a pair $(P,V)$, where:

\textbf{Verifier.} The verifier $V$ is given by
\[
V = \bigl(Q,\, \Sigma \cup \{\cent,\$\},\, \Gamma,\, \delta,\, q_0,\, Q_{acc},\, Q_{rej}\bigr),
\]
with the following components:
\begin{itemize}
  \item $Q$ is a finite set of inner states partitioned as $Q = Q_{non} \cup Q_{acc} \cup Q_{rej}$;
  \item $\Sigma$ is the input alphabet, and $\cent$ and $\$$ denote the left and right endmarkers, respectively;
  \item $\Gamma$ is the communication alphabet;
  \item $\delta$ is the transition function. For each configuration $(q,\sigma,\gamma)$, the verifier changes its state, updates the tape head position (with moves in $\{-1,0,1\}$), and writes a new symbol in the communication cell according to complex amplitudes given by $\delta(q,\sigma,\gamma,q',\gamma',d)$;
  \item $q_0\in Q$ is the initial state;
  \item $Q_{acc}$ and $Q_{rej}$ are the sets of halting (accepting and rejecting) states.
\end{itemize}
The verifier’s overall Hilbert space, denoted by $\mathit{\mathcal{H}}_V$, is spanned by basis states of the form
\[
\ket{q, k, \gamma},\quad q\in Q,\; k\in \mathbb{Z},\; \gamma\in\Gamma.
\]

\textbf{Prover.} The prover $P$ is specified by a family of unitary operators
\[
\{U_{x,P,i}\}_{i\ge1},
\]
acting on the prover’s private Hilbert space $\mathit{\mathcal{H}}_P$. In some variants (denoted by the restriction 〈c-prover〉), the prover’s unitaries are required to have only 0–1 entries, effectively making the prover deterministic.

\paragraph{\gls{qam} Systems.} In the \gls{qam} variant, the verifier is additionally required to announce his next move via the communication cell, rendering the system a public-coin protocol. Thus, while the basic structure of $(P,V)$ remains the same, a \gls{qam} system is denoted as
\[
\text{\gls{qam}}(\langle\text{restriction}\rangle) \quad \text{or} \quad \text{\gls{qip}}(\text{public}),
\]
and the transition function $\delta$ is designed so that, after each move, the pair $(q',\gamma',d)$ is revealed to the prover.

%%%%%%%%%%%%%%%%%%%%%%%%%%%%%%%%%%
% Strings Acceptance
%%%%%%%%%%%%%%%%%%%%%%%%%%%%%%%%%%
\paragraph{Strings Acceptance.}
An interactive proof system $(P,V)$ accepts an input string $x\in\Sigma^*$ if, after a prescribed sequence of interaction rounds, the verifier eventually performs a halting measurement that yields an accepting configuration with high probability. Formally, the system recognizes a language $L\subseteq\Sigma^*$ if the following conditions hold:
\begin{itemize}
  \item \textbf{Completeness:} For every $x\in L$, there exists a prover strategy $P$ such that the verifier accepts $x$ with probability at least $1-\epsilon$, where $\epsilon<1/2$.
  \item \textbf{Soundness:} For every $x\notin L$, for every prover strategy $P^*$, the verifier rejects $x$ with probability at least $1-\epsilon$.
\end{itemize}
Variants of acceptance include definitions via an isolated cutpoint or exact acceptance (i.e., acceptance with probability 1), but the bounded-error model is standard.

%%%%%%%%%%%%%%%%%%%%%%%%%%%%%%%%%%
% Set of Languages Accepted
%%%%%%%%%%%%%%%%%%%%%%%%%%%%%%%%%%
\paragraph{Set of Languages Accepted.}
The language recognition power of these interactive systems depends on the verifier’s model and the nature of the interaction:
\begin{itemize}
  \item When the verifier is a \textbf{1qfa} (one-way qfa), it has been shown that
  \[
  \text{\gls{qip}(1qfa)} = \text{REG},
  \]
  meaning that even with interaction the system recognizes only the regular languages \cite{nishimura2009application}.
  \item In contrast, when the verifier is a \textbf{2qfa} (two-way qfa), the interactive proof system can recognize languages that are not regular. For example, several protocols with 2qfa verifiers operating in expected polynomial time have been shown to outperform classical AM systems with 2pfa verifiers \cite{zheng2015power, nishimura2015interactive}.
  \item In the public-coin (\gls{qam}) variant, where the verifier reveals its next move, the additional information sometimes further enhances the system’s power, and comparisons with classical Arthur–Merlin systems have been established.
\end{itemize}

%%%%%%%%%%%%%%%%%%%%%%%%%%%%%%%%%%
% Closure Properties
%%%%%%%%%%%%%%%%%%%%%%%%%%%%%%%%%%
\paragraph{Closure Properties.}
The language classes defined by \gls{qip} and \gls{qam} systems exhibit robust closure properties:
\begin{itemize}
  \item They are closed under \emph{union} and \emph{intersection}, typically via parallel composition (using direct sums or tensor products).
  \item They are closed under \emph{complementation} (by exchanging the roles of $Q_{acc}$ and $Q_{rej}$ in the verifier’s design).
  \item They are also closed under other operations such as inverse homomorphism.
\end{itemize}
These properties are established through constructions that combine multiple protocols while preserving the bounded-error guarantees.

%%%%%%%%%%%%%%%%%%%%%%%%%%%%%%%%%%
% Summary of Advantages and Limitations
%%%%%%%%%%%%%%%%%%%%%%%%%%%%%%%%%%
\paragraph{Summary of Advantages and Limitations.}
Interactive proof systems with qfa verifiers offer several compelling advantages:
\begin{itemize}
  \item \textbf{Finite Quantum Resources:} The verifier operates with a finite-dimensional quantum system, making the model realistic for devices with limited quantum memory.
  \item \textbf{Enhanced Recognition via Interaction:} Even though a standalone qfa (especially a 1qfa) may recognize only regular languages, interaction with a powerful prover can significantly boost the verifier’s ability, particularly when using two-way qfa verifiers.
  \item \textbf{Flexibility through Protocol Variants:} By varying whether the system is a \gls{qip} (private-coin) or \gls{qam} (public-coin) system, and by imposing restrictions on the prover (quantum vs. classical), one can fine-tune the computational power and compare with classical interactive proof systems.
\end{itemize}
However, there are also limitations:
\begin{itemize}
  \item \textbf{Limited Power of One-Way Verifiers:} When restricted to one-way qfa verifiers, the system’s power is confined to the regular languages.
  \item \textbf{Potentially High Interaction Complexity:} Protocols with two-way qfa verifiers can require a large (sometimes exponential) number of rounds or running time.
  \item \textbf{Technical Complexity:} The design and analysis of these interactive protocols are intricate, involving careful balancing of quantum and classical information.
\end{itemize}

%%%%%%%%%%%%%%%%%%%%%%%%%%%%%%%%%%
% Example
%%%%%%%%%%%%%%%%%%%%%%%%%%%%%%%%%%
\paragraph{Example.}
An illustrative example is the \gls{qip} protocol for the language
\[
\mathtt{Pal\#} = \{x\#x^R \mid x\in\{0,1\}^*\},
\]
which comprises even-length palindromes separated by a delimiter. In the protocol described in \cite{nishimura2015interactive}, the verifier (modeled as a 2qfa) interacts with a quantum prover in the following way:
\begin{enumerate}
  \item The verifier scans the input (framed by the endmarkers $\cent$ and $\$$) and, based on its transition function $\delta$, generates a superposition reflecting potential midpoints.
  \item Through a sequence of rounds, the verifier requests the prover to indicate the position of the center. In the \gls{qam} variant, the verifier publicly announces his next move to assist the prover.
  \item Finally, the verifier applies a \gls{qft} to consolidate the information and performs a measurement. If the input is indeed of the form $x\#x^R$, the verifier accepts with high probability; otherwise, it rejects.
\end{enumerate}
This example clearly demonstrates how interaction compensates for the verifier's limited memory, enabling recognition of a nontrivial language.

\paragraph{Additional Topics.}
Several open problems and future research directions emerge from this line of work:
\begin{itemize}
  \item \textbf{Round Complexity:} How does limiting the number of interaction rounds (e.g., as in \gls{qip}\#($k$)) affect the recognition power and efficiency?
  \item \textbf{Prover Restrictions:} What are the precise differences in computational power when the prover is restricted to classical behavior (〈c-prover〉) versus full quantum capability?
  \item \textbf{Public vs. Private Protocols:} Further analysis is needed to understand the trade-offs between \gls{qip} (private-coin) and \gls{qam} (public-coin) systems.
  \item \textbf{Resource-Bounded Protocols:} Tightening the upper and lower bounds on running time and state complexity for these systems remains a challenging task.
\end{itemize}
These issues continue to be central to the ongoing exploration of the interplay between interaction and quantum finite automata.

%TODO: add comparisons and other minor models
\begin{itemize}
  \item \textbf{Limited-Round Interactive Systems (QIP\#($k$)):} In some works (e.g., by Nishimura and Yamakami), the number of interaction rounds is explicitly bounded. These models, often denoted by QIP\#($k$) (with $k$ indicating the maximum number of rounds), allow a more refined complexity classification of interactive protocols.
  
  \item \textbf{Interactive Proof Systems with Semi-Quantum Verifiers:} Another significant model is the one in which the verifier is not a full-fledged quantum finite automaton but a \emph{semi-quantum} two-way finite automaton (2QCFA). In such systems—as studied, for instance, by Zheng, Qiu, and Gruska—the verifier possesses both classical and quantum states, using limited quantum resources alongside classical processing. These systems (sometimes denoted QAM(2QCFA) in the public-coin setting) have been shown to recognize languages beyond those recognizable by two-way probabilistic finite automata.
  
  \item \textbf{Variants Based on Prover Restrictions:} Some works also examine the effect of restricting the prover to \emph{classical} behavior (i.e., using only 0–1 unitary operators, sometimes denoted by the restriction 〈c-prover〉). This yields interactive models that can be compared with their fully quantum counterparts.
\end{itemize}
\subsection{Multi-letter Models}
\label{sec:multi-letter-qfa}

Multiletter Quantum Finite Automata (\gls{ml-qfa}) are a generalization of traditional quantum finite automata, where transitions depend on multiple letters read from the input, rather than a single letter. This allows them to capture more complex patterns in the input string.

\subsubsection{\glsentryfull{ml-qfa}}

Multiletter QFA were introduced to extend the capability of classical and quantum models by applying unitary operations based on the last $k$ letters read rather than just one. This enables them to recognize languages outside the reach of traditional measure-once or measure-many 1QFA models \cite{belovs2007multi}.

\paragraph{Formal Definition}
A $k$-letter \gls{ml-qfa} is a 5-tuple $A = (Q, Q_{acc}, \ket{\psi_0}, \Sigma, \mu')$, where:
\begin{itemize}
    \item $Q$ is a finite set of states,
    \item $Q_{acc} \subseteq Q$ is the set of accepting states,
    \item $\ket{\psi_0}$ is the initial superposition of states with unit norm,
    \item $\Sigma$ is the input alphabet,
    \item $\mu': (\Sigma \cup \{\Lambda\})^k \to U(\mathbb{C}^n)$ is a function assigning a unitary operator to every $k$-tuple of symbols.
\end{itemize}
The transition applies the unitary associated with the last $k$ letters read. For input $\omega = x_1 x_2 \dots x_n$, the computation evolves through unitary applications as specified in Eq. (1) and acceptance is determined using a projection operator $P_{acc}$ \cite{qiu2009hierarchy}.

\paragraph{Strings Acceptance}
Acceptance is defined by exact probability, cutpoint (strict or non-strict), or bounded-error depending on how $P_A(\omega) = \| P_{acc} U_\omega \ket{\psi_0} \|^2$ compares to a threshold $\lambda$.

\paragraph{Sets of Languages Accepted}
ML-QFA can recognize a proper superset of regular languages compared to MO-1QFA and MM-1QFA. Notably, they can recognize the language $(a + b)^*a$ which is not recognizable by MO-1QFA or MM-1QFA \cite{belovs2007multi}.

\paragraph{Closure Properties}
The class of languages recognized by ML-QFA is not closed under union, intersection, or complement, especially under non-strict cutpoint semantics \cite{qiu2009hierarchy}.

\paragraph{Advantages and Limitations}
ML-QFA demonstrate higher computational power with fewer states in certain scenarios. However, equivalence and minimization are complex and computationally hard. For non-strict cutpoints, the emptiness problem is undecidable \cite{qiu2008decidability}.

\paragraph{Comparison}
ML-QFA are more expressive than MO-1QFA and MM-1QFA under the same acceptance criteria, but less so than two-way or general quantum automata with additional memory models \cite{qiu2011multi}.

\paragraph{Example}
The language $(a + b)^*a$ is a canonical example recognized by 2-letter ML-QFA, using a transition function dependent on the last two characters read \cite{belovs2007multi}.

\paragraph{Additional Topics}
Research is ongoing in determining exact hierarchies, equivalence testing, and applying ML-QFA in quantum protocol verification \cite{lin2012equivalence, qiu2008decidability}.

\subsubsection{Multiletter Measure-Many Quantum Finite Automata (ML-MMQFA)}

\paragraph{Introduction}
\gls{ml-mmqfa} combine the power of measure-many acceptance strategies with multiletter transitions, extending both classical MM-1QFA and ML-QFA. In this model, a measurement is performed after each quantum evolution step, but the evolution itself depends on the last $k$ letters read. This hybrid structure allows ML-MMQFA to accept more complex languages than ML-QFA or MM-1QFA alone \cite{lin2012equivalence}.

\paragraph{Formal Definition}
A $k$-letter \gls{ml-mmqfa} is a 7-tuple $A = (Q, Q_{acc}, Q_{rej}, \ket{\psi_0}, \Sigma, \mu', \mathcal{O})$, where:
\begin{itemize}
    \item $Q$ is the finite set of states,
    \item $Q_{acc} \subset Q$ is the set of accepting states,
    \item $Q_{rej} \subset Q$ is the set of rejecting states with $Q_{acc} \cap Q_{rej} = \emptyset$,
    \item $\ket{\psi_0}$ is the initial state,
    \item $\Sigma$ is the finite input alphabet,
    \item $\mu': (\Sigma \cup \{\Lambda, \text{\textsterling}, \$\})^k \to U(\mathbb{C}^n)$ assigns a unitary matrix to each $k$-letter word,
    \item $\mathcal{O} = \{P_{acc}, P_{rej}, P_{non}\}$ is a projective measurement partitioning the Hilbert space based on $Q_{acc}$, $Q_{rej}$, and the non-halting subspace.
\end{itemize}
Computation starts with the end-marked input £$x_1x_2 \dots x_n\$$, and proceeds by interleaving unitary evolutions with projective measurements.

\paragraph{Strings Acceptance}
For a word $\omega = x_1 x_2 \dots x_n$, the acceptance probability is:
\[
P_A(\omega) = \sum_{i=0}^{n+1} \left\| P_{acc} U_{x_i} \left( \prod_{j=i-1}^{0} P_{non} U_{x_j} \right) \ket{\psi_0} \right\|^2
\]
A string is accepted if this probability exceeds a cutpoint (for probabilistic acceptance), or is exactly 1 (for exact acceptance). Both strict and non-strict cutpoints are considered \cite{lin2012equivalence}.

\paragraph{Sets of Languages Accepted}
ML-MMQFA can recognize languages beyond the regular class. They accept some languages that are not accepted by classical QFA or even standard MM-1QFA, especially under non-strict cutpoint semantics \cite{qiu2009hierarchy, lin2012equivalence}.

\paragraph{Closure Properties}
The set of languages recognized by ML-MMQFA is not closed under union, intersection, or complementation for general acceptance modes. Notably, these properties depend heavily on the acceptance criteria used (bounded-error, cutpoint, etc.) \cite{qiu2009hierarchy}.

\paragraph{Advantages and Limitations}
ML-MMQFA are more powerful than both ML-QFA and MM-1QFA. However, they inherit the undecidability of the emptiness and equivalence problems for non-strict and strict cutpoint semantics \cite{qiu2008decidability, lin2012equivalence}. Their expressive power comes at the cost of analytical and implementation complexity.

\paragraph{Comparison} 
ML-MMQFA strictly subsume ML-QFA under the same input size and acceptance criteria. They are not comparable in power with general QFA models that allow additional memory or two-way movement. Compared to MM-1QFA, ML-MMQFA can accept non-stochastic languages \cite{qiu2009hierarchy}.

\paragraph{Example}
An ML-MMQFA with 2-letter transitions and intermediate measurements can accept the language $(a+b)^*a$ with higher robustness to probabilistic acceptance thresholds than an ML-QFA \cite{belovs2007multi}.

\paragraph{Additional Topics}
Further work focuses on state complexity, succinctness, and simulation algorithms between different classes. The diagonal sum construction plays a key role in analyzing equivalence and decidability \cite{lin2012equivalence}.



\subsubsection{\glsentryfull{ml-revqfa}}
%TODO: https://link.springer.com/chapter/10.1007/978-3-540-73208-2_9 review
\paragraph{Introduction}
\gls{ml-revqfa} extend the multiletter framework by imposing reversibility constraints on the quantum evolution. These automata are designed so that each computation step is invertible, maintaining the core principle of reversibility from quantum mechanics and enhancing coherence preservation \cite{belovs2007multi}.

\paragraph{Formal Definition}
A $k$-letter \gls{ml-revqfa} is a special case of $k$-letter \gls{ml-qfa} where each unitary transformation $\mu'(\omega)$ satisfies the additional constraint that $\mu'(\omega)^{-1} = \mu'(\omega)^\dagger$ for all $\omega \in (\Sigma \cup \{\Lambda\})^k$, and the set of such transformations forms a group under composition.

The automaton is defined as a 5-tuple $A = (Q, Q_{acc}, \ket{\psi_0}, \Sigma, \mu')$, with $\mu'$ restricted to reversible unitary matrices.

\paragraph{Strings Acceptance}
Acceptance is defined as in ML-QFA using a projector $P_{acc}$. For an input string $\omega$, the acceptance probability is:
\[
P_A(\omega) = \| P_{acc} U_{\omega} \ket{\psi_0} \|^2
\]
with $U_{\omega}$ computed from the sequence of reversible unitaries associated with the $k$-letter substrings \cite{belovs2007multi}.

\paragraph{Sets of Languages Accepted}
ML-revQFA are strictly more limited than general ML-QFA due to their reversibility constraint. They can recognize a subset of regular languages and do not accept all regular languages with bounded error, particularly under exact acceptance criteria.

\paragraph{Closure Properties}
The class of languages recognized by ML-revQFA is not closed under union or complement. Reversibility limits the computational power of these models in comparison with more general multiletter QFA models \cite{belovs2007multi}.

\paragraph{Advantages and Limitations}
Reversible automata are appealing for quantum computing implementations due to better coherence and energy efficiency. However, their expressiveness is constrained. ML-revQFA cannot recognize some simple regular languages that non-reversible ML-QFA can handle \cite{belovs2007multi}.

\paragraph{Comparison}
ML-revQFA are less powerful than both ML-QFA and ML-MMQFA. They are closely related to reversible classical automata and group automata. Compared to non-reversible models, they generally require more states or cannot recognize the same languages under equivalent semantics \cite{belovs2007multi}.

\paragraph{Example}
It was shown that ML-revQFA cannot accept the language $(a+b)^*a$ even when using multiletter transitions, while a general ML-QFA can accept it with bounded error \cite{belovs2007multi}.

\paragraph{Additional Topics}
Future research may explore connections with quantum error correction, fault-tolerant reversible computing, and applications in energy-efficient quantum hardware design.




\section{Other Models of Quantum Finite Automata}
\label{sec:other-models}

Beyond the core models of quantum finite automata discussed in the previous sections, the literature also presents several alternative models that explore different computational paradigms, theoretical extensions, or enhancements. While these models are less prominent or less widely used, they offer valuable insights into the boundaries and variations of quantum automata theory.

In this section, we provide a concise overview of some notable variants. Each model is briefly introduced with its main characteristics and distinguishing features, along with references to the original works in which they were proposed. Readers interested in further details are encouraged to consult the cited articles.

\subsection{Quantum Turing Machines} 
The \gls{qtm} is the quantum analog of a classical Turing machine, featuring an infinite tape and a moving head with quantum states and unitary transitions. It was first proposed by Deutsch in 1985 as a general model of quantum computation \cite{deutsch1985quantum}. A QTM can implement any quantum algorithm and is computationally equivalent to the quantum circuit model (Yao proved that any QTM can be efficiently simulated by quantum circuits and vice versa \cite{yao1993quantum}). Unlike finite automata models, the QTM is not limited to \glspl{reg} - it has unbounded memory and can recognize non-\glspl{reg} - but this generality comes at the cost of a much more complex machine description. In practice, QTMs serve mostly as a theoretical cornerstone since simpler models (like quantum circuits) are used for designing algorithms, yet the QTM remains important for defining quantum complexity classes and formalizing the Church-Turing principle in the quantum realm.

\subsection{Latvian Quantum Finite Automata} 
The term \gls{lqfa} refers to the one-way quantum finite automaton model introduced by Ambainis and Freivalds (who are Latvian) in 1998 \cite{ambainis19981}. This model is essentially the \emph{measure-once} 1QFA: the machine's state evolves unitarily as it reads the input, and only after reaching the end of the input is a single projective measurement performed to decide acceptance. (In contrast, the earlier \gls{qfa} model by Kondacs and Watrous allowed measurements after each step.) The Latvian 1QFA demonstrated that even with a single end-of-input measurement, a quantum automaton can recognize certain \glspl{reg} with exponentially fewer states than any equivalent deterministic automaton. However, like other 1QFAs, it cannot recognize all \glspl{reg}. The LQFA is historically significant as one of the first quantum automata models, and its state-efficiency advantages and limitations were studied in subsequent works.

\subsection{\texorpdfstring{$l$}{l}-valued Finite Automata} 
An \gls{l-vfa} is an automaton model based on multi-valued logic (in particular, on quantum logic), rather than probabilistic or binary state transitions. This model was explored by Ying (2000) and was later formalized and extended by Qiu in 2007 as a “logical” approach to quantum computation \cite{qiu2007automata}. In an l-VFA, the transition function is not strictly deterministic or probabilistic - instead, each transition from a state $p$ to a state $q$ on an input symbol $\sigma$ is assigned a truth-value from a complete orthomodular lattice $L$. Intuitively, $\delta(p,\sigma,q)$ may be 0, 1, or some intermediate truth-value in $L$. A string is accepted by an l-VFA if the aggregated truth-value of all paths leading to an accepting state evaluates to 1 in the lattice sense. This construction generalizes classical finite automata and provides a way to apply quantum logic to automata theory.

\subsection{\texorpdfstring{$l$}{l}-valued Pushdown Automata} 
The \gls{l-vpda} extends the idea of an l-VFA by adding a pushdown stack, thus enabling recognition of some non-\glspl{reg} within the $l$-valued logic framework. This model was introduced alongside l-VFAs by Qiu in 2007 \cite{qiu2007automata} as part of the effort to build automata theory on quantum logic. An l-VPDA operates similarly to a classical pushdown automaton, but its state transitions and stack operations carry truth-values in a lattice $L$ instead of deterministic outcomes.

\subsection{Quantum Automata with Advice} 
\glsentrylong{qfa-adv} are variants of 1QFA that are supplemented with an additional input - an advice string or quantum state - that depends only on the input length $n$ and is provided to the automaton to improve its computation. This idea was studied by Yamakami (2014) \cite{yamakami2014one}. In his model, the machine can utilize a pre-prepared quantum advice state during its computation, allowing for potentially improved computational power while still remaining weaker than full quantum Turing machines.

\subsection{Enhanced Quantum Finite Automata} 
\gls{e-1qfa} is a variant of the one-way QFA where the machine's state can be measured after each symbol is read, rather than restricting measurement to occur only at the end of the input. This model was introduced by Nayak \cite{nayak1999optimal} and studied further by Lin \cite{lin2012another}. It allows the computation to dynamically adapt based on partial measurement outcomes, making it slightly more powerful than traditional one-way QFAs in certain contexts.

\subsection{Postselection Quantum Finite Automata} 
\gls{pqfa} is a theoretical model that augments a quantum finite automaton with the power of \emph{postselection} - the ability to conditionally proceed based on a desired measurement outcome. This powerful but unphysical feature was used to explore computational limits, and the model was studied in depth by Scegulnaja-Dubrovska et al. \cite{scegulnaja2010postselection} and originally proposed in the context of quantum complexity by Aaronson \cite{aaronson2005quantum}.

\subsection{\texorpdfstring{$\omega$}{omega} Quantum Finite Automata}
\gls{omega-qfa} extend quantum finite automata to operate on infinite input strings. Bhatia and Kumar (2019) introduced several formal models with different acceptance conditions like Büchi, Rabin, and Streett \cite{bhatia2019quantum}. These models are important for exploring quantum computation over streams or continuous inputs and show intriguing differences from their classical counterparts.

\subsection{Promise Problems and Quantum Finite Automata}

Promise problems are a generalization of language recognition where an automaton is required to correctly classify inputs from two disjoint sets: the “yes” instances and the “no” instances. This relaxed setting provides a useful framework for analyzing subtle distinctions in computational power, especially when comparing classical and quantum models.

\gls{qfa} have demonstrated significant advantages in the context of promise problems. These models are often more state-efficient or capable of solving problems that classical automata cannot handle with bounded error. One notable study by Zheng et al.\ \cite{zheng2013state} investigates the \gls{2qcfa} model and demonstrates its exponential state succinctness over classical counterparts for families of promise problems. For example, they construct a 2QCFA that solves a problem with constant quantum memory and logarithmic classical memory, whereas equivalent classical automata require exponentially more states.

Other works explore theoretical implications of quantum advantages under promises. Rashid and Yakaryilmaz \cite{rashid2014implications} analyze how quantum automata solving promise problems can relate to foundational concepts like contextuality in quantum theory. Bianchi et al.\ \cite{bianchi2014complexity} examine the computational complexity of promise problems across classical and quantum finite automata, identifying specific contexts where quantum models are strictly more efficient. Gruska et al.\ \cite{gruska2015potential} further study promise problems under exact acceptance and show that \gls{qfa} can solve certain structured promise problems with significantly fewer states than their classical counterparts.

Overall, the study of promise problems has emerged as a rich area to highlight the computational advantages of quantum models, often revealing separations that are not observable in standard language recognition settings.
