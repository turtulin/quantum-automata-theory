\subsection{One-Way Quantum Finite Automata}
\label{sec:1-way-qfa}

\gls{1qfa} are the quantum analog of classical one-way finite automata where the input tape is read from left to right without revisiting symbols. They provide a model for finite quantum computation that is more restricted than \glspl{2qfa} but often simpler to implement and analyse.

Among the most studied variants of \gls{1qfa} are the \gls{mo-1qfa} and the \gls{mm-1qfa}, which are addressed in detail in Section~\ref{sec:detailed-models}. These automata differ primarily in the timing of their measurements: \gls{mo-1qfa} perform a single measurement at the end of the computation, while \gls{mm-1qfa} perform measurements after reading each input symbol.

Beyond these foundational models, several other types of \gls{1qfa} have been proposed, each offering unique computational perspectives or enhancements. We present a comprehensive overview of such models, each in its own subsubsection.

\subsubsection{\glsentryfull{mon-1qfa}}
The \gls{mon-1qfa} is a special model in which only measurement operations are used for computation, with no intermediate unitary evolutions. This model simplifies quantum computation by relying solely on projective measurements.

\begin{definition}[\gls{mon-1qfa}]
A \gls{mon-1qfa} is defined as a tuple \( A = (Q, \Sigma, \rho_0, \{P_{\sigma}\}_{\sigma \in \Sigma}, Q_{acc}) \) where:
\begin{itemize}
    \item \( Q \) is a finite set of states,
    \item \( \Sigma \) is a finite input alphabet,
    \item \( \rho_0 \) is the initial quantum state (density matrix),
    \item \( P_{\sigma} \) is the measurement operator for each input symbol \( \sigma \),
    \item \( Q_{acc} \subseteq Q \) is the set of accepting states.
\end{itemize}
\end{definition}

\paragraph{Strings Acceptance}
Acceptance is determined by applying the appropriate projective measurement after each symbol and measuring the final state. Acceptance can be defined with a bounded-error threshold.

\paragraph{Sets of Languages Accepted}
The class of languages accepted by \gls{mon-1qfa} is strictly less powerful than the class of \glspl{reg}. It corresponds to a particular class of \glspl{reg} known as literally idempotent piecewise testable languages \cite{bertoni2010trace}.

\paragraph{Closure Properties}
The languages recognised by \gls{mon-1qfa} are not closed under union or complementation \cite{bertoni2010trace}.

\paragraph{Advantages and Limitations}
They offer a hardware-friendly model due to the absence of unitaries, but are strictly less powerful than \gls{mo-1qfa} and \gls{mm-1qfa}.

\paragraph{Comparison}
Compared to \gls{mo-1qfa} and \gls{mm-1qfa}, \gls{mon-1qfa} are less powerful due to the absence of unitary evolution.

\paragraph{Additional Topics}
Measure-only models relate to trace monoids with idempotent generators and have been used in language algebraic characterizations \cite{comin2013extended, bertoni2010trace}.

\begin{example}
An example is the language of all strings over \{a, b\} with an even number of a's. It can be recognised by a suitable choice of projective measurements.
\end{example}

\subsubsection{\glsentryfull{1qfa2}}
The \gls{1qfa2} model introduces a second observable to the \gls{mo-1qfa} framework, enhancing the capacity to distinguish input strings. It can be seen as an intermediate enhancement over classical \gls{mo-1qfa}.

\begin{definition}[\gls{1qfa2}]
A \gls{1qfa2} is defined similarly to a standard \gls{mo-1qfa}, but it includes two projective measurements applied in alternation during computation:
\[
A = (Q, \Sigma, \rho_0, \{U_{\sigma}\}, \{P_1, P_2\}, Q_{acc})
\]
where the two measurements \( P_1 \) and \( P_2 \) alternate throughout the computation.
\end{definition}

\paragraph{Strings Acceptance}
A string is accepted based on the final outcome after alternating between the two measurements. Acceptance is typically defined with bounded error.

\paragraph{Sets of Languages Accepted}
The class of languages recognised is still a proper subset of \glspl{reg}, although strictly more than \gls{mo-1qfa}.

\paragraph{Closure Properties}
These automata do not have known closure under union or intersection.

\paragraph{Advantages and Limitations}
The addition of a second observable enables more refined discrimination of inputs, albeit still with limited computational power.

\paragraph{Comparison}
The model is more expressive than \gls{mo-1qfa} but remains less powerful than general \gls{mm-1qfa}.

\paragraph{Additional Topics}
This line of work aligns with the broader research aim of incrementally extending the expressive power of \gls{1qfa} models, as seen in the approach of \cite{ciamarra2001quantum}.

\subsubsection{\glsentryfull{2t1qfa2}}
The \gls{2t1qfa2} model incorporates two tapes and two heads, enabling cross-comparison between symbols of the input and reference tape, enhancing the recognition of complex patterns.

\begin{definition}[\gls{2t1qfa2}]
Formally, a \gls{2t1qfa2} is a 7-tuple \[ A = (Q, \Sigma, \delta, q_0, Q_{acc}, Q_{rej}, \mathbb{T}), \] where \( \mathbb{T} \) denotes the tape set. The quantum evolution occurs in tandem over both tapes with corresponding heads.
\end{definition}

\paragraph{Strings Acceptance}
Acceptance is determined through measurement at halting, typically with bounded error.

\paragraph{Sets of Languages Accepted}
\gls{2t1qfa2} can recognise some non\glspl{reg}, exceeding the capabilities of classical one-way automata and standard \gls{1qfa} models \cite{ganguly20162}.

\paragraph{Closure Properties}
Closure properties remain largely unexplored, with no known results under union or complement.

\paragraph{Advantages and Limitations}
The addition of a second tape expands the computational power substantially, though it also introduces practical complexity.

\paragraph{Comparison}
Outperforms classical and most one-way quantum models in expressive power.

\paragraph{Additional Topics}
This model exemplifies a practical direction in extending \gls{1qfa} expressivity by architectural enhancement.

\subsubsection{\glsentryfull{nqfa}}
\gls{nqfa} introduce nondeterminism in the quantum setting. Unlike probabilistic nondeterminism, here the nondeterminism arises from quantum measurement outcomes and amplitude branches.

\begin{definition}[\gls{nqfa}]
An \gls{nqfa} is a 5-tuple \( A = (Q, \Sigma, \psi_0, \{U_\sigma\}, Q_{acc}) \), and uses a cutpoint acceptance mode, where a string \( x \) is accepted if \( \mathbb{P}(x) > \lambda \) for some threshold \( \lambda \).
\end{definition}

\paragraph{Strings Acceptance}
\gls{nqfa} recognise strings based on acceptance with nonzero amplitude, allowing acceptance of non\glspl{reg} with bounded error \cite{yakaryilmaz2009languages}.

\paragraph{Sets of Languages Accepted}
This model strictly recognises more than \glspl{reg}, offering power comparable to or exceeding classical nondeterministic finite automata.

\paragraph{Closure Properties}
\gls{nqfa} are not closed under complement or union.

\paragraph{Advantages and Limitations}
The model is powerful yet lacks constructive methods for deterministic acceptance, making some verification tasks harder.

\paragraph{Comparison}
More expressive than \gls{mo-1qfa}, \gls{mm-1qfa}, and even probabilistic finite automata in some cases.

\subsubsection{\glsentryfull{revqfa}}
\gls{revqfa} enforce reversibility in state transitions, in line with quantum mechanics principles, where computation steps are invertible.

\begin{definition}[\gls{revqfa}]
Defined similarly to \gls{mo-1qfa}, but all transitions are reversible, and state evolution is enforced to be unitary across all paths. Measurement occurs only at the end.
\end{definition}

\paragraph{Strings Acceptance}
A string is accepted based on the final state post unitary evolution.

\paragraph{Sets of Languages Accepted}
Recognises all \glspl{reg} \cite{yamakami2014one}, differing from many other \gls{1qfa} models.

\paragraph{Closure Properties}
Closed under Boolean operations due to equivalence with classical deterministic automata.

\paragraph{Advantages and Limitations}
They enjoy a strong correspondence to classical reversible automata with quantum efficiency benefits, though do not surpass \glspl{reg}.

\paragraph{Comparison}
Unlike most \gls{1qfa}, \gls{revqfa} can simulate any DFA, thus closing the gap between quantum and classical finite automata in expressiveness.

\paragraph{Additional Topics}
The interest in \gls{revqfa} stems from the desire to harness quantum reversibility for computation, a core direction explored by \cite{ciamarra2001quantum}.
